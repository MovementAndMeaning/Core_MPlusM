\ProvidesFile{matchSyntax.tex}[v1.0.0]
\appendixStart[ServiceMatchSyntax]{\textitcorr{Service~Match~Syntax}}%
Services are matched against 'expressions', which consist of one or more
'constraints'.\\
'constraints' consist of one or more 'fields with values'.\\
'fields with values' consist of a 'field name' and either a single value or a list of
values.\\
A 'field name' is one of the names recognized as being in the internal database, which is
described in the\\
\appendixRef{InternalDatabaseStructure}{Internal~Database~Structure} appendix.\\

The following BNF statements describe the syntax in more precise terms:
\outputBegin{}
<Expression> ::= <Constraint> ',' <Expression> | <Constraint>\\
;; Expressions are the union of the constraints.\\

<Constraint> ::= <FieldWithValues> '\&' <Constraint> | <FieldWithValues>\\
;; Constraints are the 'AND' or intersection of the field matches.\\

<FieldWithValues> ::= <FieldName> <Separator> <MatchValue>\\
<Separator> ::= ':' | ' '\\
;; Whitespace is optional between elements, but can act as a separator as\\
;; well.\\

<MatchValue> ::= <SingleValue> | '(' <ValueList> ')'\\
<ValueList> ::= <SingleValue> ',' <ValueList> | <SingleValue>\\
;; The comma used to separate values doesn't get confused with the comma\\
;; used between constraints. A value list is the 'OR' or union of the\\
;; values within the list.\\

\settowidth{\utilLen}{<FieldName> ::= }%
<FieldName> ::= description | details | executable | input | keyword |\\
\hspace*{\utilLen}name | output | channelname | request |\\
\hspace*{\utilLen}requestsdescription | version\\
;; The field names are case-insensitive and correspond to the columns in\\
;; the internal database.\\

\settowidth{\utilLen}{<SingleValue> ::= }%
<SingleValue> ::= <DoubleQuote> <String> <DoubleQuote> |\\
\hspace*{\utilLen}<SingleQuote> <String> <SingleQuote> | <String>\\
<DoubleQuote> ::= '"'\\
<SingleQuote> ::= "'"\\
;; If the matching value includes an unescaped double quote character, it\\
;; should be surrounded by single quote characters and, if it includes an\\
;; unescaped single quote character, it should be surrounded by double\\
;; quote characters. Otherwise, the use of quote characters is optional.\\
;; Note that, if white space is part of the matching value, the white\\
;; space needs to be quoted or the matching value should be surrounded by\\
;; quote characters.\\
\newpage
<String> ::= <Character> <String> | <Character>\\
\settowidth{\utilLen}{<Character> ::= }%
<Character> ::= <Escape> <Reserved> | <WildSpan> | <WildSingle> |\\
\hspace*{\utilLen}<Normal>\\
<Escape> ::= '\textbackslash'\\
\settowidth{\utilLen}{<Reserved> ::= }%
<Reserved> ::= <DoubleQuote> | <SingleQuote> | <WildSpan> |\\
\hspace*{\utilLen}<WildSingle> | <Escape>\\
<WildSpan> ::= '*'\\
<WildSingle> ::= '?'\\
<Normal> ::= a non-reserved character\\
;; Strings consist of one or more characters. The 'wild span' character\\
;; will match zero or more characters and the 'wild single' character\\
;; will match a single character.
\outputEnd{}

\objDiagram{matchSyntaxrails.ps}{matchSyntaxrails}{Service Match Syntax diagram}

\appendixEnd{}
