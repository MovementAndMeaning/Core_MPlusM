\ProvidesFile{OSservice.tex}[v1.0.0]
\primaryStart[TheOSIService]{The~\OSI{}~Service}
The \asCode{m+mOpenStageInputService} application is an Input service,
generating a stream of information on the position and orientation of one or more bodies.
The application responds to the standard Input service requests and can be used as a
standalone data generator, without the need for a client connection.\\

The \emph{configure} request has two arguments \longDash{} a string value for the host
name of the Organic Motion OpenStage controller and an integer value for the port on the
Organic Motion OpenStage controller to connect to.
These values will be applied when the output stream is started or restarted.\\ 

The \emph{restartStreams} request stops and then starts the output stream.\\

The \emph{startStreams} request request initiates listening to the Organic Motion
OpenStage controller.
Once started, the service will send groups of body data via the output \yarp{} network
connection.\\

The \emph{stopStreams} request stops the Organic Motion OpenStage controller listener,
which stops the output \yarp{} network connection.\\ 

Note that the application will exit if the \emph{\RS} is not running.
\secondaryStart[StartingFromCommandLine]{Starting from the command\longDash{}line}
The application has two optional arguments \longDash{} the host name and port for the
Organic Motion OpenStage controller.
\insertAppParameters
\insertTagDescription{\OSI}
\insertInputServiceComment\\

If the application is running from a terminal and has not been automatically started via
the `\asCode{go}' option, the following commands are available:
\begin{itemize}
\item\cmdItem{?}{display a list of the available commands}
\item\exSp\cmdItem{b}{start the output stream, sending body data}
\item\exSp\cmdItem{c}{configure the service by providing the host name and port for the
Organic Motion OpenStage controller}
\item\exSp\cmdItem{e}{stop the output stream}
\item\exSp\cmdItem{q}{quit the application}
\item\exSp\cmdItem{r}{restart the output stream}
\item\exSp\cmdItem{u}{reset the configuration so that it will be reprocessed when the
output stream is restarted}
\end{itemize}
\secondaryEnd
\condPage
\secondaryStart[StartingFromMMManager]{Starting from \emph{\MMMU}}
If the service is selected for execution from within the \emph{\MMMU}
application, the following dialog will be presented:
\objScaledDiagram{mpm_images/launchOpenStageInputService}%
{launchServiceOpenStage}{Launch options for the \emph{\OSI} service}{0.8}

For details on the usage of the `endpoint' and `tag' options, see the \emph{\MMMU} manual.
Once the options are set, the following image will appear in the \emph{\MMMU} window when
the service has successfully started:
\objScaledDiagram{mpm_images/runningOpenStageInputService}%
{serviceRunningOpenStage}{The \emph{\MMMU} entity for the \emph{\OSI} service}{1.0}
\secondaryEnd
\primaryEnd
\primaryStart[TheOSIService]{The~\OSBI{}~Service}
The \asCode{m+mOpenStageBlobInputService} application is an Input service,
generating a stream of information on the position and orientation of one or more bodies.
The application responds to the standard Input service requests and can be used as a
standalone data generator, without the need for a client connection.\\

The \emph{configure} request has three arguments \longDash{} a floating\longDash{}point
value for the translation scale to apply to the joint coordinates, a string value for
the host name of the Organic Motion OpenStage controller and an integer value for the port
on the Organic Motion OpenStage controller to connect to.
These values will be applied when the output stream is started or restarted.\\ 

The \emph{restartStreams} request stops and then starts the output stream.\\

The \emph{startStreams} request request initiates listening to the Organic Motion
OpenStage controller.
Once started, the service will send groups of body data via the output \yarp{} network
connection.\\

The \emph{stopStreams} request stops the Organic Motion OpenStage controller listener,
which stops the output \yarp{} network connection.\\ 

Note that the application will exit if the \emph{\RS} is not running.
\secondaryStart[StartingFromCommandLine]{Starting from the command\longDash{}line}
The application has three optional arguments \longDash{} the translation scale to be
applied to the coordinates and the host name and port for the Organic Motion OpenStage
controller.
\insertAppParameters
\insertTagDescription{\OSBI}
\insertInputServiceComment\\

If the application is running from a terminal and has not been automatically started via
the `\asCode{go}' option, the following commands are available:
\begin{itemize}
\item\cmdItem{?}{display a list of the available commands}
\item\exSp\cmdItem{b}{start the output stream, sending body data}
\item\exSp\cmdItem{c}{configure the service by providing the translation scale to be used
and the host name and port for the Organic Motion OpenStage controller}
\item\exSp\cmdItem{e}{stop the output stream}
\item\exSp\cmdItem{q}{quit the application}
\item\exSp\cmdItem{r}{restart the output stream}
\item\exSp\cmdItem{u}{reset the configuration so that it will be reprocessed when the
output stream is restarted}
\end{itemize}
\secondaryEnd
\condPage
\secondaryStart[StartingFromMMManager]{Starting from \emph{\MMMU}}
If the service is selected for execution from within the \emph{\MMMU}
application, the following dialog will be presented:
\objScaledDiagram{mpm_images/launchOpenStageBlobInputService}%
{launchServiceOpenStageBlob}{Launch options for the \emph{\OSBI} service}{0.8}

For details on the usage of the `endpoint' and `tag' options, see the \emph{\MMMU} manual.
Once the options are set, the following image will appear in the \emph{\MMMU} window when
the service has successfully started:
\objScaledDiagram{mpm_images/runningOpenStageBlobInputService}%
{serviceRunningOpenStageBlob}{The \emph{\MMMU} entity for the \emph{\OSBI} service}{1.0}
\secondaryEnd
\primaryEnd{}
