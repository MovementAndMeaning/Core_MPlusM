\ProvidesFile{CMnoYarp.tex}[v1.0.0]
\appendixStart{\textitcorr{What to do if \textbf{\yarp} is not running}}
If \yarp{} is not running when the \emph{Channel Manager} application is launched and the
\yarp{} executable can't be found, the following will be displayed:
\objScaledDiagram{mpm_images/noYarpFound}{noYarpFound}%
{\textbf{YARP} not running and could not be located}{1.0}

At this point, it's best to exit the \emph{Channel Manager} application and reinstall
\mplusm, since the installation is likely to have become damaged.
If, instead, the following is displayed:
\objScaledDiagram{mpm_images/noRunningYarp}{noRunningYarp}%
{\textbf{YARP} not running when application launched}{0.85}

then a \yarp{} executable was found and can be launched.
If the \textbf{No} button is clicked, you should exit the \emph{Channel Manager}
application and either manually start the \yarp{} server locally or update the \yarp{}
configuration to refer to an available \yarp{} server.
\condPage{}
If, instead, the \textbf{Yes} button is clicked, the following will be displayed:
\objScaledDiagram{mpm_images/launchYarp}{launchYarp}%
{Launching a private \textbf{YARP} server}{1.0}

Note that only valid IP addresses or network ports can be selected; clicking on the
\textbf{Cancel} button will exit the dialog box, with no \yarp{} server launched, while
clicking on the \textbf{OK} button will start a new \yarp{} server in the background, that
will be stopped when the \emph{Channel Manager} application exits.\\

If you have chosen to launch a new \yarp{} server, you will then be asked about launching
the \emph{Registry Service}, as described in the
\appendixRef{RegistryServiceNotRunning}{What to do if the \emph{Registry Service} is not
 running} appendix.
\appendixEnd{}
