\ProvidesFile{MpMcaseStudy.tex}[v1.0.0]
\appendixStart[CaseStudy]{\textitcorr{Case~Study:\ LeapMotionInputService}}
The Leap Motion Controller is a sensor that can report the position and orientation of one
or more hands or cylindrical tools, such as pointers, pens or pencils.
It provides detailed information on each finger and the bones that constitute them, as
well as the overall parameters of a hand.\\

The SDK \textbraceleft\companyReference{https://developer.leapmotion.com}{https://%
developer.leapmotion.com}\textbraceright{} is readily available for Windows and Macintosh
OS~X, so implementing an Input Service for a Leap Motion Controller should be
straightforward.\\

With the Leap Motion API, there are two approaches to gathering the data from the
controller \longDash{} polling or responding to an event.
As well, if the event\longDash{}handling mechanism is chosen, the service can either use
multiple inheritance with \asCode{Leap::Listener} on the Input service class, or use a
separate \asCode{Leap::Listener} subclass to respond to the events.\\

For simplicity, we will use event\longDash{}handling with a separate
\asCode{Leap::Listener} subclass, and a namespace of\\
\asCode{LeapMotion}.
The steps being described are for Macintosh~OS~X; the section
\appendixRef{UsingExemplars}{Using~the~Exemplar~Files} describes the steps that would be
involved in building the Leap Motion Input Service on Windows.
\secondaryStart{Implementation}
\tertiaryStart{First~step}
Following the steps outlined in \appendixRef{UsingExemplars}{Using~the~Exemplar~Files},
we make a duplicate of the \asCode{Exemplars} directory, named \asCode{LeapMotion}.\\

We then rename the source files, replacing `\asCode{Exemplar}' with `\asCode{LeapMotion}'. 
\tertiaryEnd
\tertiaryStart{Second~step}
Within each file, we replace `\asCode{Exemplar}' with `\asCode{LeapMotion}', and edit the
comments to refer to Leap Motion.
At this point, we can build the Input service, but it does not actually connect to the
Leap Motion controller.
\tertiaryEnd
\tertiaryStart{Third~step}
We now install the Leap Motion API, copying the header files \asCode{Leap.h} and
\asCode{LeapMath.h} into our directory.
There is also a dynamic library to be copied and, for the Macintosh, this is
\asCode{libLeap.dylib}; we will edit the\\
\asCode{LeapMotionInputService/CMakeLists.txt} file to include this in the
`\asCode{target\textunderscore{}link\textunderscore{}libraries}' statement.
Running the `\asCode{ccmake~.}', `\asCode{cmake~.}' and `\asCode{make~clean~\&\&~make}'
commands again will verify that we are linking properly with the dynamic library.\\

Since we've decided to use a separate \asCode{Leap::Listener} subclass, rather than
polling the Leap Motion controller, we no longer need the \asCode{LeapMotionInputThread}
class or it's files \longDash{} the files should be removed from the project and the\\
\asCode{LeapMotionInputService/CMakeLists.txt} file; the references to the
\asCode{LeapMotionInputThread} class in the\\
\asCode{LeapMotionInputService} should be commented out, as they will be replaced soon.\\

Now we should confirm that the Input service builds.
\tertiaryEnd
\tertiaryStart{Fourth~step}
Add a private member variable of type \asCode{Leap::Controller~*} to the
\asCode{LeapMotionInputService} class, and initialized it with
`\asCode{new~Leap::Controller}' in the class constructor.
Make sure to delete it in the class destructor.\\

In the \asCode{LeapMotionInputService::startStreams} and
\asCode{LeapMotionInputService::stopStreams} methods, add `\asCode{if}' statements to
check for the non\longDash\asCode{NULL} state of the new member variable.\\

In the \asCode{LeapMotionInputService::setUpStreamDescriptions} method, change the
port protocol string to\\
`\asCode{LEAP}'.
\tertiaryEnd
\tertiaryStart{Fifth~step}
We now create a new \compLang{C++} class, \asCode{LeapMotionInputListener}, that derives
from \asCode{Leap::Listener} and add the source file to the
\asCode{LeapMotionInputService/CMakeLists.txt} file \longDash{} the class should be in the
\asCode{LeapMotion} namespace.\\

Add the header file \asCode{M+MLeapMotionInputListener.h} to the file
\asCode{M+MLeapMotionInputService.cpp} and replace the references to
\asCode{LeapMotionInputThread} with \asCode{LeapMotionInputListener} in the\\
\asCode{LeapMotionInputService} that were commented out.\\

In the \asCode{LeapMotionInputListener} class, add `placeholder' functions for, at least,
the virtual method `\asCode{onFrame}' \longDash{} this will receive the motion events from
the Leap Motion controller.
Add a private member variable of type\\
\asCode{Common::GeneralChannel~*} to the class, which will be used to connect to one of
the Input service output channels.
Add an argument of the same type to the class constructor, and have the constructor set
the member variable to the argument provided.\\

In the \asCode{LeapMotionInputService::startStreams} method, within the `\asCode{if}'
statement that was added earlier, add an assignment statement to set this new member
variable to the value\\
`\asCode{new~LeapMotionInputListener(\textunderscore{}outStreams.at(0))}' and
follow it with `\asCode{XXX->addListener(*YYYY}',\\
where `\asCode{XXX}' is the member variable of type \asCode{Leap::Controller~*} that we
added in an earlier step and `\asCode{YYY}' is the member variable that we just added.\\

In the \asCode{LeapMotionInputService::stopStreams} method, within the `\asCode{if}'
statement that was added earlier, add the statements
`\asCode{XXX->removeListener(*YYY);}', '\asCode{delete~YYY;}' and \asCode{YYY~=~NULL;}',
where `\asCode{XXX}' and `\asCode{YYY}' are as above.
\tertiaryEnd
\tertiaryStart{Sixth~step}
In the \asCode{LeapMotionInputListener} class, add the public method
\asCode{clearOutputChannel} and have it set the private member variable of type
\asCode{Common::GeneralChannel~*} to \asCode{NULL}.\\

In the \asCode{LeapMotionInputService::shutDownOutputStreams} method, add the statement\\
`\asCode{YYY->clearOutputChannel();}', where `\asCode{YYY}' is the member variable of type
\asCode{Common::GeneralChannel~*} that we added in the previous step.\\

Now we should confirm that the Input service builds, and it should be possible to launch
it, if the Leap Motion controller is plugged into the computer.
\tertiaryEnd
\tertiaryStart{Seventh~step}
Now, fill in the body of the \asCode{LeapMotionInputListener::onFrame} method, using the
\asCode{Leap::Controller~\&} argument to gain access to the event data (using its
\asCode{frame} method) and sending output via `\asCode{YYY->write(\textellipsis)}', where
`\asCode{YYY}' is the member variable of type \asCode{Common::GeneralChannel~*} that we
added in an earlier step.\\

The details are left as an exercise for the reader.
\tertiaryEnd
\secondaryEnd
\appendixEnd{}
