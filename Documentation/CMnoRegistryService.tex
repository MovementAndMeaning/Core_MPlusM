\ProvidesFile{CMnoRegistryService.tex}[v1.0.0]
\appendixStart[RegistryServiceNotRunning]{\textitcorr{What to do if the \emph{Registry
Service} is not running}}
If the \emph{Registry Service} is not running when the \emph{Channel Manager} application
is launched and the \emph{Registry Service} executable can't be found, the following will
be displayed:
\objScaledDiagram{mpm_images/noRegistryServiceFound}{noRegistryServiceFound}%
{\emph{Registry Service} not running and could not be located}{1.0}

At this point, it's best to exit the \emph{Channel Manager} application and reinstall
\mplusm, since the installation is likely to have become damaged.
If, instead, the following is displayed:
\objScaledDiagram{mpm_images/noRunningRegistryService}{noRunningRegistryService}%
{\emph{Registry Service} not running when application launched}{0.85}

then a \emph{Registry Service} executable was found and can be launched.
If the \textbf{No} button is clicked, you should exit the \emph{Channel Manager}
application and either manually start the \emph{Registry Service} locally or update the
\yarp{} configuration to refer to an available \yarp{} server that has a running
\emph{Registry Service}.
\condPage{}
If, instead, the \textbf{Yes} button is clicked, the following will be displayed:
\objScaledDiagram{mpm_images/launchRegistryService}{launchRegistryService}%
{Launching the \emph{Registry Service}}{1.0}

Note that only valid network ports can be entered, and it is usually not necessary to use
a non-default port; clicking on the \textbf{Cancel} button will exit the dialog box, with
no \emph{Registry Service} launched, while clicking on the \textbf{OK} button will start a
new \emph{Registry Service} in the background, that will be stopped when the
\emph{Channel Manager} application exits.
\appendixEnd{}
