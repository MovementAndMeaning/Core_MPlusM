\ProvidesFile{SMservice.tex}[v1.0.0]
\primaryStart[TheSMOService]{The~\SMO{}~Service}
The \asCode{m+mSendToMQOutputService} application is an Output service, sending a stream
of \yarp{} values as \json{} text to an ActiveMQ broker.
The application responds to the standard Output service requests and can be used as a
standalone data sink, without the need for a client connection.\\

The \emph{configuration} request has no arguments and returns the string value for the
topic or queue to set up on the ActiveMQ broker and an integer value for the `queue flag'.
A value of \asBoldCode{0} for the `queue flag' indicates that a topic will be used for
sending to the ActiveMQ broker and a value of \asBoldCode{1} indicates that a queue will
be used.\\

The \emph{configure} request request has two arguments \longDash{} a string value for the
topic or queue to set up on the ActiveMQ broker and an integer value for the `queue flag'.
These values will be applied when the input stream is started or restarted.\\

The \emph{restartStreams} request stops and then starts the input stream.\\

The \emph{startStreams} request initiates listening on the input stream and creates a
connection to the ActiveMQ broker.\\

The \emph{stopStreams} request terminates listening ont the input stream and drops the
connection to the ActiveMQ broker.\\ 

Note that the application will exit if the \emph{\RS} is not running.
\secondaryStart[StartingFromCommandLine]{Starting from the command\longDash{}line}
The application has four required arguments \longDash{} a string value for the host name
of the ActiveMQ broker, an integer value for the port on the ActiveMQ broker to
connect to, a string value for the user identifier to use on the ActiveMQ broker and a
string value for the password for the user on the ActiveMQ broker.
The application has two optional arguments \longDash{} a string value for the
topic or queue to set up on the ActiveMQ broker and a flag indicating whether a queue will
be used to send to the ActiveMQ broker.
\insertAppParameters
\insertTagDescription{\SMO}
\insertInputServiceComment

The other parameters provide the topic or queue name to set up on the ActiveMQ broker and
the queue flag; if not specified, a topic or queue name of \asCode{m+m} and a queue flag
of \asCode{0} \longDash{} indicating sending of the data via a topic \longDash{} will be
used.
Note that the topic or queue name and queue flag can also be set via commands, if the
application is running from a terminal.\\

\insertStandardServiceCommands
\secondaryEnd
\condPage
\secondaryStart[StartingFromMMManager]{Starting from the \emph{\MMMU} application}
If the service is selected for execution from within the \emph{\MMMU} application, the
following dialog will be presented:
\objScaledDiagram{mpm_images/launchSendToMQOutputService}%
{launchServiceSendToMQ}{Launch options for the \emph{\SMO} service}{0.8}
\condPage{}
For details on the usage of the `endpoint' and `tag' options, see the \emph{\MMMU} manual.
Once the options are set, the following image will appear in the \emph{\MMMU} window when
the service has successfully started:
\objScaledDiagram{mpm_images/runningSendToMQOutputService}%
{serviceRunningSendToMQ}{The \emph{\MMMU} entity for the \emph{\SMO} service}{1.0}

If the \textbf{Configure the service} menu item has been selected, the following will
appear:
\objScaledDiagram{mpm_images/configureSendToMQOutputService}%
{configureServiceSendToMQ}{Configuration window for the \emph{\SMO} service}{1.0}

Note that the queue flag can be specified by \asBoldCode{0}, \asBoldCode{f} or
\asBoldCode{n} for sending the data via a topic and \asBoldCode{1}, \asBoldCode{t} or
\asBoldCode{y} for sending the data via a queue.
\secondaryEnd
\primaryEnd{}
