\ProvidesFile{sharedSetup.tex}[v1.0.0]
\documentclass[letterpaper,titlepage,twoside]{report}

\usepackage{times}
\usepackage{amsmath}
\usepackage{enumerate}
\usepackage{ifthen}
\usepackage{alltt}
\usepackage{calc}
\usepackage{shortvrb}
\usepackage{varioref}
\usepackage{graphicx}
\usepackage{color}
\usepackage{makeidx}
\usepackage{xspace}
\usepackage{fancyhdr}
\usepackage[section]{tocbibind}
\usepackage{epstopdf}

% Control the code, depending on whether a hyper-linked PDF is being generated:
\newboolean{generatingHyperPDF}
\setboolean{generatingHyperPDF}{true}

% If the package 'hyperref' is disabled by commenting out the following lines,
% be sure to set the boolean 'generatingHyperPDF' to false.
\ifthenelse{\boolean{generatingHyperPDF}}%
 {\usepackage[colorlinks=true,
    linkcolor=webgreen,
    filecolor=webbrown,
    citecolor=webgreen,
    urlcolor=webblue,
    pdftitle={\docTitle},
    pdfauthor={\docAuthor},
    pdfkeywords={\docKeywords},
    pdfsubject={\docSubject},
    bookmarks,
    raiselinks=true,
    plainpages=false,
    bookmarksopen=true,
    pdfstartview=Fit,
    pdfpagemode=UseOutlines]{hyperref}}
 {\newcommand{\hyperpage}[1]{#1}}

\usepackage{mysects}

% Adjust the paper edges:
\setlength{\parindent}{0em}
\setlength{\textwidth}{\paperwidth-144pt}% 2"
\setlength{\marginparsep}{0pt}
\setlength{\marginparwidth}{0pt}
\setlength{\evensidemargin}{-18pt}% 0.25"
\setlength{\oddsidemargin}{-18pt}% 0.25"

% Some colours for the web:
\definecolor{webgreen}{rgb}{0,0.5,0}
\definecolor{webbrown}{rgb}{0.6,0,0}
\definecolor{webblue}{rgb}{0,0,0.5}

% Define some common strings:
\newcommand{\default}{!-!-!}
\newcommand{\MMM}{Movement~and~Meaning~Middleware}
\newcommand{\mplusm}{\textbf{M+M}}
\newcommand{\nothing}{\ }
\newcommand{\yarp}{\begin{footnotesize}\textsf{YARP}\end{footnotesize}}

% Set the float behaviour:
\setcounter{bottomnumber}{2}
\setcounter{totalnumber}{4}

% Suppress the normal numbering of sections, et cetera:
\setcounter{secnumdepth}{-3}
\setcounter{tocdepth}{2}

% A couple of useful commands to handle italic-to-normal transitions:
\newcommand{\textitcorr}[1]{\textit{#1}\/}
\newcommand{\emphcorr}[1]{\emph{#1}\/}

% Some commands to make examples match actual output more closely:
\newcommand{\pseudotab}{\thinspace\boldmath{$\vdash$}\thinspace}
\newcommand{\pseudotwotabs}{\thinspace\boldmath{$\vdash$}\thinspace\boldmath{$\vdash$}\thinspace}
\newcommand{\clientName}{\textunderscore{}client\textunderscore}
\newcommand{\dollarService}{\textdollar{}ervice}
\newcommand{\serviceName}{\textunderscore{}service\textunderscore}
\newcommand{\inputOutput}{Input~/~Output}
\newcommand{\openSq}{\boldmath{$\lbrack$}}
\newcommand{\closeSq}{\boldmath{$\rbrack$}}
\newcommand{\sqPair}{\boldmath{$\lbrack{}\ \rbrack$}}
\newcommand{\fatUnderscore}{\begin{Large}\textbf{\textunderscore}\end{Large}}
\newlength{\utilLen}

\newcommand*{\compLang}[1]{\emphcorr{#1}}

\newcommand{\objCaption}[1]{\caption{#1\ }}
% Create a framed box as a figure, centred horizontally (note that we can't use \textttt{}, as the third argument
% could be quite large):
\newcommand{\objFileDescription}[4][0.75]{\begin{figure}[!ht]%
\setlength{\fboxsep}{3pt}\setlength{\fboxrule}{0.75bp}%
\begin{center}\fbox{\begin{minipage}[t]{#1\textwidth}\begin{flushleft}\footnotesize\ttfamily #4\end{flushleft}%
\end{minipage}}\objCaption{#2}\label{file:#3}\end{center}\end{figure}}

\newcommand*{\objPicture}[1]{\begin{center}\includegraphics{#1}\end{center}}
\newcommand*{\objScaledPicture}[2]{\begin{center}\includegraphics[scale=#2]{#1}\end{center}}
\newcommand*{\objDiagram}[3]{\begin{figure}[!ht]\objPicture{#1}\objCaption{#3}\label{diagram:#2}\end{figure}}

% First argument: index category
% Second argument: index subcategory
% Third argument: alternate index subcategory
% Fourth argument: index name
% Fifth argument: index suffix 
\newcommand{\multiindex}[5]{%
  \ifthenelse{\equal{#2}{\default{}}}%
    {\index{#1!#4#5}}%
    {\index{#1!#2!#4#5}}}

% First argument: label category
% Second argument: label subcategory
% Third argument: alternate label subcategory
% Fourth argument: label name
\newcommand{\multilabel}[4]{%
  \ifthenelse{\equal{#3}{\default{}}}%
    {\label{#1:#4}}%
    {\label{#1:#3:#4}}}

% First argument: reference category
% Second argument: reference subcategory
% Third argument: alternate reference subcategory
% Fourth argument: reference name
\newcommand{\multiref}[4]{%
  \ifthenelse{\equal{#3}{\default{}}}%
    {\ref{#1:#4}}%
    {\ref{#1:#3:#4}}}

% First argument: hyperlink/index category
% Second argument: hyperlink/index name
% Third argument: alternate hyperlink/index name
\ifthenelse{\boolean{generatingHyperPDF}}%
  {\newcommand{\genTag}[3]{\hyperlink{hyper.#1.#2}{\textitcorr{#3}}}}%  command if generatingHyperPDF
  {\newcommand{\genTag}[3]{\textitcorr{#3}}}%  command if not generatingHyperPDF
  
% First argument: hyperlink/index category
% Second argument: hyperlink/index name
\ifthenelse{\boolean{generatingHyperPDF}}%
  {\newcommand{\genTarget}[2]{\hypertarget{hyper.#1.#2}{}}}%  command if generatingHyperPDF
  {\newcommand{\genTarget}[2]{}}%  command if not generatingHyperPDF

% The net effect is as follows:
%  generatingHyperPDF
%   'D' \textitcorr{\color{webgreen}#3}  \hypertarget{hyper.#2.#3}{}\label{#2:#3}  {}
%   'E' {}  {}  {}
%	'M' {} \hypertarget{hyper.#2.#3}{}\label{#2:#3} {}
%   'P' {} \hypertarget{hyper.#2.#3}{}\label{#2:#3} \index{#2!#3}
%   'R' \hyperlink{hyper.#2.#3}{\textitcorr{#3}}  \index{#2!#3}  \ref{#2:#3}
%   'S' \textitcorr{#3}  \index{#2!#3}  {}
%   'X' \hyperlink{hyper.#2.#3}{\textitcorr{#3}}  {}  {}
%  not generatingHyperPDF
%   'D' \textitcorr{\color{webgreen}#3} \index{#2!#3|(textbf}\label{#2:#3}  {}
%   'E' {} \index{#2!#3|)textbf}  {}
%   'M' {} \index{#2!#3|(textbf}\label{#2:#3}  {}
%   'P' {} \label{#2:#3} \index{#2!#3}
%   'R' \textitcorr{#3}  \index{#2!#3}  \ref{#2:#3}
%   'S' \textitcorr{#3}  \index{#2!#3}  {}
%   'X' \textitcorr{#3}  {}  {}

%   D = Define the object (emphasize the index, create a label);
%   E = End of the object definition (close the index, no text);
%	M = Define the object (no visible text)
%   P = 
%   R = Refer to the object in the index (the default);
%   S = Reference to a standard object and
%   X = Don't add a reference for the object to the index (any letter except D or
%         R could be used, X is preferred for mnemonic value)

% First argument: hyperlink/index category
% Second argument: hyperlink/index subcategory
% Third argument: alternate hyperlink/index subcategory
% Fourth argument: hyperlink/index name
% Fifth argument: alternate hyperlink/index name
\newcommand{\entityNameD}[5]{%
  \genTarget{#1}{#4}%
  \ifthenelse{\equal{#5}{\default{}}}%
    {\textitcorr{\color{webgreen}#4}\multiindex{#1}{#2}{#3}{#4}{|(textbf}}%
    {\textitcorr{\color{webgreen}#5}\multiindex{#1}{#2}{#3}{#5}{|(textbf}}%
  \multilabel{#1}{#2}{#3}{#4}}

% First argument: hyperlink/index category
% Second argument: hyperlink/index subcategory
% Third argument: alternate hyperlink/index subcategory
% Fourth argument: hyperlink/index name
% Fifth argument: alternate hyperlink/index name
\newcommand{\entityNameE}[5]{%
  \ifthenelse{\equal{#1}{#4}}%
    {}% if first and fourth argument match
    {\ifthenelse{\equal{#5}{\default{}}}%
      {\multiindex{#1}{#2}{#3}{#4}{|)textbf}}%
      {\multiindex{#1}{#2}{#3}{#5}{|)textbf}}}}

% First argument: hyperlink/index category
% Second argument: hyperlink/index subcategory
% Third argument: alternate hyperlink/index subcategory
% Fourth argument: hyperlink/index name
% Fifth argument: alternate hyperlink/index name
\newcommand{\entityNameM}[5]{%  command if not generatingHyperPDF
  \genTarget{#1}{#4}%
  \ifthenelse{\equal{#5}{\default{}}}%
    {\multiindex{#1}{#2}{#3}{#4}{|(textbf}}%
    {\multiindex{#1}{#2}{#3}{#5}{|(textbf}}%
  \multilabel{#1}{#2}{#3}{#4}}

% First argument: hyperlink/index category
% Second argument: hyperlink/index subcategory
% Third argument: alternate hyperlink/index subcategory
% Fourth argument: hyperlink/index name
% Fifth argument: alternate hyperlink/index name
\newcommand{\entityNameP}[5]{%  command if generatingHyperPDF
  \genTarget{#1}{#4}%
  \ifthenelse{\equal{#5}{\default{}}}%
	{#4\multiindex{#1}{#2}{#3}{#4}{|(textbf}}%
	{#5\multiindex{#1}{#2}{#3}{#5}{|(textbf}}%
  \multilabel{#1}{#2}{#3}{#4}}

% First argument: hyperlink/index category
% Second argument: hyperlink/index subcategory
% Third argument: alternate hyperlink/index subcategory
% Fourth argument: hyperlink/index name
% Fifth argument: alternate hyperlink/index name
\newcommand{\entityNameR}[5]{%  command if generatingHyperPDF
  \ifthenelse{\equal{#5}{\default{}}}%
    {\genTag{#1}{#4}{#4}%
    \ifthenelse{\equal{#1}{#4}}%
      {}% if first and fourth argument match
      {\multiindex{#1}{#2}{#3}{#4}{}}}%
    {\genTag{#1}{#4}{#5}%
    \ifthenelse{\equal{#1}{#4}}%
      {}% if first and fourth argument match
      {\multiindex{#1}{#2}{#3}{#5}{}}}%
  \multiref{#1}{#2}{#3}{#4}}

% First argument: hyperlink/index category
% Second argument: hyperlink/index subcategory
% Third argument: alternate hyperlink/index subcategory
% Fourth argument: hyperlink/index name
% Fifth argument: alternate hyperlink/index name
\newcommand{\entityNameS}[5]{%
  \ifthenelse{\equal{#5}{\default{}}}%
	{\textitcorr{#4}}%
	{\textitcorr{#5}}%
  \ifthenelse{\equal{#1}{#4}}%
    {}% if first and fourth argument match
    {\multiindex{#1}{#2}{#3}{#4}{}}}

% First argument: hyperlink/index category
% Second argument: hyperlink/index subcategory
% Third argument: alternate hyperlink/index subcategory
% Fourth argument: hyperlink/index name
% Fifth argument: alternate hyperlink/index name
\newcommand{\entityNameX}[5]{%  command if generatingHyperPDF
  \ifthenelse{\equal{#5}{\default{}}}%
    {\genTag{#1}{#4}{#4}}%
    {\genTag{#1}{#4}{#5}}}%

% First argument: hyperlink/index category
% Second argument: hyperlink/index name
\ifthenelse{\boolean{generatingHyperPDF}}%
  {\newcommand{\doVpage}[2]{}}%  command if generatingHyperPDF
  {\newcommand{\doVpage}[2]{ \vpageref[(][(]{#1:#2}}}%  command if not generatingHyperPDF
  
% Use \entityReference, rather than \entityName, for the first mention of an object within
% another object, so that page ranges will be present.
\newcommand{\entityReference}[2]{\entityNameR{#1}{\default{}}{\default{}}{#2}{\default{}}\doVpage{#1}{#2}}%  command if generatingHyperPDF

\ifthenelse{\boolean{generatingHyperPDF}}%
  {\newcommand{\companyReference}[2]{\href{#1}{#2}}}%  command if generatingHyperPDF
  {\newcommand{\companyReference}[2]{#2}}% command if not generatingHyperPDF

% First argument [optional]: alternate name
% Second argument: entity name
\newcommand{\clientNameD}[2][\default{}]{\entityNameD{Clients}{\default{}}{\default{}}{#2}{#1}}% shortcut
\newcommand{\clientNameE}[2][\default{}]{\entityNameE{Clients}{\default{}}{\default{}}{#2}{#1}}% shortcut
\newcommand{\clientNameM}[2][\default{}]{\entityNameM{Clients}{\default{}}{\default{}}{#2}{#1}}% shortcut
\newcommand{\clientNameP}[2][\default{}]{\entityNameP{Clients}{\default{}}{\default{}}{#2}{#1}}% shortcut
\newcommand{\clientNameR}[2][\default{}]{\entityNameR{Clients}{\default{}}{\default{}}{#2}{#1}}% shortcut
\newcommand{\clientNameS}[2][\default{}]{\entityNameS{Clients}{\default{}}{\default{}}{#2}{#1}}% shortcut
\newcommand{\clientNameX}[2][\default{}]{\entityNameX{Clients}{\default{}}{\default{}}{#2}{#1}}% shortcut

% First argument [optional]: alternate name
% Second argument: entity name
\newcommand{\serviceNameD}[2][\default{}]{\entityNameD{Services}{\default{}}{\default{}}{#2}{#1}}% shortcut
\newcommand{\serviceNameE}[2][\default{}]{\entityNameE{Services}{\default{}}{\default{}}{#2}{#1}}% shortcut
\newcommand{\serviceNameM}[2][\default{}]{\entityNameM{Services}{\default{}}{\default{}}{#2}{#1}}% shortcut
\newcommand{\serviceNameP}[2][\default{}]{\entityNameP{Services}{\default{}}{\default{}}{#2}{#1}}% shortcut
\newcommand{\serviceNameR}[2][\default{}]{\entityNameR{Services}{\default{}}{\default{}}{#2}{#1}}% shortcut
\newcommand{\serviceNameS}[2][\default{}]{\entityNameS{Services}{\default{}}{\default{}}{#2}{#1}}% shortcut
\newcommand{\serviceNameX}[2][\default{}]{\entityNameX{Services}{\default{}}{\default{}}{#2}{#1}}% shortcut

% First argument [optional]: alternate name
% Second argument: entity name
\newcommand{\utilityNameD}[2][\default{}]{\entityNameD{Utilities}{\default{}}{\default{}}{#2}{#1}}% shortcut
\newcommand{\utilityNameE}[2][\default{}]{\entityNameE{Utilities}{\default{}}{\default{}}{#2}{#1}}% shortcut
\newcommand{\utilityNameM}[2][\default{}]{\entityNameM{Utilities}{\default{}}{\default{}}{#2}{#1}}% shortcut
\newcommand{\utilityNameP}[2][\default{}]{\entityNameP{Utilities}{\default{}}{\default{}}{#2}{#1}}% shortcut
\newcommand{\utilityNameR}[2][\default{}]{\entityNameR{Utilities}{\default{}}{\default{}}{#2}{#1}}% shortcut
\newcommand{\utilityNameS}[2][\default{}]{\entityNameS{Utilities}{\default{}}{\default{}}{#2}{#1}}% shortcut
\newcommand{\utilityNameX}[2][\default{}]{\entityNameX{Utilities}{\default{}}{\default{}}{#2}{#1}}% shortcut

% First argument [optional]: alternate name
% Second argument: Subcategory
% Third argument: entity name
\newcommand{\examplesNameD}[3][\default{}]{\entityNameD{Examples}{#2}{#2}{#3}{#1}}% shortcut
\newcommand{\examplesNameE}[3][\default{}]{\entityNameE{Examples}{#2}{#2}{#3}{#1}}% shortcut
\newcommand{\examplesNameM}[3][\default{}]{\entityNameM{Examples}{#2}{#2}{#3}{#1}}% shortcut
\newcommand{\examplesNameP}[3][\default{}]{\entityNameP{Examples}{#2}{#2}{#3}{#1}}% shortcut
\newcommand{\examplesNameR}[3][\default{}]{\entityNameR{Examples}{#2}{#2}{#3}{#1}}% shortcut
\newcommand{\examplesNameS}[3][\default{}]{\entityNameS{Examples}{#2}{#2}{#3}{#1}}% shortcut
\newcommand{\examplesNameX}[3][\default{}]{\entityNameX{Examples}{#2}{#2}{#3}{#1}}% shortcut

% First argument [optional]: alternate name
% Second argument: Subcategory
% Third argument: Alternate subcategory name
% Fourth argument: entity name
\newcommand{\requestsNameD}[4][\default{}]{\entityNameD{Requests}{#2}{#3}{#4}{#1}}% shortcut
\newcommand{\requestsNameE}[4][\default{}]{\entityNameE{Requests}{#2}{#3}{#4}{#1}}% shortcut
\newcommand{\requestsNameM}[4][\default{}]{\entityNameM{Requests}{#2}{#3}{#4}{#1}}% shortcut
\newcommand{\requestsNameP}[4][\default{}]{\entityNameP{Requests}{#2}{#3}{#4}{#1}}% shortcut
\newcommand{\requestsNameR}[4][\default{}]{\entityNameR{Requests}{#2}{#3}{#4}{#1}}% shortcut
\newcommand{\requestsNameS}[4][\default{}]{\entityNameS{Requests}{#2}{#3}{#4}{#1}}% shortcut
\newcommand{\requestsNameX}[4][\default{}]{\entityNameX{Requests}{#2}{#3}{#4}{#1}}% shortcut

% First argument [optional]: alternate name
% Second argument: Subcategory
% Third argument: entity name
\newcommand{\exemplarsNameD}[2][\default{}]{\entityNameD{Exemplars}{\default{}}{\default{}}{#2}{#1}}% shortcut
\newcommand{\exemplarsNameE}[2][\default{}]{\entityNameE{Exemplars}{\default{}}{\default{}}{#2}{#1}}% shortcut
\newcommand{\exemplarsNameM}[2][\default{}]{\entityNameM{Exemplars}{\default{}}{\default{}}{#2}{#1}}% shortcut
\newcommand{\exemplarsNameP}[2][\default{}]{\entityNameP{Exemplars}{\default{}}{\default{}}{#2}{#1}}% shortcut
\newcommand{\exemplarsNameR}[2][\default{}]{\entityNameR{Exemplars}{\default{}}{\default{}}{#2}{#1}}% shortcut
\newcommand{\exemplarsNameS}[2][\default{}]{\entityNameS{Exemplars}{\default{}}{\default{}}{#2}{#1}}% shortcut
\newcommand{\exemplarsNameX}[2][\default{}]{\entityNameX{Exemplars}{\default{}}{\default{}}{#2}{#1}}% shortcut

% First argument [optional]: alternate name
% Second argument: Subcategory
% Third argument: entity name
\newcommand{\classNameD}[3][\default{}]{\entityNameD{Classes}{#2}{#2}{#3}{#1}}% shortcut
\newcommand{\classNameE}[3][\default{}]{\entityNameE{Classes}{#2}{#2}{#3}{#1}}% shortcut
\newcommand{\classNameM}[3][\default{}]{\entityNameM{Classes}{#2}{#2}{#3}{#1}}% shortcut
\newcommand{\classNameP}[3][\default{}]{\entityNameP{Classes}{#2}{#2}{#3}{#1}}% shortcut
\newcommand{\classNameR}[3][\default{}]{\entityNameR{Classes}{#2}{#2}{#3}{#1}}% shortcut
\newcommand{\classNameS}[3][\default{}]{\entityNameS{Classes}{#2}{#2}{#3}{#1}}% shortcut
\newcommand{\classNameX}[3][\default{}]{\entityNameX{Classes}{#2}{#2}{#3}{#1}}% shortcut

\newcommand*{\insertpart}[2]{\clearpage\renewcommand{\mymark}{#1}#2}

% First argument [optional]: alternate hyperlink name
% Second argument: section title
% Third argument: hyperlink section
% Fourth argument: simplified version of title
% Fifth argument: prefix to display before title
\ifthenelse{\boolean{generatingHyperPDF}}%
  {\newcommand*{\sectionStart}[5][\default{}]{\clearpage\section{#5\texorpdfstring{#2}{#4}}%
   \renewcommand{\mymark}{#4}%
   \ifthenelse{\equal{#1}{\default{}}}%
    {\hypertarget{hyper.#3.#4}{}}%
    {\hypertarget{hyper.#3.#1}{}}}}%
  {\newcommand*{\sectionStart}[5][\default{}]{\clearpage\section{#5#2}\renewcommand{\mymark}{#4}}}

\newcommand*{\sectionEnd}[1]{#1} % just a notational convenience

% First argument: hyperlink section
% Second argument: hyperlink name
% Third argument: simplified version of title
\ifthenelse{\boolean{generatingHyperPDF}}%
  {\newcommand{\sectionRef}[3]{\hyperlink{hyper.#1.#2}{\textitcorr{#3}}}}%
  {\newcommand{\sectionRef}[3]{}}

% First argument: [optional] alternate hypertarget name
% Second argument: subsection title
% Third argument: hypertarget section
% Fourth argument: simplified version of title
\ifthenelse{\boolean{generatingHyperPDF}}%
  {\newcommand*{\subsectionStart}[4][\default{}]{\subsection{\texorpdfstring{#2}{#4}}%
   \ifthenelse{\equal{#1}{\default{}}}%
    {\hypertarget{hyper.#3.#4}{}}%
    {\hypertarget{hyper.#3.#1}{}}}}%
  {\newcommand*{\subsectionStart}[4][\default{}]{\subsection{#2}}}
    
\newcommand*{\subsectionEnd}[1]{#1} % just a notational convenience

% First argument: [optional] alternate hypertarget name
% Second argument: subsubsection title
% Third argument: hypertarget section
% Fourth argument: simplified version of title
\ifthenelse{\boolean{generatingHyperPDF}}%
  {\newcommand*{\subsubsectionStart}[4][\default{}]{\subsubsection{\texorpdfstring{#2}{#4}}%
   \ifthenelse{\equal{#1}{\default{}}}%
    {\hypertarget{hyper.#3.#4}{}}%
    {\hypertarget{hyper.#3.#1}{}}}}%
  {\newcommand*{\subsubsectionStart}[4][\default{}]{\subsubsection{#2}\index{#2}}}
    
\newcommand*{\subsubsectionEnd}[1]{#1} % just a notational convenience

% First argument: [optional] alternate section name
% Second argument: section title
\newcommand*{\primaryStart}[2][\default{}]{%
  \sectionStart[#1]{#2}{Primary}{#2}{}}

\newcommand*{\primaryEnd}[1]{#1} % just a notational convenience

% First argument: subsubsection name
% Second argument: link title
\newcommand{\primaryRef}[2]{\sectionRef{Primary}{#1}{#2}}

% First argument: [optional] alternate subsection name
% Second argument: subsection title
\newcommand*{\secondaryStart}[2][\default{}]{%
  \subsectionStart[#1]{#2}{Secondary}{#2}}

\newcommand*{\secondaryEnd}[1]{#1} % just a notational convenience

% First argument: subsubsection name
% Second argument: link title
\newcommand{\secondaryRef}[2]{\sectionRef{Secondary}{#1}{#2}}

% First argument: [optional] alternate subsubsection name
% Second argument: subsubsection title
\newcommand*{\tertiaryStart}[2][\default{}]{%
  \subsubsectionStart[#1]{#2}{Tertiary}{#2}}

\newcommand*{\tertiaryEnd}[1]{#1} % just a notational convenience

% First argument: subsubsection name
% Second argument: link title
\newcommand{\tertiaryRef}[2]{\sectionRef{Tertiary}{#1}{#2}}

% First argument: [optional] alternate appendix name
% Second argument: appendix title
\newcommand*{\appendixStart}[2][\default{}]{%
  \sectionStart[#1]{#2}{Appendix}{#2}{\appendixname{}:~}}

\newcommand*{\appendixEnd}[1]{#1} % just a notational convenience

% First argument: appendix name
% Second argument: link title
\newcommand{\appendixRef}[2]{\sectionRef{Appendix}{#1}{#2}}

\newenvironment{histList}
  {\begin{list}
    {}
    {\setlength{\labelwidth}{108pt}% 1.5"
    \setlength{\leftmargin}{\labelwidth+\labelsep}
    \setlength{\rightmargin}{36pt}% 0.5"
    \setlength{\parsep}{0ex}
    \renewcommand{\makelabel}[1]{\textbf{##1}\hfill}
    }}
  {\end{list}}
\newcommand*{\histListBegin}{\begin{histList}}
\newcommand*{\histListEnd}{\end{histList}}
\newcommand{\histListItem}[1]{\item[#1]}

\newcommand*{\outputBegin}{\begin{quote}\begin{ttfamily}\begin{small}}
\newcommand*{\outputEnd}{\end{small}\end{ttfamily}\end{quote}}
\newcommand*{\codeBegin}{\begin{ttfamily}\begin{small}}
\newcommand*{\codeEnd}{\end{small}\end{ttfamily}}
\newcommand*{\asCode}[1]{\codeBegin{}#1\codeEnd{}}
\newcommand*{\asBoldCode}[1]{\textbf{\asCode{#1}}}
\newcommand*{\asEmphCode}[1]{\textit{\asCode{#1}}}

\newenvironment{tightItems}%
{\begin{itemize}%
\setlength{\leftmargin}{0pt}%
\setlength{\itemsep}{0pt}%
\setlength{\parsep}{0pt}%
\setlength{\parskip}{0pt}}%
{\end{itemize}}
