\ProvidesFile{JSservice.tex}[v1.0.0]
\primaryStart[TheJSService]{The~\JSF{}~Service}
The \asCode{m+mJavaScriptFilterService} application is a Filter service, executing a set
of \JS{} functions contained in a file.
The application responds to the standard Filter service requests and can be used as a
standalone data generator, without the need for a client connection.\\

The \emph{configuration} request has no arguments and returns nothing.\\

The \emph{configure} request request has no arguments and does nothing.\\

The \emph{restartStreams} request stops and then starts the processing thread and the
input and output streams.\\

The \emph{startStreams} request starts a processing thread and the input and output
streams.
Once started, the processing thread will execute the \asCode{scriptThread} function with
an interval provided by the value of \asCode{scriptInterval}.\\

The \emph{stopStreams} request stops the processing thread and the input and output
streams.\\ 

Note that the application will exit if the \emph{\RS} is not running.
\secondaryStart[StartingFromCommandLine]{Starting from the command\longDash{}line}
The application has one required argument \longDash{} the path to the \JS{} file to be
used.
Any additional arguments are passed to the \JS{} environment.
\insertAppParameters
\insertTagDescription{\JSF}
\insertFilterServiceComment
\condPage
\insertStandardServiceCommands
\secondaryEnd
\condPage
\secondaryStart[StartingFromMMManager]{Starting from the \emph{\MMMU} application}
If the service is selected for execution from within the \emph{\MMMU} application, the
following dialog will be presented:
\objScaledDiagram{mpm_images/launchJavaScriptFilterService}%
{launchServiceJavaScript}{Launch options for the \emph{\JSF} service}{0.8}

Note that the script file path is required.
For details on the usage of the `endpoint' and `tag' options, see the \emph{\MMMU} manual.
Once the options are set, the following image will appear in the \emph{\MMMU} window when
the service has successfully started:
\objScaledDiagram{mpm_images/runningJavaScriptFilterService}%
{serviceRunningJavaScript}{The \emph{\MMMU} entity for the \emph{\JSF} service}{1.0}

The script file name forms part of the names of the channels and is added to the title bar
of the service, as though it were a `tag'.
\condPage{}
If a `tag' value is supplied, it is merged with the script file name, as shown here:
\objScaledDiagram{mpm_images/javaScriptWithTag}%
{javaScriptWithTag}{The \emph{\MMMU} entity for the \emph{\JSF} service with a tag}{1.0}

If an `endpoint' value is specified, the script file name forms part of the title bar but
does not affect the channel names:
\objScaledDiagram{mpm_images/javaScriptWithEndpoint}%
{javaScriptWithEndpoint}{The \emph{\MMMU} entity for the \emph{\JSF} service with an
endpoint}{1.0}

If both a `tag' value and an `endpoint' value are provided, the `endpoint' value
determines the channel names while the `tag' value is combined with the script file name
as part of the title bar:
\objScaledDiagram{mpm_images/javaScriptWithEndpointAndTag}%
{javaScriptWithEndpointAndTag}{The \emph{\MMMU} entity for the \emph{\JSF} service with an
endpoint and tag}{1.0}

If the \asBoldCode{+ scriptArgument} button is clicked, additional text entry fields are
provided to allow entering arguments that a particular script may need.
\openSq{}Due to an anomaly with the particular GUI library used to create the
\emph{\MMMU} application, the additional text entry fields cannot be removed from the
dialog, once added.\closeSq{}
\secondaryEnd
\primaryEnd{}
