\ProvidesFile{JSservice.tex}[v1.0.0]
\primaryStart[TheJSService]{The~\JSIO{}~Service}
The \asCode{mpmJavaScriptService} application is an \inputOutput{} service,
executing a set of \JS{} functions contained in a file.
The application responds to the standard \inputOutput{} service requests and can be used
as a standalone data generator, without the need for a client connection.\\

The \emph{configure} request request has no arguments and does nothing.\\

The \emph{restartStreams} request stops and then starts processing thread and the input
and output streams.\\

The \emph{startStreams} request starts a processing thread and the input and output
streams.
Once started, the processing thread will execute the \asCode{scriptThread} function with
an interval provided by the value of \asCode{scriptInterval}.\\

The \emph{stopStreams} request stops the processing thread and the input and output
streams.\\ 

Note that the application will exit if the \emph{Registry Service} is not running.\\

The application has one argument -- the path to the \JS{} file to be used.
Any additional arguments are passed to the \JS{} environment.
The application has four optional parameters:
\begin{itemize}
\item \textbf{-e:} specifies an alternative endpoint name to be used
\item \textbf{-p:} specifies the port number to be used, if a non--default port is desired
\item \textbf{-r:} report the service metrics when the application exits
\item \textbf{-t:} specifies the tag to be used as part of the service name
\end{itemize}
The tag is added to the standard name of the service, so that more than one copy of the
service can execute -- an \mplusm{} installation can support multiple copies of each
\inputOutput{} service, but the \emph{Channel Manager} application cannot display them
without a distinguishing `tag'.
If the tag is not specified, the standard name of the service will be used.
As well as the service name, the output stream name is modified if a tag is specified and
the default endpoint is being used.\\

If the application is running from a terminal, the following commands are available:
\begin{itemize}
\item \textbf{?:} display a list of the available commands and any help text provided in
the \asCode{scriptHelp} value.
\item \textbf{b:} start the processing thread and the input and output streams. 
\item \textbf{c:} configure the service; this has no effect, as the service has no
configurable parameters. 
\item \textbf{e:} stop the processing thread and input and output streams. 
\item \textbf{q:} quit the application. 
\item \textbf{r:} restart the processing thread and input and output streams. 
\item \textbf{u:} reset the configuration so that it will be reprocessed when the
processing thread and input and output streams are restarted. 
\end{itemize}
\primaryEnd{}
