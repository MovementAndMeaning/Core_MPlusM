\ProvidesFile{JSservice.tex}[v1.0.0]
\primaryStart[TheJSService]{The~\JSIO{}~Service}
The \asCode{mpmJavaScriptService} application is an \inputOutput{} service,
executing a set of \JS{} functions contained in a file.
The application responds to the standard \inputOutput{} service requests and can be used
as a standalone data generator, without the need for a client connection.\\

The \emph{configure} request request has no arguments and does nothing.\\

The \emph{restartStreams} request stops and then starts processing thread and the input
and output streams.\\

The \emph{startStreams} request starts a processing thread and the input and output
streams.
Once started, the processing thread will execute the \asCode{scriptThread} function with
an interval provided by the value of \asCode{scriptInterval}.\\

The \emph{stopStreams} request stops the processing thread and the input and output
streams.\\ 

Note that the application will exit if the \emph{Registry Service} is not running.
\secondaryStart[StartingFromCommandLine]{Starting from the command\longDash{}line}
The application has one argument \longDash{} the path to the \JS{} file to be used.
Any additional arguments are passed to the \JS{} environment.
\insertAppParameters
\insertTagDescription{\JSIO}
\insertFilterServiceComment
\condPage{}
If the application is running from a terminal and has not been automatically started via
the `\asCode{go}' option, the following commands are available:
\begin{itemize}
\item\cmdItem{?}{display a list of the available commands and any help text provided in
the \asCode{scriptHelp} value}
\item\exSp\cmdItem{b}{start the processing thread and the input and output streams}
\item\exSp\cmdItem{c}{configure the service; this has no effect, as the service has no
configurable parameters}
\item\exSp\cmdItem{e}{stop the processing thread and input and output streams}
\item\exSp\cmdItem{q}{quit the application}
\item\exSp\cmdItem{r}{restart the processing thread and input and output streams}
\item\exSp\cmdItem{u}{reset the configuration so that it will be reprocessed when the
processing thread and input and output streams are restarted}
\end{itemize}
\secondaryEnd
\condPage
\secondaryStart[StartingFromChannelManager]{Starting from \emph{Channel Manager}}
If the service is selected for execution from within the \emph{Channel Manager}
application, the following dialog will be presented:
\objScaledDiagram{mpm_images/launchJavaScriptFilterService}%
{launchServiceJavaScript}{Launch options for the \JSIO{} service}{1.0}

Note that the script file path is required.
For details on the usage of the `endpoint' and `tag' options, see the \emph{Channel
Manager Utility} manual.
Once the options are set, the following image will appear in the \emph{Channel Manager}
window when the service has successfully started:
\objScaledDiagram{mpm_images/runningJavaScriptFilterService}%
{serviceRunningJavaScript}{The \emph{Channel Manager} entity for the \JSIO{} service}{1.0}

The script file name forms part of the names of the channels and is added to the title bar
of the service, as though it were a `tag'.
\condPage{}
If a `tag' value is supplied, it is merged with the script file name, as shown here:
\objScaledDiagram{mpm_images/javaScriptWithTag}%
{javaScriptWithTag}{The \emph{Channel Manager} entity for the \JSIO{} service with a tag}{1.0}

If an `endpoint' value is specified, the script file name forms part of the title bar but
does not affect the channel names:
\objScaledDiagram{mpm_images/javaScriptWithEndpoint}%
{javaScriptWithEndpoint}{The \emph{Channel Manager} entity for the \JSIO{} service with an
endpoint}{1.0}

If both a `tag' value and an `endpoint' value are provided, the `endpoint' value
determines the channel names while the `tag' value is combined with the script file name
as part of the title bar:
\objScaledDiagram{mpm_images/javaScriptWithEndpointAndTag}%
{javaScriptWithEndpointAndTag}{The \emph{Channel Manager} entity for the \JSIO{} service
with an endpoint and tag}{1.0}
\secondaryEnd
\primaryEnd{}
