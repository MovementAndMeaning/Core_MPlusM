\ProvidesFile{MMexemplars.tex}[v1.0.0]
\appendixStart[UsingExemplars]{\textitcorr{Using~the~Exemplar~Files}}
The exemplars that are part of the \mplusm{} set of source files are
provided as templates for custom application development.
They are based on the example applications, modified to be buildable outside the \mplusm{}
environment, provided that the \mplusm{} libraries and headers have been installed.\\

The exemplar files use a common namespace, \asCode{Exemplars}, to allow for custom
applications that re\longDash{}use class names; it is expected that the new applications
will use a unique namespace.
\secondaryStart{Exemplar~directories}
\tertiaryStart{\exemplarsNameD{ExemplarAdapter}}
The files in \exemplarsNameX{ExemplarAdapter} are based on the source files for the
\examplesNameR{Adapters}{mpmRandomNumberAdapter} example application.
The files \asCode{M+MRandomNumberXXX.YYY} correspond to the files
\asCode{M+MExemplarXXX.YYY}, where \asCode{XXX} is \asCode{AdapterData},\\
\asCode{AdapterMain} or \asCode{BaseInputHandler}.
\tertiaryEnd{\exemplarsNameE{ExemplarAdapter}}
\tertiaryStart{\exemplarsNameD{ExemplarClient}}
The files in \exemplarsNameX{ExemplarClient} are based on the source files for the
\examplesNameR{Clients}{mpmRandomNumberClient} example application.
The files \asCode{M+MRandomNumberXXX.YYY} correspond to the files
\asCode{M+MExemplarXXX.YYY}, where \asCode{XXX} is \asCode{Client} or \asCode{ClientMain}.
\tertiaryEnd{\exemplarsNameE{ExemplarClient}}
\tertiaryStart{\exemplarsNameD{ExemplarCommon}}
The file in \exemplarsNameX{ExemplarCommon} are based on a source file from the
\examplesNameR{Services}{mpmRandomNumberService} example application.
The file \asCode{M+MRandomNumberRequests.h} corresponds to the file
\asCode{M+MExemplarRequests.h}.
\tertiaryEnd{\exemplarsNameE{ExemplarCommon}}
\tertiaryStart{\exemplarsNameD{ExemplarFilterService}}
The files in \exemplarsNameX{ExemplarFilterService} are based on the source files for the
\examplesNameR{Services}{mpmTruncateFloatService} example application.
The files \asCode{M+MTruncateFilterXXX.YYY} correspond to the files
\asCode{M+MExemplarFilterXXX.YYY}, where \asCode{XXX} is\\
\asCode{InputHandler}, \asCode{Requests}, \asCode{Service} or \asCode{ServiceMain}.
\tertiaryEnd{\exemplarsNameE{ExemplarFilterService}}
\tertiaryStart{\exemplarsNameD{ExemplarInputService}}
The files in \exemplarsNameX{ExemplarInputService} are based on the source files for the
\examplesNameR{Services}{mpmRandomBurstService} example application.
The files \asCode{M+MRandomBurstXXX.YYY} correspond to the files
\asCode{M+MExemplarInputXXX.YYY}, where \asCode{XXX} is \asCode{Requests},
\asCode{Service}, \asCode{ServiceMain} or \asCode{Thread}.
\tertiaryEnd{\exemplarsNameE{ExemplarInputService}}
\tertiaryStart{\exemplarsNameD{ExemplarOutputService}}
The files in \exemplarsNameX{ExemplarOutputService} are based on the source files for the
\examplesNameR{Services}{mpmRecordIntegersService} example application.
The files \asCode{M+MRecordIntegersXXX.YYY} correspond to the files
\asCode{M+MExemplarOutputXXX.YYY}, where \asCode{XXX} is \asCode{InputHandler},
\asCode{Requests}, \asCode{Service} or \asCode{ServiceMain}.
\tertiaryEnd{\exemplarsNameE{ExemplarOutputService}}
\tertiaryStart{\exemplarsNameD{ExemplarService}}
The files in \exemplarsNameX{ExemplarService} are based on the source files for the
\examplesNameR{Services}{mpmRandomNumberService} example application.
The files \asCode{M+MRandomNumberXXX.YYY} correspond to the files
\asCode{M+MExemplarXXX.YYY}, where \asCode{XXX} is \asCode{Service} or
\asCode{ServiceMain} and the files \asCode{M+MRandomRequestHandler.YYY} correspond to the
files\\
\asCode{M+MSimpleRequestHandler.YYY}.
\tertiaryEnd{\exemplarsNameE{ExemplarService}}
\secondaryEnd
\secondaryStart{Exemplar~classes}
\tertiaryStart{\classNameD{Exemplar}{ExemplarAdapterData}}
The \classNameX{Exemplar}{ExemplarAdapterData} class is a subclass of
\classNameR{Common}{BaseAdapterData} and is used to provide a linkage between instances of
the \classNameR{Exemplar}{ExemplarInputHandler} class and instances of the
\classNameR{Exemplar}{ExemplarService} class; it does not have any methods or variables
in addition to those provided by its superclass and primarily serves as a placeholder for
customization.
\tertiaryEnd{\classNameE{Exemplar}{ExemplarAdapterData}}
\tertiaryStart{\classNameD{Exemplar}{ExemplarClient}}
The \classNameX{Exemplar}{ExemplarClient} class is a subclass of
\classNameR{Common}{BaseClient} and has two `placeholder' methods,
\asCode{getOneRandomNumber} and\\
\asCode{getRandomNumbers}, which should be replaced when developing custom applications.
\tertiaryEnd{\classNameE{Exemplar}{ExemplarClient}}
\tertiaryStart{\classNameD{Exemplar}{ExemplarFilterInputHandler}}
The \classNameX{Exemplar}{ExemplarFilterInputHandler} class is a subclass of
\classNameR{Common}{BaseInputHandler} and implements the \asCode{handleInput} method,
processing floating\longDash{}point and integer data; this should be changed when
developing a custom application.
\tertiaryEnd{\classNameE{Exemplar}{ExemplarFilterInputHandler}}
\tertiaryStart{\classNameD{Exemplar}{ExemplarFilterService}}
The \classNameX{Exemplar}{ExemplarFilterService} class is a subclass of
\classNameR{Common}{BaseFilterService} and implements the minimal functionality necessary
to support its behaviour.
In particular, the stream descriptions provided in the method\\
\asCode{setUpStreamDescriptions} should be replaced in a custom application.
\tertiaryEnd{\classNameE{Exemplar}{ExemplarFilterService}}
\tertiaryStart{\classNameD{Exemplar}{ExemplarInputHandler}}
The \classNameX{Exemplar}{ExemplarInputHandler} class is a subclass of
\classNameR{Common}{BaseInputHandler} and invokes the `placeholder' methods of the\\
\classNameR{Exemplar}{ExemplarClient} within the \asCode{handleInput} method, which should
be replaced when developing custom applications.
\tertiaryEnd{\classNameE{Exemplar}{ExemplarInputHandler}}
\tertiaryStart{\classNameD{Exemplar}{ExemplarInputService}}
The \classNameX{Exemplar}{ExemplarInputService} is a subclass of
\classNameR{Common}{BaseInputService} and implements the minimal functionality necessary
to support its behaviour.
In particular, the stream description provided in the method
\asCode{setUpStreamDescriptions} should be replaced in a custom application.
\tertiaryEnd{\classNameE{Exemplar}{ExemplarInputService}}
\tertiaryStart{\classNameD{Exemplar}{ExemplarInputThread}}
The \classNameX{Exemplar}{ExemplarInputThread} class is a subclass of
\asCode{yarp::os::Thread} and implements the \asCode{run} method to generate a stream of
random numbers \longDash{} this behaviour should be changed when developing a custom
application.
Note that the method \asCode{clearOutputChannel} provides a mechanism for the
\classNameR{Exemplar}{ExemplarInputService} instance to signal that the thread is to
stop \longDash{} this may not be sufficient in a custom application.
\tertiaryEnd{\classNameE{Exemplar}{ExemplarInputThread}}
\tertiaryStart{\classNameD{Exemplar}{ExemplarOutputInputHandler}}
The \classNameX{Exemplar}{ExemplarOutputInputHandler} class is a subclass of
\classNameR{Common}{BaseInputHandler} and simply writes out the data received by the
\asCode{handleInput} method; this should be replaced when developing a custom application.
\tertiaryEnd{\classNameE{Exemplar}{ExemplarOutputInputHandler}}
\tertiaryStart{\classNameD{Exemplar}{ExemplarOutputService}}
The \classNameX{Exemplar}{ExemplarOutputService} class is a subclass of
\classNameR{Common}{BaseOutputService} and implements the minimal functionality necessary
to support its behaviour.
In particular, the stream description provided in the method
\asCode{setUpStreamDescriptions} should be replaced in a custom application.
\tertiaryEnd{\classNameE{Exemplar}{ExemplarOutputService}}
\tertiaryStart{\classNameD{Exemplar}{ExemplarService}}
The \classNameX{Exemplar}{ExemplarService} class is a subclass of
\classNameR{Common}{BaseService} and depends on the
\classNameR{Exemplar}{SimpleRequestHandler} class to provide the service functionality.
When developing a custom application, the methods \asCode{attachRequestHandlers} and
\asCode{detachRequestHandlers} need to be extended to support the requests defined by the
service.
\tertiaryEnd{\classNameE{Exemplar}{ExemplarService}}
\tertiaryStart{\classNameD{Exemplar}{SimpleRequestHandler}}
The \classNameX{Exemplar}{SimpleRequestHandler} class is a subclass of
\classNameR{Common}{BaseRequestHandler} and implements the `get a random number' and
`get several random numbers' requests sent by the \classNameR{Exemplar}{ExemplarClient}
client application; a custom application should replace this class with one or more
request handlers corresponding to the requests defined by the service, with the
corresponding changes made to the \classNameR{Exemplar}{ExemplarService} class.
\tertiaryEnd{\classNameE{Exemplar}{SimpleRequestHandler}}
\secondaryEnd
\secondaryStart{Creating~a~custom~application}
\tertiaryStart{Determine the kind of application}
There are four kinds of applications that can be build from the exemplars:
\begin{itemize}
\item\textbf{Client~/~Service~/~Adapter:} This will use the
\exemplarsNameR{ExemplarClient}, \exemplarsNameR{ExemplarCommon} and
\exemplarsNameR{ExemplarService} directories, with the \exemplarsNameR{ExemplarAdapter}
directory, if an adapter is required
\item\exSp\textbf{Filter~Service} This will use the \exemplarsNameR{ExemplarFilterService}
directory
\item\exSp\textbf{Input~Service} This will use the \exemplarsNameR{ExemplarInputService}
directory
\item\exSp\textbf{Output~Service} This will use the \exemplarsNameR{ExemplarOutputService}
directory 
\end{itemize}
A custom application set can include one or more of these applications.
For the purposes of this section, we will include all four kinds of application.
\tertiaryEnd
\tertiaryStart{Determine the namespace}
Note that the classes in the source files are all within the \asCode{M+M:Exemplar}
namespace \longDash{} this will need to be changed to avoid conflicts with other uses of
the exemplar files.
Choose a name that describes the purpose of the custom application, such as
\asCode{Tracker} for an input service to communicate with a tracking device.
For the purposes of this section, we will use the name \asCode{Foo}.
\tertiaryEnd
\tertiaryStart{Copy the source files}
Copy the full directories from \asCode{Exemplars}; note that it isn't strictly necessary
to copy a subdirectory if it is not going to be built, but the files at the top level of
the directory are still needed.
\tertiaryEnd
\tertiaryStart{Rename the source files}
The description of the relationships between the example directories and the exemplar
directories indicate how this renaming should be done.
For example, the file \asCode{M+MExemplarFilterInputHandler.cpp} would be renamed to
\asCode{M+MFooFilterInputHandler.cpp}.
Doing the renaming at this point will help avoid duplicate files or incorrect file
inclusion.
\tertiaryEnd
\tertiaryStart{Edit the source files}
This is by far the biggest task, since it will involve editing every file in each
directory.
\begin{itemize}
\item In the \asCode{CMakeLists.txt} files, change the file references from
\asCode{M+MExemplarXXX} to \asCode{M+MFooXXX}
\item\exSp{} In the \compLang{C++} header and source files, change the class names from
\asCode{ExemplarXXX} to \asCode{FooXXX}, change the file references from
\asCode{M+MExemplarXXX.h} to \asCode{M+MFooXXX.h}, change the namespace from
\asCode{Exemplar} to \asCode{Foo} and replace references to `\asCode{exemplar}' in
comments with `\asCode{foo}' \longDash{} note that the copyright notice at the beginning
of each file includes the word `\asCode{EXEMPLARY}', which should not be changed
\item\exSp{} In the main files for each application (\asCode{M+MFooAdapterMain.cpp},
\asCode{M+MFooClientMain.cpp}, et cetera),
change any text strings that contain `\asCode{exemplar}' to contain `\asCode{foo}' or that
contain `\asCode{Exemplar}' to contain `\asCode{Foo}' \longDash{} these will most often be
the names of \yarp{} network connections and must not conflict with the ports from other
services and clients
\item\exSp{} In the files \asCode{M+MFooRequests.h} and \asCode{M+MFooService.cpp},
replace
\asCode{MpM\textunderscore{}EXEMPLAR\textunderscore{}CANONICAL\textunderscore{}NAME}
with\\
\asCode{MpM\textunderscore{}FOO\textunderscore{}CANONICAL\textunderscore{}NAME}
\item\exSp{} In the files \asCode{M+MFooXXXService.h} and
\asCode{M+MFooXXXServiceMain.cpp}, replace\\
\asCode{DEFAULT\textunderscore{}EXEMPLARxxx\textunderscore{}SERVICE\textunderscore{}%
NAME} with
\asCode{DEFAULT\textunderscore{}FOOxxx\textunderscore{}SERVICE\textunderscore{}NAME},
where \asCode{xxx} is \asCode{FILTER}, \asCode{INPUT} or \asCode{OUTPUT}
\item\exSp{} In the files \asCode{M+MFooService.h} and \asCode{M+MFooServiceMain.cpp},
replace
\asCode{DEFAULT\textunderscore{}EXEMPLAR\textunderscore{}SERVICE\textunderscore{}NAME}
with \asCode{DEFAULT\textunderscore{}FOO\textunderscore{}SERVICE\textunderscore{}NAME}
\item\exSp{} In the files \asCode{M+MFooXXXRequests.h} and \asCode{M+MFooXXXService.cpp},
replace\\
\asCode{MpM\textunderscore{}EXEMPLARXXX\textunderscore{}CANONICAL\textunderscore{}NAME}
with \asCode{MpM\textunderscore{}FOOXXX\textunderscore{}CANONICAL\textunderscore{}NAME},
where \asCode{XXX} is \asCode{FILTER}, \asCode{INPUT} or\\
\asCode{OUTPUT}
\item\exSp{} In the file \asCode{M+MFooXXXServiceMain.cpp}, replace
\asCode{EXEMPLARXXX\textunderscore{}OPTIONS} with \asCode{FOOXXX\textunderscore{}OPTIONS},
where \asCode{XXX} is \asCode{FILTER}, \asCode{INPUT} or \asCode{OUTPUT}
\item\exSp{} In the top\longDash{}level file \asCode{CMakeLists.txt}, comment out the
directories that are not being built, by putting a hash\longDash{}mark (or pound sign),
`\asCode{\#}', at the beginning of the \asCode{add\textunderscore{}subdirectory} statement
for the directory
\item\exSp{} And lastly, change the ``\asCode{Written by}'' string in the comment block at
the beginning of each file to reference the actual writer of the new custom application
set
\end{itemize}
\tertiaryEnd
\tertiaryStart{Build the application(s)}
At this point, it should be possible to build the custom applications.
The procedures are different for Macintosh~OS~X, Windows and Linux, although they start
from the same source files. 
\begin{center}\textbf{On~Macintosh~OS~X:}\end{center}
\begin{itemize}
\item Execute `\asCode{ccmake~.}' from a terminal session that has the current directory
set to the top\longDash{}level directory for the custom files and then perform the
`\textbf{c}' (configure) and `\textbf{g}' (generate) commands
\item\exSp{} If \asCode{ccmake} does not exit with success, use the `\textbf{q}' (quit)
command and review the \asCode{CMakeLists.txt} files to make sure that they've been edited
correctly \longDash{} the error screen from \asCode{ccmake} will indicate which files have
problems
\item\exSp{} Execute `\asCode{cmake~.}' from the terminal session, to create the
`makefiles' used to construct the applications
\item\exSp{} Execute `\asCode{make~clean~\&\&~make}' to build the applications; errors at
this point will be a result of incomplete or incorrect edits and should be easy to fix
\item\exSp{} After testing the applications, they can be installed on the system, if
desired, via the command\\
`\asCode{sudo~make~install}'
\end{itemize}
\begin{center}\textbf{On~Windows:}\end{center}
\begin{itemize}
\item Execute the tool \asCode{cmake} to create the project files used to construct the
application
\item\exSp\TBD
\end{itemize}
\begin{center}\textbf{On~Linux:}\end{center}
\begin{itemize}
\item Execute the tool \asCode{cmake} to create the project files used to construct the
application
\item\exSp\TBD
\end{itemize}
\tertiaryEnd
\tertiaryStart{Add custom code}
Once the applications have successfully built, it's time to add the custom functionality
that is desired.
\begin{itemize}
\item For \inputOutput{} services, any custom configuration data should be gathered from
the command line or requested interactively (in the main file for the service
application) and processed in the \asCode{configure} method of the class for the service;
for an Input or Output service, a typical configuration value might be a device
identifier or a file\longDash{}system path
\item\exSp{} For Filter services, the processing of its input is done in the
\asCode{handleInput} method in the\\
\asCode{M+MFooFilterInputHandler.cpp} file
\item\exSp{} For Input services, the \asCode{run} method in the
\asCode{M+MFooInputThread.cpp} file is asynchronous \longDash{} relative to \mplusm{}
\longDash{} to gather input and route it to the \yarp{} network connections for the
service
\item\exSp{} For Output services, the processing of its input is done in the
\asCode{handlInput} method in the\\
\asCode{M+MFooOutputInputHandler.cpp} file
\item\exSp{} For regular services and clients, custom requests must be listed in the
\asCode{M+MFooRequests.h} file, replacing the `simple' request that is present.
These requests must be implemented as new classes in the `\asCode{FooService}' directory
and pointers to instances of the new request handler classes must be added to the
\asCode{FooService} class, using the `simple' request handler as a template.
Corresponding methods will need to be added to the \asCode{FooClient} class, in order to
issue the new requests.
\item\exSp{} For adapters, any new methods added to the \asCode{FooClient} class to
support the new requests for the \asCode{FooService} class will require similar changes to
the \asCode{FooAdapter} class.
\end{itemize}
\tertiaryEnd
\tertiaryStart{Install the applications}
\begin{center}\textbf{On~Macintosh~OS~X:}\end{center}
Once all the customization has been developed and tested, the applications can then be
installed by executing\\
`\asCode{sudo~make~install}' from a terminal session in the top\longDash{}level directory.
\begin{center}\textbf{On~Windows:}\end{center}
\TBD
\begin{center}\textbf{On~Linux:}\end{center}
\TBD
\tertiaryEnd
\secondaryEnd
\appendixEnd{}
