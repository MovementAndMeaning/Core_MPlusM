\ProvidesFile{VDSservice.tex}[v1.0.0]
\primaryStart[TheVDSIService]{The~\VDSI{}~Service}
The \asCode{m+mViconDataStreamInputService} application is an Input service,
generating a stream of information on the position and orientation of one or more bodies.
The application responds to the standard Input service requests and can be used as a
standalone data generator, without the need for a client connection.\\

The \emph{configure} request has two arguments \longDash{} a string value for the host
name of the Vicon DataStream server and an integer value for the port on the Vicon
DataStream server to connect to.
These values will be applied when the output stream is started or restarted.\\ 

The \emph{restartStreams} request stops and then starts the output stream.\\

The \emph{startStreams} request request initiates listening to the Vicon DataStream
server.
Once started, the service will send groups of subject data via the output \yarp{} network
connection.\\

The \emph{stopStreams} request stops the Vicon DataStream server listener, which stops the
output \yarp{} network connection.\\ 

Note that the application will exit if the \emph{\RS} is not running.
\secondaryStart[StartingFromCommandLine]{Starting from the command\longDash{}line}
The application has two optional arguments \longDash{} host name for the Vicon device
server and the port for the Vicon device server.
\insertAppParameters
\insertTagDescription{\VDSI}
\insertInputServiceComment\\

If the application is running from a terminal and has not been automatically started via
the `\asCode{go}' option, the following commands are available:
\begin{itemize}
\item\cmdItem{?}{display a list of the available commands}
\item\exSp\cmdItem{b}{start the output stream, sending subject data}
\item\exSp\cmdItem{c}{configure the service by providing the host name and port for the
Vicon DataStream server}
\item\exSp\cmdItem{e}{stop the output stream}
\item\exSp\cmdItem{q}{quit the application}
\item\exSp\cmdItem{r}{restart the output stream}
\item\exSp\cmdItem{u}{reset the configuration so that it will be reprocessed when the
output stream is restarted}
\end{itemize}
\secondaryEnd
\condPage
\secondaryStart[StartingFromChannelManager]{Starting from \emph{\MMMU}}
If the service is selected for execution from within the \emph{\MMMU} application, the
following dialog will be presented:
\objScaledDiagram{mpm_images/launchViconDataStreamInputService}%
{launchServiceViconDataStream}{Launch options for the \emph{\VDSI} service}{0.8}

For details on the usage of the `endpoint' and `tag' options, see the \emph{\MMMU} manual.
Once the options are set, the following image will appear in the \emph{\MMMU} window when
the service has successfully started:
\objScaledDiagram{mpm_images/runningViconDataStreamInputService}%
{serviceRunningViconDataStream}{The \emph{\MMMU} entity for the \emph{\VDSI} service}{1.0}
\secondaryEnd
\primaryEnd{}
\primaryStart[TheVBIService]{The~\VBI{}~Service}
The \asCode{m+mViconBlobInputService} application is an Input service,
generating a stream of information on the position and orientation of one or more bodies.
The application responds to the standard Input service requests and can be used as a
standalone data generator, without the need for a client connection.\\

The \emph{configure} request request has a single argument \longDash{} a
floating\longDash{}point value for the translation scale to apply to the segment
coordinates.
The value will be applied when the output stream is started or restarted.\\ 

The \emph{restartStreams} request stops and then starts the output stream.\\

The \emph{startStreams} request request initiates listening to the Vicon DataStream
server.
Once started, the service will send groups of subject data via the output \yarp{} network
connection.\\

The \emph{stopStreams} request stops the Vicon DataStream server listener, which stops the
output \yarp{} network connection.\\ 

Note that the application will exit if the \emph{\RS} is not running.
\secondaryStart[StartingFromCommandLine]{Starting from the command\longDash{}line}
The application has one optional argument \longDash{} the translation scale to be applied
to the finger coordinates.
\insertAppParameters
\insertTagDescription{\VBI}
\insertInputServiceComment
\condPage{}
If the application is running from a terminal and has not been automatically started via
the `\asCode{go}' option, the following commands are available:
\begin{itemize}
\item\cmdItem{?}{display a list of the available commands}
\item\exSp\cmdItem{b}{start the output stream, sending hand data}
\item\exSp\cmdItem{c}{configure the service by providing the translation scale to be used}
\item\exSp\cmdItem{e}{stop the output stream}
\item\exSp\cmdItem{q}{quit the application}
\item\exSp\cmdItem{r}{restart the output stream}
\item\exSp\cmdItem{u}{reset the configuration so that it will be reprocessed when the
output stream is restarted}
\end{itemize}
\secondaryEnd
\condPage
\secondaryStart[StartingFromChannelManager]{Starting from \emph{\MMMU}}
If the service is selected for execution from within the \emph{\MMMU} application, the
following dialog will be presented:
\objScaledDiagram{mpm_images/launchViconBlobInputService}%
{launchServiceLeapVicon}{Launch options for the \emph{\VBI} service}{0.8}

For details on the usage of the `endpoint' and `tag' options, see the \emph{\MMMU} manual.
Once the options are set, the following image will appear in the \emph{\MMMU} window when
the service has successfully started:
\objScaledDiagram{mpm_images/runningViconBlobInputService}%
{serviceRunningViconBlob}{The \emph{\MMMU} entity for the \emph{\VBI} service}{1.0}
\secondaryEnd
\primaryEnd{}
