\ProvidesFile{MMexamples.tex}[v1.0.0]
\appendixStart{\textitcorr{Examples}}
The examples that are part of the \mplusm{} set of source files are provided to support
the development of custom applications.
There are example clients, services, adapters as well as examples of the more specialized
Input, Output and Filter services.\\

All the example applications respond to a `HUP' signal and will exit gracefully when one
is sent to them.
\secondaryStart{Example~Services}
The set of example services include a pair of simple services
[\examplesNameR{Services}{mpmEchoService} and
\examplesNameR{Services}{mpmRandomNumberService}], a service with client context
[\examplesNameR{Services}{mpmRunningSumService}] as well as an input service
[\examplesNameR{Services}{mpmRandomBurstService}], three output services
[\examplesNameR{Services}{mpmAbsorberService},
\examplesNameR{Services}{mpmRecordAsJSONService} and
\examplesNameR{Services}{mpmRecordIntegersService}] and a filter service
[\examplesNameR{Services}{mpmTruncateFloatService}].\\

Note that services normally have no direct interface.
That is, they are `faceless' applications and perform their operations without user
interaction.
\tertiaryStart{\examplesNameP{Services}{mpmAbsorberService}}
The \examplesNameX{Services}{mpmAbsorberService} application is an Output
service, accepting (and ignoring) any input data.
The application responds to the standard Output service requests and can be used as a
standalone data sink, without the need for a client connection.\\

The \requestsNameR{\inputOutput}{InputOutput}{configure} request has no arguments and
does nothing.\\

The \requestsNameR{\inputOutput}{InputOutput}{restartStreams} request stops and then
starts the input stream.\\

The \requestsNameR{\inputOutput}{InputOutput}{startStreams} request initiates listening
on the input stream.\\

The \requestsNameR{\inputOutput}{InputOutput}{stopStreams} request terminates listening
on the input stream.\\

Note that the application will exit if the
\serviceNameR[Registry~Service]{RegistryService} is not running.\\

\insertAppParameters
\insertTagDescription{Absorber Output}
\insertOutputServiceComment\\

If the application is running from a terminal and has not been automatically started via
the `\asCode{go}' option, the following commands are available:
\begin{itemize}
\item\cmdItem{?}{display a list of the available commands}
\item\exSp\cmdItem{b}{start the input stream}
\item\exSp\cmdItem{c}{configure the service; this has no effect, as the service has no
configurable parameters}
\item\exSp\cmdItem{e}{stop the input stream}
\item\exSp\cmdItem{q}{quit the application}
\item\exSp\cmdItem{r}{restart the input stream}
\item\exSp\cmdItem{u}{reset the configuration so that it will be reprocessed when the
input stream is restarted}
\end{itemize}
\condPage
If the service is selected for execution from within \emph{Channel Manager}, the following
dialog will be presented:
\objScaledDiagram{mpm_images/launchAbsorberOutputService}%
{launchServiceAbsorber}{Launch options for the Absorber output service}{1.0}

Once the options are set, the following image will appear in the \emph{Channel Manager}
window when the service has successfully started:
\objScaledDiagram{mpm_images/runningAbsorberOutputService}%
{serviceRunningAbsorber}{The \emph{Channel Manager} entity for the Absorber output
service}{1.0}
\tertiaryEnd{\examplesNameE{Services}{mpmAbsorberService}}
\condPage
\tertiaryStart{\examplesNameP{Services}{mpmEchoService}}
The \examplesNameX{Services}{mpmEchoService} application responds to
\requestsNameR{Miscellaneous}{Miscellaneous}{echo} requests from the
\examplesNameR{Clients}{mpmEchoClient} application; it simply returns any arguments
sent with the request.\\

Note that the application will exit if the
\serviceNameR[Registry~Service]{RegistryService} is not running.\\

\insertAutoAppParameters
\insertTagDescription{Echo}
The matching client application does not directly use the endpoint name to establish a
connection to the service.
\condPage
If the service is selected for execution from within \emph{Channel Manager}, the following
dialog will be presented:
\objScaledDiagram{mpm_images/launchEchoService}%
{launchServiceEcho}{Launch options for the Echo service}{1.0}

Once the options are set, the following image will appear in the \emph{Channel Manager}
window when the service has successfully started:
\objScaledDiagram{mpm_images/runningEchoService}%
{serviceRunningEcho}{The \emph{Channel Manager} entity for the Echo service}{1.0}
\tertiaryEnd{\examplesNameE{Services}{mpmEchoService}}
\tertiaryStart{\examplesNameP{Services}{mpmRandomBurstService}}
The \examplesNameX{Services}{mpmRandomBurstService} application is an Input service,
generating a stream of random floating\longDash{}point values, with a given `burst size'
\longDash{} the number of values sent in a group, and a `burst period' \longDash{} the
number of seconds between groups.
The application responds to the standard Input service requests and can be used as a
standalone data generator, without the need for a client connection.\\

The \requestsNameR{\inputOutput}{InputOutput}{configure} request has two arguments
\longDash{} a floating\longDash{}point value for the `burst period' and an integer value
for the `burst size'.
These values will be applied when the random number generator is started or restarted.\\

The \requestsNameR{\inputOutput}{InputOutput}{restartStreams} request stops and then
starts the random number generator, which restarts the output \yarp{} network
connection.\\

The \requestsNameR{\inputOutput}{InputOutput}{startStreams} request starts a random
number generator for the output stream, using the configured `burst period' and
`burst size'.
Once started, the random number generator will send groups of random
floating\longDash{}point values via the output \yarp{} network connection.\\

The \requestsNameR{\inputOutput}{InputOutput}{stopStreams} request stops the random
number generator, which stops the output \yarp{} network connection.\\ 

Note that the application will exit if the
\serviceNameR[Registry~Service]{RegistryService} is not running.\\

The application has two optional arguments \longDash{} the burst period, in seconds and
the number of random values to generate in each burst.
\insertAppParameters
\insertTagDescription{Random Burst Input}
\insertInputServiceComment\\

The other parameters provide the initial burst period and size; if not specified, a burst
period of 1 second and burst size of 1 will be used.
Note that the burst period and size can also be set via commands, if the application is
running from a terminal.\\

If the application is running from a terminal and has not been automatically started via
the `\asCode{go}' option, the following commands are available:
\begin{itemize}
\item\cmdItem{?}{display a list of the available commands}
\item\exSp\cmdItem{b}{start the output stream, sending bursts of data of the specified
size and period}
\item\exSp\cmdItem{c}{configure the service by providing the burst size and period}
\item\exSp\cmdItem{e}{stop the output stream}
\item\exSp\cmdItem{q}{quit the application}
\item\exSp\cmdItem{r}{restart the output stream}
\item\exSp\cmdItem{u}{reset the configuration so that it will be reprocessed when the
output stream is restarted}
\end{itemize}
\condPage
If the service is selected for execution from within \emph{Channel Manager}, the following
dialog will be presented:
\objScaledDiagram{mpm_images/launchRandomBurstInputService}%
{launchServiceRandomBurst}{Launch options for the Random Burst input service}{1.0}

Once the options are set, the following image will appear in the \emph{Channel Manager}
window when the service has successfully started:
\objScaledDiagram{mpm_images/runningRandomBurstInputService}%
{serviceRunningRandomBurst}{The \emph{Channel Manager} entity for the Random Burst input
service}{1.0}
\tertiaryEnd{\examplesNameE{Services}{mpmRandomBurstService}}
\condPage
\tertiaryStart{\examplesNameP{Services}{mpmRandomNumberService}}
The \examplesNameX{Services}{mpmRandomNumberService} application responds to
\requestsNameR{Examples}{Examples}{random} requests from the
\examplesNameR{Clients}{mpmRandomNumberClient} application; it returns a group of random
numbers to the client.\\

Note that the application will exit if the
\serviceNameR[Registry~Service]{RegistryService} is not running.\\

\insertAutoAppParameters
\insertTagDescription{Random Number Input}
The matching client application does not directly use the endpoint name to establish a
connection to the service.\\

The \requestsNameD{Examples}{Examples}{random} request has an argument of the
count of floating\longDash{}point random numbers to be returned as a response to the
request.
\requestsNameE{Examples}{Examples}{random}%
\condPage
If the service is selected for execution from within \emph{Channel Manager}, the following
dialog will be presented:
\objScaledDiagram{mpm_images/launchRandomNumberService}%
{launchServiceRandomNumber}{Launch options for the Random Number service}{1.0}

Once the options are set, the following image will appear in the \emph{Channel Manager}
window when the service has successfully started:
\objScaledDiagram{mpm_images/runningRandomNumberService}%
{serviceRunningRandomNumber}{The \emph{Channel Manager} entity for the Random Number
service}{1.0}
\tertiaryEnd{\examplesNameE{Services}{mpmRandomNumberService}}
\condPage
\tertiaryStart{\examplesNameP{Services}{mpmRecordAsJSONService}}
The \examplesNameX{mpmRecordAsJSONService} application is an Output
service, recording a stream of \yarp{} values as JSON structures to an external file.
The application responds to the standard Output service requests and can be used as a
standalone data generator, without the need for a client connection.\\

The \requestsNameR{\inputOutput}{InputOutput}{configure} request has a single argument,
the file\longDash{}system path to use for the output file.
The path will be used when the input stream is started or restarted.\\

The \requestsNameR{\inputOutput}{InputOutput}{restartStreams} request stops and then
starts the input stream.\\

The \requestsNameR{\inputOutput}{InputOutput}{startStreams} request opens a file to be
used for output, using the configured output file path.\\

The \requestsNameR{\inputOutput}{InputOutput}{stopStreams} request closes the output
file that is being used.\\

Note that the application will exit if the
\serviceNameR[Registry~Service]{RegistryService} is not running.\\

The application has one optional argument \longDash{} the output file path to be used.
\insertAppParameters
\insertTagDescription{Record as JSON Output}
\insertOutputServiceComment\\

The output file path parameter provides the initial output file path; if not specified, a
random path in the system temporary directory will be used.
For Microsoft Windows, the temporary directory being used is ``\textbackslash{}tmp''
while, for Macintosh OS X, it will be ``/tmp'' and, for Linux, it is \TBD.
Note that the output file path can also be set via commands, if the application is
running from a terminal.\\

If the application is running from a terminal and has not been automatically started via
the `\asCode{go}' option, the following commands are available:
\begin{itemize}
\item\cmdItem{?}{display a list of the available commands}
\item\exSp\cmdItem{b}{start the input stream, using the specified output file path to
record the input data}
\item\exSp\cmdItem{c}{configure the service by providing the output file path to be used}
\item\exSp\cmdItem{e}{stop the input stream, which will close the output file}
\item\exSp\cmdItem{q}{quit the application}
\item\exSp\cmdItem{r}{restart the input stream, which will cause the output file to be
closed and a new output file to be opened}
\item\exSp\cmdItem{u}{reset the configuration so that it will be reprocessed when the
input stream is restarted}
\end{itemize}
\condPage
If the service is selected for execution from within \emph{Channel Manager}, the following
dialog will be presented:
\objScaledDiagram{mpm_images/launchRecordAsJSONOutputService}%
{launchServiceRecordAsJSON}{Launch options for the Record As JSON output service}{1.0}

Once the options are set, the following image will appear in the \emph{Channel Manager}
window when the service has successfully started:
\objScaledDiagram{mpm_images/runningRecordAsJSONOutputService}%
{serviceRunningRecordAsJSON}{The \emph{Channel Manager} entity for the Record As JSON
output service}{1.0}
\tertiaryEnd{\examplesNameE{Services}{mpmRecordAsJSONService}}
\condPage
\tertiaryStart{\examplesNameP{Services}{mpmRecordIntegersService}}
The \examplesNameX{Services}{mpmRecordIntegersService} application is an Output
service, recording a stream of integer values to an external file.
The application responds to the standard Output service requests and can be used as a
standalone data generator, without the need for a client connection.\\

The \requestsNameR{\inputOutput}{InputOutput}{configure} request has a single argument,
the file\longDash{}system path to use for the output file.
The path will be used when the input stream is started or restarted.\\

The \requestsNameR{\inputOutput}{InputOutput}{restartStreams} request stops and then
starts the input stream.\\

The \requestsNameR{\inputOutput}{InputOutput}{startStreams} request opens a file to be
used for output, using the configured output file path.\\

The \requestsNameR{\inputOutput}{InputOutput}{stopStreams} request closes the output
file that is being used.\\

Note that the application will exit if the
\serviceNameR[Registry~Service]{RegistryService} is not running.\\

The application has one optional argument \longDash{} the output file path to be used.
\insertAppParameters
\insertTagDescription{Record Integers Output}
\insertOutputServiceComment\\

The output file path parameter provides the initial output file path; if not specified, a
random path in the system temporary directory will be used.
For Microsoft Windows, the temporary directory being used is ``\textbackslash{}tmp''
while, for Macintosh OS X, it will be ``/tmp'' and, for Linux it is \TBD.
Note that the output file path can also be set via commands, if the application is
running from a terminal.\\

If the application is running from a terminal and has not been automatically started via
the `\asCode{go}' option, the following commands are available:
\begin{itemize}
\item\cmdItem{?}{display a list of the available commands}
\item\exSp\cmdItem{b}{start the input stream, using the specified output file path to
record the input data}
\item\exSp\cmdItem{c}{configure the service by providing the output file path to be used}
\item\exSp\cmdItem{e}{stop the input stream, which will close the output file}
\item\exSp\cmdItem{q}{quit the application}
\item\exSp\cmdItem{r}{restart the input stream, which will cause the current output file
to be closed and a new output file to be opened}
\item\exSp\cmdItem{u}{reset the configuration so that it will be reprocessed when the
input stream is restarted}
\end{itemize}
\condPage
If the service is selected for execution from within \emph{Channel Manager}, the following
dialog will be presented:
\objScaledDiagram{mpm_images/launchRecordIntegersOutputService}%
{launchServiceRecordIntegers}{Launch options for the Record Integers output service}{1.0}

Once the options are set, the following image will appear in the \emph{Channel Manager}
window when the service has successfully started:
\objScaledDiagram{mpm_images/runningRecordIntegersOutputService}%
{serviceRunningRecordIntegers}{The \emph{Channel Manager} entity for the Record Integers
output service}{1.0}
\tertiaryEnd{\examplesNameE{Services}{mpmRecordIntegersService}}
\condPage
\tertiaryStart{\examplesNameP{Services}{mpmRunningSumService}}
The \examplesNameX{Services}{mpmRunningSumService} application responds to
requests from the \examplesNameR{Clients}{mpmRunningSumClient} application or\\
\examplesNameR{Adapters}{mpmRunningSumAdapter} or
\examplesNameR{Adapters}{mpmRunningSumAltAdapter} adapters; it returns the running sum
associated with the requesting application or adapter.\\

Note that the application will exit if the
\serviceNameR[Registry~Service]{RegistryService} is not running.
\insertAutoAppParameters
\insertTagDescription{Running Sum}
The matching client application does not directly use the endpoint name to establish a
connection to the service.\\

The \requestsNameD{Examples}{Examples}{addtosum} request updates the running sum
corresponding to the requesting application or adapter and returns the resulting value.
Note that an \requestsNameX{Examples}{Examples}{addtosum} request implicitly starts the
calculation.\\
\requestsNameE{Examples}{Examples}{addtosum}%

The \requestsNameD{Examples}{Examples}{resetsum} request clears the running sum
corresponding to the requesting application or adapter.\\
\requestsNameE{Examples}{Examples}{resetsum}%

The \requestsNameD{Examples}{Examples}{startsum} request starts the running sum
calculation for the requesting application or adapter.
Note that a \requestsNameX{Examples}{Examples}{startsum} request implicitly resets the
calculation, if it has already been started.\\
\requestsNameE{Examples}{Examples}{startsum}%

The \requestsNameD{Examples}{Examples}{stopsum} request stops the running sum
calculation for the requesting application or adapter.
\requestsNameE{Examples}{Examples}{stopsum}%
\condPage
If the service is selected for execution from within \emph{Channel Manager}, the following
dialog will be presented:
\objScaledDiagram{mpm_images/launchRunningSumService}%
{launchServiceRunningSum}{Launch options for the Running Sum service}{1.0}

Once the options are set, the following image will appear in the \emph{Channel Manager}
window when the service has successfully started:
\objScaledDiagram{mpm_images/runningRunningSumService}%
{serviceRunningRunningSum}{The \emph{Channel Manager} entity for the Running Sum service}%
{1.0}
\tertiaryEnd{\examplesNameE{Services}{mpmRunningSumService}}
\tertiaryStart{\examplesNameP{Services}{mpmTruncateFloatService}}
The \examplesNameX{Services}{mpmTruncateFloatService} application is a Filter
service, converting a stream of floating\longDash{}point values into a stream of integers.
The application responds to the standard Filter service requests and can be used as a
standalone data generator, without the need for a client connection.\\

The \requestsNameR{\inputOutput}{InputOutput}{configure} request has no arguments and
does nothing.\\

The \requestsNameR{\inputOutput}{InputOutput}{restartStreams} request stops and then
starts the input and output streams.\\

The \requestsNameR{\inputOutput}{InputOutput}{startStreams} request initiates listening
on the input stream; the output stream is only written to when input is seen, so nothing
special is done with it.\\

The \requestsNameR{\inputOutput}{InputOutput}{stopStreams} request terminates listening
on the input stream, which will cause sending on the output stream to cease as well.\\

Note that the application will exit if the
\serviceNameR[Registry~Service]{RegistryService} is not running.\\

\insertAppParameters
\insertTagDescription{Truncate Float Filter}
\insertFilterServiceComment\\

If the application is running from a terminal and has not been automatically started via
the `\asCode{go}' option, the following commands are available:
\begin{itemize}
\item\cmdItem{?}{display a list of the available commands}
\item\exSp\cmdItem{b}{start the input and output streams}
\item\exSp\cmdItem{c}{configure the service; this has no effect, as the service has no
configurable parameters}
\item\exSp\cmdItem{e}{stop the input and output streams}
\item\exSp\cmdItem{q}{quit the application}
\item\exSp\cmdItem{r}{restart the input and output streams}
\item\exSp\cmdItem{u}{reset the configuration so that it will be reprocessed when the
input and output streams are restarted}
\end{itemize}
\condPage
If the service is selected for execution from within \emph{Channel Manager}, the following
dialog will be presented:
\objScaledDiagram{mpm_images/launchTruncateFloatFilterService}%
{launchServiceTruncateFloat}{Launch options for the Truncate Float filter service}{1.0}

Once the options are set, the following image will appear in the \emph{Channel Manager}
window when the service has successfully started:
\objScaledDiagram{mpm_images/runningTruncateFloatFilterService}%
{serviceRunningTruncateFloat}{The \emph{Channel Manager} entity for the Truncate Float
filter service}{1.0}
\tertiaryEnd{\examplesNameE{Services}{mpmTruncateFloatService}}
\secondaryEnd
\condPage
\secondaryStart{Example~Clients}
The example client applications provided use simple terminal\longDash{}based interfaces to
generate requests for their matching service applications.
\tertiaryStart{\examplesNameP{Clients}{mpmEchoClient}}
The \examplesNameX{Clients}{mpmEchoClient} application reads input from a terminal and
sends it to the \examplesNameR{Services}{mpmEchoService} application using an
\requestsNameR{Miscellaneous}{Miscellaneous}{echo} request; it exits when an empty line is
entered.\\

\insertShortClientParameters{}

Note that the application will also exit if the
\serviceNameR[Registry~Service]{RegistryService} or the
\examplesNameR{Services}{mpmEchoService} application are not running or if the application
is not running from an interactive terminal.
\tertiaryEnd{\examplesNameE{Clients}{mpmEchoClient}}
\tertiaryStart{\examplesNameP{Clients}{mpmRandomNumberClient}}
The \examplesNameX{Clients}{mpmRandomNumberClient} application reads the count of
random numbers desired from a terminal and sends a
\requestsNameR{Examples}{Examples}{random} request to the
\examplesNameR{Services}{mpmRandomNumberService} application; it exits when a zero is
entered for the count of random numbers desired.

\insertShortClientParameters{}

Note that the application will also exit if the
\serviceNameR[Registry~Service]{RegistryService} or the
\examplesNameR{Services}{mpmRandomNumberService} application are not running or if the
application is not running from an interactive terminal.
\tertiaryEnd{\examplesNameE{Clients}{mpmRandomNumberClient}}
\condPage
\tertiaryStart{\examplesNameP{Clients}{mpmRunningSumClient}}
The \examplesNameX{Clients}{mpmRunningSumClient} application reads commands from a
terminal and sends corresponding requests to the
\examplesNameR{Services}{mpmRunningSumService} application.\\

The commands are:
\begin{itemize}
\item\cmdItem{?}{display a list of the available commands}
\item\exSp\cmdItem{+}{read a number from the terminal and send it to the service via an
\requestsNameR{Examples}{Examples}{addtosum} request, to update the running sum for this
client}
\item\exSp\cmdItem{q}{send a \requestsNameR{Examples}{Examples}{stopsum} request to the
service so that it will stop calculating the running sum for this client and then exit the
application}
\item\exSp\cmdItem{r}{send a \requestsNameR{Examples}{Examples}{resetsum} request to the
service so that it will reset the running sum for this client}
\item\exSp\cmdItem{s}{send a \requestsNameR{Examples}{Examples}{startsum} request to the
service so that it will start calculating the running sum for this client}
\end{itemize}

\insertShortClientParameters{}

Note that the application will also exit if the
\serviceNameR[Registry~Service]{RegistryService} or the
\examplesNameR{Services}{mpmRunningSumService} application are not running or if the
application is not running from an interactive terminal.
\tertiaryEnd{\examplesNameE{Clients}{mpmRunningSumClient}}
\secondaryEnd
\condPage
\secondaryStart{Example~Adapters}
Adapters are applications that use a client object to connect to an \mplusm{} service,
routing input from one or more input \yarp{} network connection via the client object to
the service and returning results from the client to one or more output \yarp{} network
connections.\\

The example adapters demonstrate using a single input \yarp{} network connection
[\examplesNameR{Adapters}{mpmRandomNumberAdapter} and
\examplesNameR{Adapters}{mpmRunningSumAltAdapter}] or multiple input \yarp{} network
connections [\examplesNameR{Adapters}{mpmRunningSumAdapter}].\\

Adapters have ten optional parameters:
\begin{itemize}
\item\optItem{a}{}{args}{display the argument descriptions for the executable and leave
\longDash{} note that this option is primarily for use by the \emph{Channel Manager}
application}
\item\exSp\optItem{c}{}{channel}{display the endpoint name after applying all other
options and leave}
\item\exSp\optItem{e}{v}{endpoint}{specifies an alternative endpoint name `\textit{v}' to
be used}
\item\exSp\optItem{g}{}{go}{indicates that the adapter is to be started immediately}
\item\exSp\optItem{h}{}{help}{display the list of optional parameters and arguments and
leave}
\item\exSp\optItem{i}{}{info}{display the type of the executable, the valid options, the
matching criteria used to locate the service to which the adapter will attach and a
description of the executable and leave \longDash{} note that this option is primarily
for use by the \emph{Channel Manager} application}
\item\exSp\optItem{p}{v}{port}{specifies the port number `\textit{v}' to be used, if a
non\longDash{}default port is desired}
\item\exSp\optItem{r}{}{report}{report the adapter metrics when the application exits}
\item\exSp\optItem{t}{v}{tag}{specifies the tag `\textit{v}' to be used as part of the
adapter name}
\item\exSp\optItem{v}{}{vers}{display the version and copyright information and leave}
\end{itemize}

Note that adapters normally have no direct interface.
That is, they are `faceless' applications and perform their operations without user
interaction.
\condPage
\tertiaryStart{\examplesNameP{Adapters}{mpmRandomNumberAdapter}}
The \examplesNameX{Adapters}{mpmRandomNumberAdapter} application reads the count of
random numbers desired from an input \yarp{} network connection and sends a
\requestsNameR{Examples}{Examples}{random} request to the
\examplesNameR{Services}{mpmRandomNumberService} application.
The random numbers returned from the service are sent out an output \yarp{} network
connection.
Note that the application will also exit if the
\serviceNameR[Registry~Service]{RegistryService} or the
\examplesNameR{Services}{mpmRandomNumberService} application are not running.\\

If the adapter is selected for execution from within \emph{Channel Manager}, the following
dialog will be presented:
\objScaledDiagram{mpm_images/launchRandomNumberAdapter}%
{launchAdapterRandomNumber}{Launch options for the Random Number adapter}{1.0}

Once the options are set, the following image will appear in the \emph{Channel Manager}
window when the adapter has successfully started:
\objScaledDiagram{mpm_images/runningRandomNumberAdapter}%
{adapterRunningRandomNumber}{The \emph{Channel Manager} entity for the Random Number
adapter}{1.0}
\tertiaryEnd{\examplesNameE{Adapters}{mpmRandomNumberAdapter}}
\tertiaryStart{\examplesNameP{Adapters}{mpmRunningSumAdapter}}

The \examplesNameX{Adapters}{mpmRunningSumAdapter} application reads  values from an input
`data' \yarp{} network connection and commands from an input `control' \yarp{} network
connection and sends corresponding requests to the
\examplesNameR{Services}{mpmRunningSumService} application.\\

If a numeric value is read from the `data' input \yarp{} network connection, it is sent to
the service via an \requestsNameR{Examples}{Examples}{addtosum} request, to update the
running sum for this adapter.
If a string is read from the `control' input \yarp{} network connection, it is considered
to be one of the following commands:
\begin{itemize}
\item\textbf{q:} exit the application.
\item\exSp\textbf{r:} send a \requestsNameR{Examples}{Examples}{resetsum} request to the
service so that it will reset the running sum for this adapter.
\item\exSp\textbf{s:} send a \requestsNameR{Examples}{Examples}{startsum} request to the
service so that it will start calculating the running sum for this adapter.
\item\exSp\textbf{x:} send a \requestsNameR{Examples}{Examples}{stopsum} request to the
service so that it will stop calculating the running sum for this adapter.
\end{itemize}
The running sums returned from the service are sent out an output \yarp{} network
connection.\\

Note that the application will also exit if the
\serviceNameR[Registry~Service]{RegistryService} or the
\examplesNameR{Services}{mpmRunningSumService} application are not running.\\

If the adapter is selected for execution from within \emph{Channel Manager}, the following
dialog will be presented:
\objScaledDiagram{mpm_images/launchRunningSumAdapter}%
{launchAdapterRunningSum}{Launch options for the Running Sum adapter}{1.0}
\condPage{}
Once the options are set, the following image will appear in the \emph{Channel Manager}
window when the adapter has successfully started:
\objScaledDiagram{mpm_images/runningRunningSumAdapter}%
{adapterRunningRunningSum}{The \emph{Channel Manager} entity for the Running Sum adapter}%
{1.0}
\tertiaryEnd{\examplesNameE{Adapters}{mpmRunningSumAdapter}}
\tertiaryStart{\examplesNameP{Adapters}{mpmRunningSumAltAdapter}}
The \examplesNameX{Adapters}{mpmRunningSumAltAdapter} application reads  values and
commands from an input \yarp{} network connection and sends corresponding requests to the
\examplesNameR{Services}{mpmRunningSumService} application.\\

If a numeric value is read from the input \yarp{} network connection, it is sent to the
service via an \requestsNameR{Examples}{Examples}{addtosum} request, to update the running
sum for this adapter.
If a string is read from the input \yarp{} network connection, is is considered to be one
of the following commands:
\begin{itemize}
\item\textbf{q:} exit the application.
\item\exSp\textbf{r:} send a \requestsNameR{Examples}{Examples}{resetsum} request to the
service so that it will reset the running sum for this adapter.
\item\exSp\textbf{s:} send a \requestsNameR{Examples}{Examples}{startsum} request to the
service so that it will start calculating the running sum for this adapter.
\item\exSp\textbf{x:} send a \requestsNameR{Examples}{Examples}{stopsum} request to the
service so that it will stop calculating the running sum for this adapter.
\end{itemize}
The running sums returned from the service are sent out an output \yarp{} network
connection.\\

Note that the application will also exit if the
\serviceNameR[Registry~Service]{RegistryService} or the
\examplesNameR{Services}{mpmRunningSumService} application are not running.
\condPage{}
If the adapter is selected for execution from within \emph{Channel Manager}, the following
dialog will be presented:
\objScaledDiagram{mpm_images/launchRunningSumAltAdapter}%
{launchAdapterRunningSumAlt}{Launch options for the Running Sum alternate adapter}{1.0}

Once the options are set, the following image will appear in the \emph{Channel Manager}
window when the adapter has successfully started:
\objScaledDiagram{mpm_images/runningRunningSumAltAdapter}%
{adapterRunningRunningSumAlt}{The \emph{Channel Manager} entity for the Running Sum
alternate adapter}{1.0}
\tertiaryEnd{\examplesNameE{Adapters}{mpmRunningSumAltAdapter}}
\secondaryEnd
\appendixEnd{}
