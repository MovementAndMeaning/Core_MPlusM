\ProvidesFile{MMclasses.tex}[v1.0.0]
\primaryStart[ClassesAndTheirInterfaces]{Classes~and~Their~Interfaces}
The \mplusm{} classes are organized into seven namespaces:
\begin{itemize}
\item\textbf{\mplusm{}::Common} classes that support all the \mplusm{} activities
\item\exSp\textbf{\mplusm{}::Example} classes that demonstrate features of \mplusm{}
\item\exSp\textbf{\mplusm{}::Parser} classes that are used by the
\requestsNameR{Registry~Service}{RegistryService}{match} request of the
\serviceNameR[\RS]{RegistryService}
\item\exSp\textbf{\mplusm{}::Registry} classes that work with the internal database to
manage the \mplusm{} services
\item\exSp\textbf{\mplusm{}::RequestCounter} classes that support measuring the
performance of an \mplusm{} installation
\item\exSp\textbf{\mplusm{}::Test} classes that are used during unit\longDash{}testing of
the \mplusm{} source code
\item\exSp\textbf{\mplusm{}::Utilities} classes that are used only with the \mplusm{}
utility programs
\end{itemize}
There are 45 classes or structures in the \textbf{\mplusm{}::Common} namespace, 34 in
\textbf{\mplusm{}::Example}, 8 in \textbf{\mplusm{}::Parser}, 11 in
\textbf{\mplusm{}::Registry}, 6 in \textbf{\mplusm{}::RequestCounter}, 20 in
\textbf{\mplusm{}::Test} and 11 in \textbf{\mplusm{}::Utilities}.
The following sections describe the most critical of the classes in the
\textbf{\mplusm{}::Common} namespace \longDash{} many of the other classes, in all the
namespaces, derive from these classes. 
\secondaryStart{BaseAdapterData}\classNameM{Common}{BaseAdapterData}
The class \classNameX{Common}{BaseAdapterData} is used to provide common methods and
variables for data that is shared between instances of derived classes of
\classNameR{Common}{BaseAdapterService} and instances of derived classes of
\classNameR{Common}{BaseInputHandler}.
In particular, the shared data includes a reference to the \classNameR{Common}{BaseClient}
that is being communicated with by the adapter and the primary output stream to use when
responding to input data.
\secondaryEnd{\classNameE{Common}{BaseAdapterData}}
\secondaryStart{BaseAdapterService}\classNameM{Common}{BaseAdapterService}
Instances of derived classes of \classNameX{Common}{BaseAdapterService} are responsible
for managing \mplusm{} communication between one or more standard \yarp{} ports and an
instance of a derived class of \classNameR{Common}{BaseClient}; for example, an instance
of the class \asCode{M+M::Example::RandomNumberAdapter} uses an instance of the class
\asCode{M+M::Example::RandomNumberClient} to make requests to an instance of the class
\asCode{M+M::Example::RandomNumberService}.\\

The bulk of the mechanisms needed to perform this are provided by the
\classNameX{Common}{BaseAdapterService} class itself, and the derived classes need only
provide service\longDash{}specific routines to handle their inputs.
\secondaryEnd{\classNameE{Common}{BaseAdapterService}}
\secondaryStart{BaseArgumentDescriptor}\classNameM{Utilities}{BaseArgumentDescriptor}
Instances of derived classes of \classNameX{Utilities}{BaseArgumentDescriptor} are
responsible for serializing argument descriptions, providing argument prompts and
validating user input.
\secondaryEnd{\classNameE{Utilities}{BaseArgumentDescriptor}}
\condPage
\secondaryStart{BaseChannel}\classNameM{Common}{BaseChannel}
The class \classNameX{Common}{BaseChannel} provides common resources and methods for more
specialized channels.
\secondaryEnd{\classNameE{Common}{BaseChannel}}
\secondaryStart{BaseClient}\classNameM{Common}{BaseClient}
Instances of derived classes of \classNameX{Common}{BaseClient} are responsible for
mediating \mplusm{} communication with corresponding instances of derived classes of
\classNameR{Common}{BaseService}.\\

The bulk of the mechanisms needed to perform this are provided by the
\classNameX{Common}{BaseClient} class itself, and the derived classes need only provide
service\longDash{}specific routines, such as \asCode{getOneRandomNumber()}, which is a
method of the \asCode{M+M::Example::RandomNumberClient} class.\\

The following methods are available to code that uses \classNameX{Common}{BaseClient}
classes:
\begin{itemize}
\item\asBoldCode{FindMatchingService} This static function is provided for the use of
utility programs that need to identify active services
\item\exSp\asBoldCode{connectToService} This method is use by clients to establish a
connection to the endpoint of their companion service
\item\exSp\asBoldCode{disconnectFromService} This method is used by clients to release the
connection to their companion service
\item\exSp\asBoldCode{findService} This method is used by clients to locate the channel to
be used for communication with their companion service
\item\exSp\asBoldCode{reconnectIfDisconnected} This method is used by clients to
re\longDash{}establish a dropped connection with their companion service
\item\exSp\asBoldCode{send} This method is used by clients to send a request to their
companion service
\end{itemize}
\secondaryEnd{\classNameE{Common}{BaseClient}}
\secondaryStart{BaseContext}\classNameM{Common}{BaseContext}
Instances of derived classes of \classNameR{Common}{BaseService} that have \yarp{}
network connections with persistent state use instances of
derived classes of \classNameX{Common}{BaseContext} to track the information that is
retained from each request for the following request.
The \classNameX{Common}{BaseContext} class itself provides no functionality.
\secondaryEnd{\classNameE{Common}{BaseContext}}
\secondaryStart{BaseFilterService}\classNameM{Common}{BaseFilterService}
The \classNameX{Common}{BaseFilterService} class is a subclass of
\classNameR{Common}{BaseInputOutputService} that provides implementations of several of
the required methods of its superclass.\\

Classes derived from \classNameX{Common}{BaseFilterService} are expected to fill in the
`protected' member variables \asCode{\textunderscore{}inDescriptions} and
\asCode{\textunderscore{}outDescriptions}, which are used by the methods
\asCode{setUpInputStreams} and \asCode{setUpOutputStreams}, respectively.
The method \asCode{setUpStreamDescriptions} must be implemented in classes derived from
\classNameX{Common}{BaseFilterService} and should fill in
\asCode{\textunderscore{}inDescriptions} and \asCode{\textunderscore{}outDescriptions}.
\secondaryEnd{\classNameE{Common}{BaseFilterService}}
\secondaryStart{BaseInputHandler}\classNameM{Common}{BaseInputHandler}
Instances of derived classes of \classNameX{Common}{BaseInputHandler} are responsible for
responding to input data on \yarp{} network connections.
They must implement the method \asCode{handleInput}, which is called when a complete
message has been received on a connection.
\secondaryEnd{\classNameE{Common}{BaseInputHandler}}
\secondaryStart{BaseInputHandlerCreator}\classNameM{Common}{BaseInputHandlerCreator}
Instances of derived classes of \classNameX{Common}{BaseInputHandlerCreator} provide
`factories' to generate instances of derived classes of
\classNameR{Common}{BaseInputHandler}.
\secondaryEnd{\classNameE{Common}{BaseInputHandlerCreator}}
\secondaryStart{BaseInputOutputService}\classNameM{Common}{BaseInputOutputService}
The \classNameX{Common}{BaseInputOutputService} class is designed to support easy access
to multiple instances of the same service, each of which can be used to provide an
interface to an external device or to a simple flow\longDash{}through operation.\\

Classes derived from \classNameX{Common}{BaseInputOutputService} are required to provide
implementations of the following methods:
\begin{itemize}
\item\asBoldCode{configure} This method is invoked when a
\requestsNameR{\inputOutput}{InputOutput}{configure} request is sent to the service; it
is responsible for setting attributes of the input and output streams of the service
\item\exSp\asBoldCode{restartStreams} This method is invoked when a
\requestsNameR{\inputOutput}{InputOutput}{restartStreams} request is sent to the
service; it is responsible for stopping and then starting the input and output streams of
the service
\item\exSp\asBoldCode{setUpStreamDescriptions} This method is invoked when the service is
started and it is responsible for providing a list of port names and protocols for use by
the current instance of the service
\item\exSp\asBoldCode{startStreams} This method is invoked when a
\requestsNameR{\inputOutput}{InputOutput}{startStreams} request is sent to the service;
it is responsible for starting the input and output streams of the service
\item\exSp\asBoldCode{stopStreams} This method is invoked when a
\requestsNameR{\inputOutput}{InputOutput}{stopStreams} request is sent to the service;
it is responsible for stopping the input and output streams of the service
\end{itemize}
The \asCode{configure}, \asCode{restartStreams}, \asCode{startStreams} and
\asCode{stopStreams} methods can also be invoked by code that instantiates the service.
\secondaryEnd{\classNameE{Common}{BaseInputOutputService}}
\secondaryStart{BaseInputService}\classNameM{Common}{BaseInputService}
The \classNameX{Common}{BaseInputService} class is a subclass of
\classNameR{Common}{BaseInputOutputService} that provides implementations of several of
the required methods of its superclass and is specialized to only support output
streams.\\

Classes derived from \classNameX{Common}{BaseInputService} are expected to fill in the
`protected' member variable \asCode{\textunderscore{}outDescriptions}, which
is used by the method \asCode{setUpOutputStreams}.
The method \asCode{setUpStreamDescriptions} must be implemented in classes derived from
\classNameX{Common}{BaseInputService} and should fill in
\asCode{\textunderscore{}outDescriptions}.
\secondaryEnd{\classNameE{Common}{BaseInputService}}
\secondaryStart{BaseMatcher}\classNameM{Common}{BaseMatcher}
The \classNameX{Common}{BaseMatcher} class represents a `handle' to a top\longDash{}down
pattern matcher used by the \serviceNameR[\RS]{RegistryService} to convert the arguments
of \requestsNameR{Registry~Service}{RegistryService}{match} requests into \asCode{SQL}
operations.
The `outermost' matcher that is parsed from the original request is used to generate the
\asCode{SQL} \longDash{} the \classNameX{Common}{BaseMatcher} class acts as a
placeholder.
\secondaryEnd{\classNameE{Common}{BaseMatcher}}
\secondaryStart{BaseNameValidator}\classNameM{Parser}{BaseNameValidator}
Instances of derived classes of \classNameX{Parser}{BaseNameValidator} class provide
field name handling.
\secondaryEnd{\classNameE{Parser}{BaseNameValidator}}
\secondaryStart{BaseOutputService}\classNameX{Common}{BaseOutputService}
The \classNameX{Common}{BaseOutputService} class is a subclass of
\classNameR{Common}{BaseInputOutputService} that provides implementations of several of
the required methods of its superclass and is specialized to only support input streams.\\

Classes derived from \classNameX{Common}{BaseOutputService} are expected to fill in the
`protected' member variable \asCode{\textunderscore{}inDescriptions}, which is used by the
method \asCode{setUpInputStreams}.
The method \asCode{setUpStreamDescriptions} must be implemented in classes derived from
\classNameX{Common}{BaseOutputService} and should fill in
\asCode{\textunderscore{}inDescriptions}.
\secondaryEnd{\classNameE{Common}{BaseOutputService}}
\secondaryStart{BaseRequestHandler}\classNameM{Common}{BaseRequestHandler}
The \classNameX{Common}{BaseRequestHandler} class is designed to provide a consistent
request handling mechanism for \mplusm{} services.\\

Classes derived from \classNameX{Common}{BaseRequestHandler} are required to provide
implementations of the following methods:
\begin{itemize}
\item\asBoldCode{fillInAliases} This method is invoked when a service is registering its
request handlers; it is responsible for providing a list of alternative names for the
request
\item\exSp\asBoldCode{fillInDescription} This method is invoked when a service is
responding to an \requestsNameR{Basic}{Basic}{info} or a
\requestsNameR{Basic}{Basic}{list} request; it is responsible for providing a detailed
description of the request
\item\exSp\asBoldCode{processRequest} This method is invoked with the request is seen by
the service and it is responsible for performing the operations that correspond to the
request
\end{itemize}
\secondaryEnd{\classNameE{Common}{BaseRequestHandler}}
\secondaryStart{BaseService}\classNameM{Common}{BaseService}
Instances of derived classes of \classNameX{Common}{BaseService} are responsible for
responding to requests from corresponding instances of derived classes of
\classNameR{Common}{BaseClient}.
As well, they can respond on their secondary \yarp{} network connections to requests or
process data from the secondary connections.

Requests are mainly handled by instances of derived classes of the
\classNameR{Common}{BaseRequestHandler} class that are owned by the service
instance; the \classNameX{Common}{BaseService} class manages the standard request
handlers.\\

Classes derived from \classNameX{Common}{BaseService} should implement the
following methods:
\begin{itemize}
\item\asBoldCode{attachRequestHandlers} This method should be called in the constructor
for the service class and should register handlers for its custom requests
\item\exSp\asBoldCode{detachRequestHandlers} This method should be called in the
destructor for the service class and should unregister handlers for its custom requests
\item\exSp\asBoldCode{start} This method is invoked by the service application and should
invoke the same method of its parent class; it should set up any special conditions that
are needed by the service
\item\exSp\asBoldCode{stop} This method is invoked by the service application and should
invoke the same method of its parent class; it should release any resources used by the
service
\end{itemize}
\condPage{}
The following methods are available to code that uses \classNameX{Common}{BaseService}
classes:
\begin{itemize}
\item\asBoldCode{gatherMetrics} This method returns a list of the metrics for the
connections associated with the service
\item\exSp\asBoldCode{getEndpoint} This method provides access to the
\classNameR{Common}{Endpoint} that is provided for the service and is useful for
establishing a connection to an `embedded' service
\end{itemize}
The following convenience methods are available to derived classes of
\classNameX{Common}{BaseService}:
\begin{itemize}
\item\asBoldCode{addContext} This method is used to record the data that holds the
persistent state for a connection
\item\exSp\asBoldCode{clearContexts} This method is used to remove all the persistent
state for connections
\item\exSp\asBoldCode{findContext} This method is used to retrieve the data that holds the
persistent state of a connection
\item\exSp\asBoldCode{registerRequestHandler} This method is used to register the object
that will handle a request
\item\exSp\asBoldCode{removeContext} This method is used to remove the persistent state of
a connection
\item\exSp\asBoldCode{setDefaultRequestHandler} This method is used to register the object
that will handle unknown requests
\item\exSp\asBoldCode{unregisterRequestHandler} This method is used to unregister a request
handler
\end{itemize}
\secondaryEnd{\classNameE{Common}{BaseService}}
\secondaryStart{Endpoint}\classNameM{Common}{Endpoint}
Instances of the \classNameX{Common}{Endpoint} class manage a \yarp{} network connection
for a service.
They are used to simplify the communication mechanisms that each service must provide and
to standardize on the mid\longDash{}level protocols that \mplusm{} services and clients
share.
\secondaryEnd{\classNameE{Common}{Endpoint}}
\secondaryStart{ServiceRequest}\classNameM{Common}{ServiceRequest}
Instances of the \classNameX{Common}{ServiceRequest} class are used to represent requests
in the mid\longDash{}level protocols of \mplusm{} services in a uniform format.
\secondaryEnd{\classNameE{Common}{ServiceRequest}}
\secondaryStart{ServiceResponse}\classNameM{Common}{ServiceResponse}
Instances of the \classNameX{Common}{ServiceResponse} class are used to represent
responses in the mid\longDash{}level protocols of \mplusm{} services in a uniform format.
\secondaryEnd{\classNameE{Common}{ServiceResponse}}
\primaryEnd{}
