\ProvidesFile{services.tex}[v1.0.0]
\primaryStart[ServicesAndTheirProtocols]{Services~and~Their~Protocols}
An \mplusm{} installation consists of a number of service applications and their companion
client applications, communicating via \mplusm{} requests and responses on a \yarp{}
network.\\

There is one special service, the \serviceNameR[Service~Registry]{ServiceRegistry}, which
manages information about all other active services; all services register themselves
with the \serviceNameR[Service~Registry]{ServiceRegistry} so that client applications and
utilities can get information about the service.\\

All services and clients utilize a request / response protocol that is defined by the
\mplusm{} \compLang{C++} core classes.\\

The standard requests are in 4 groups:
\begin{itemize}
\item \textbf{\secondaryRef{BasicRequests}{Basic~Requests:}} requests that support the
fundamental \mplusm{} service mechanisms
\item \textbf{\secondaryRef{ServiceRegistryRequests}{Service~Registry~Requests:}} requests
that are specific to the \serviceNameR[Service~Registry]{ServiceRegistry}
\item \textbf{\secondaryRef{InputOutputRequests}{Input~/~Output~Requests:}} requests that
are specific to Input~/~Output services
\item \textbf{\secondaryRef{MiscellaneousRequests}{Miscellaneous~Requests:}} other
requests
\end{itemize}
Note that the \secondaryRef{BasicRequests}{Basic~Requests} are part of every service and
automatically supported in the base class of all services,
\begin{Large}\textbf{TBD--Base Service--TBD}\end{Large}.\\

When a client sends a request to a service, it can optionally request a response from the
service.
If no response is requested, the request is processed but no response is sent.
\secondaryStart[StandardServices]{Standard~Services}
Their are two standard services that are always part of an \mplusm{} installation.
\utilityNameR{mpmRequestCounterService} and its companion application
\utilityNameR{mpmRequestCounterClient} are described elsewhere, and the
\serviceNameR[Service~Registry]{ServiceRegistry} is described in the following section.
\tertiaryStart{mpmRegistryService (also known as the
\serviceNameD[Service~Registry]{ServiceRegistry})}
The \serviceNameX[Service~Registry]{ServiceRegistry} application is a background service
that is used to manage other services and their connections.
Its primary purpose is to serve as a repository of information on the active services in
an \mplusm{} installation.\\

It uses an internal database, as described in the
\appendixRef{InternalDatabaseStructure}{Internal~Database~Structure} appendix.
It responds to the requests in the
\secondaryRef{ServiceRegistryRequests}{Service~Registry~Requests} group.\\

Note that only one copy of the \serviceNameX[Service~Registry]{ServiceRegistry}
application can be running at a time in an \mplusm{} installation, due to its fixed
\yarp{} network port.
\tertiaryEnd{\serviceNameE[Service~Registry]{mpmRegistryService}}
\secondaryEnd{}
\secondaryStart[BasicRequests]{Basic~Requests}
These requests are implemented in all \mplusm{} services.
They constitute the fundamental mechanism that is used by the
\serviceNameR[Service~Registry]{ServiceRegistry} to identify each active service.
The base class for request handlers is
\begin{Large}\textbf{TBD--Base Request Handler--TBD}\end{Large}.
\tertiaryStart{\requestsNameD{Basic}{Basic}{channels}}
The \requestsNameX{Basic}{Basic}{channels} request returns a list of the secondary input
channels and a list of the secondary output channels for its service.\\

It is used by the \utilityNameR{mpmServiceLister} and
\utilityNameR[Channel~Manager]{ChannelManager} utilities to determine the active \yarp{}
network connections for each service.
\tertiaryEnd{\requestsNameE{Basic}{Basic}{channels}}
\tertiaryStart{\requestsNameD{Basic}{Basic}{clients}, alias:\ %
\requestsNameD{Basic}{Basic}{c}}
The \requestsNameX{Basic}{Basic}{clients} request returns a list of the active clients of
its service, if the service has \yarp{} network connections with persistent state.\\

It is used by the \utilityNameR{mpmClientList} utility to display a list of the clients
of services that have \yarp{} network connections with state.
\tertiaryEnd{\requestsNameE{Basic}{Basic}{clients}}
\tertiaryStart{\requestsNameD{Basic}{Basic}{detach}}
The \requestsNameX{Basic}{Basic}{detach} request is used by clients of a service to
indicate that they are no longer actively communicating with the service.\\

It is used by all client applications and adapters, such as
\examplesNameR{Clients}{mpmEchoClient},
\examplesNameR{Adapters}{mpmRandomNumberAdapter},\\
\examplesNameR{Clients}{mpmRandomNumberClient}, \utilityNameR{mpmRequestCounterClient},
\examplesNameR{Adapters}{mpmRunningSumAdapter},
\examplesNameR{Adapters}{mpmRunningSumAltAdapter} and
\examplesNameR{Clients}{mpmRunningSumClient}, to cleanly disconnect from their
corresponding service.
\tertiaryEnd{\requestsNameE{Basic}{Basic}{detach}}
\tertiaryStart{\requestsNameD{Basic}{Basic}{info}, alias:\ %
\requestsNameD{Basic}{Basic}{i}}
The \requestsNameX{Basic}{Basic}{info} request returns details about a single request
handled by a service.
The request details include the standard name of the request, the version number of the
request handler for the request, search keywords that can be used when matching requests,
a description of the request as well as representations of the expected arguments for the
request and output of the request.\\

It is used by the \utilityNameR{mpmRequestInfo} utility to gather information about a
request available from an active service.
\tertiaryEnd{\requestsNameE{Basic}{Basic}{info}}
\tertiaryStart{\requestsNameD{Basic}{Basic}{list}, alias:\ %
\requestsNameD{Basic}{Basic}{l}}
The \requestsNameX{Basic}{Basic}{list} request returns details about one or more requests
handled by a service.
The request details include the standard name of the request, the version number of the
request handler for the request, search keywords that can be used when matching requests,
a description of the request as well as representations of the expected arguments for the
request and output of the request.\\

It is used by the \serviceNameR[Service~Registry]{ServiceRegistry} and the
\utilityNameR{mpmRequestInfo} utility to gather information about the requests available
from each active service.
\tertiaryEnd{\requestsNameE{Basic}{Basic}{list}}
\tertiaryStart{\requestsNameD{Basic}{Basic}{name}, alias:\ %
\requestsNameD{Basic}{Basic}{n}}
The \requestsNameX{Basic}{Basic}{name} request returns details about its service, in the
form of the canonical name of the service, its description, its kind, the path to the
executable for the service and a description of the requests for the service.\\

It is used by the \serviceNameR[Service~Registry]{ServiceRegistry} and the
\utilityNameR{mpmServiceLister} and \utilityNameR[Channel~Manager]{ChannelManager}
utilities to collect basic information about each service.
\tertiaryEnd{\requestsNameE{Basic}{Basic}{name}}
\secondaryEnd{}
\secondaryStart[ServiceRegistryRequests]{Service~Registry~Requests}
The requests in this group are used exclusively by the
\serviceNameR[Service~Registry]{ServiceRegistry} application to manage its internal
database and to respond to information requests from client applications.
\tertiaryStart{\requestsNameD{Service~Registry}{ServiceRegistry}{associate}}
The \requestsNameX{Service~Registry}{ServiceRegistry}{associate} request is sent by
adapter applications to indicate that a particular \yarp{} network connection is part of
the same process as the client connection that sent the request.\\

It is used by the adapter applications \examplesNameR{Adapters}{mpmRandomNumberAdapter},
\examplesNameR{Adapters}{mpmRunningSumAdapter} and\\
\examplesNameR{Adapters}{mpmRunningSumAltAdapter} to indicate the input and output
\yarp{} network connections that are part of the application.
\tertiaryEnd{\requestsNameE{Service~Registry}{ServiceRegistry}{associate}}
\tertiaryStart{\requestsNameD{Service~Registry}{ServiceRegistry}{disassociate}}
The \requestsNameX{Service~Registry}{ServiceRegistry}{disassociate} request is sent by
adapter applications to indicate that a particular \yarp{} network connection is no
longer part of the same process as the client connection that sent the request.
This is normally done as part of the shutdown process for the application.\\

It is used by the adapter applications \examplesNameR{Adapters}{mpmRandomNumberAdapter},
\examplesNameR{Adapters}{mpmRunningSumAdapter} and\\
\examplesNameR{Adapters}{mpmRunningSumAltAdapter} to indicate that all the input and
output \yarp{} network connections that are part of the application are no longer active.
\tertiaryEnd{\requestsNameE{Service~Registry}{ServiceRegistry}{disassociate}}
\tertiaryStart{\requestsNameD{Service~Registry}{ServiceRegistry}{getAssociates}}
The \requestsNameX{Service~Registry}{ServiceRegistry}{getAssociates} request returns
a list of the input \yarp{} network connections and a list of the output \yarp{} network
connections for a given connection; if the requested port corresponds to a secondary
\yarp{} network connection, then the request returns the client connection that it is
associated with.\\

It is used by the \utilityNameR{mpmPortLister} and
\utilityNameR[Channel~Manager]{ChannelManager} utilities to identify relationships
between \yarp{} network connections.
\tertiaryEnd{\requestsNameE{Service~Registry}{ServiceRegistry}{getAssociates}}
\tertiaryStart{\requestsNameD{Service~Registry}{ServiceRegistry}{match}, alias:\ %
\requestsNameD{Service~Registry}{ServiceRegistry}{find}}
The \requestsNameX{Service~Registry}{ServiceRegistry}{match} request returns a list of
input \yarp{} network connections or service names for services that match the criteria
provided as an argument to the request.\\

It is used by all client and adapter applications to identify the service to which they
need to connect as well as by the \utilityNameR{mpmClientList},
\utilityNameR{mpmPortLister}, \utilityNameR{mpmRequestInfo},
\utilityNameR{mpmServiceLister} and \utilityNameR[Channel~Manager]{ChannelManager}
utilities to identify services in order to gather information about them.\\

For details on the syntax used with the criteria, see the 
\appendixRef{ServiceMatchSyntax}{Service~Match~Syntax} appendix.
\tertiaryEnd{\requestsNameE{Service~Registry}{ServiceRegistry}{match}}
\tertiaryStart{\requestsNameD{Service~Registry}{ServiceRegistry}{ping}}
The \requestsNameX{Service~Registry}{ServiceRegistry}{ping} request is sent by each
active service (except the \serviceNameR[Service~Registry]{ServiceRegistry}) to indicate
that it is still 'alive'.\\

Sending a \requestsNameX{Service~Registry}{ServiceRegistry}{ping} request results in a
sequence of requests being sent from the \serviceNameR[Service~Registry]{ServiceRegistry}
to the active service if the information on the service is no longer in the internal
database.
This will happen if the service is considered to be 'stale' -- it has not sent a
\requestsNameX{Service~Registry}{ServiceRegistry}{ping} request often enough to be
remembered.
The sequence of requests sent to the active service is described in the 
\appendixRef{RegistrationSequence}{Registration~Sequence} appendix.
\tertiaryEnd{\requestsNameE{Service~Registry}{ServiceRegistry}{ping}}
\tertiaryStart{\requestsNameD{Service~Registry}{ServiceRegistry}{register}, alias:\ %
\requestsNameD{Service~Registry}{ServiceRegistry}{remember}}
The \requestsNameX{Service~Registry}{ServiceRegistry}{register} request is sent by each
active service (except the \serviceNameR[Service~Registry]{ServiceRegistry}) when it
starts execution.\\

Sending a \requestsNameX{Service~Registry}{ServiceRegistry}{register} request results in a
sequence of requests being sent from the \serviceNameR[Service~Registry]{ServiceRegistry}
to the active service.
The sequence of requests sent to the active service is described in the 
\appendixRef{RegistrationSequence}{Registration~Sequence} appendix.
\tertiaryEnd{\requestsNameE{Service~Registry}{ServiceRegistry}{register}}
\tertiaryStart{\requestsNameD{Service~Registry}{ServiceRegistry}{unregister}, alias:\ %
\requestsNameD{Service~Registry}{ServiceRegistry}{forget}}
The \requestsNameX{Service~Registry}{ServiceRegistry}{unregister} request is sent by each
service (except the \serviceNameR[Service~Registry]{ServiceRegistry}) just before it stops
execution.\\

Upon receiving the \requestsNameX{Service~Registry}{ServiceRegistry}{unregister} request,
the \serviceNameR[Service~Registry]{ServiceRegistry} removes all entries from the internal
database that are related to the service being removed.
\tertiaryEnd{\requestsNameE{Service~Registry}{ServiceRegistry}{unregister}}
\secondaryEnd{}
\secondaryStart[InputOutputRequests]{Input~/~Output~Requests}

			||TBD TBD TBD||

			||/TBD TBD TBD||

\tertiaryStart{\requestsNameD{Input~/~Output}{InputOutput}{configure}}
The \requestsNameX{Input~/~Output}{InputOutput}{configure} request

			||TBD TBD TBD||

			||/TBD TBD TBD||

\tertiaryEnd{}
\tertiaryStart{\requestsNameD{Input~/~Output}{InputOutput}{restartStreams}}
The \requestsNameX{Input~/~Output}{InputOutput}{restartStreams} request

			||TBD TBD TBD||

			||/TBD TBD TBD||

\tertiaryEnd{\requestsNameE{Input~/~Output}{InputOutput}{restartStreams}}
\tertiaryStart{\requestsNameD{Input~/~Output}{InputOutput}{startStreams}}
The \requestsNameX{Input~/~Output}{InputOutput}{startStreams} request

			||TBD TBD TBD||

			||/TBD TBD TBD||

\tertiaryEnd{\requestsNameE{Input~/~Output}{InputOutput}{startStreams}}
\tertiaryStart{\requestsNameD{Input~/~Output}{InputOutput}{stopStreams}}
The \requestsNameX{Input~/~Output}{InputOutput}{stopStreams} request

			||TBD TBD TBD||

			||/TBD TBD TBD||

\tertiaryEnd{\requestsNameE{Input~/~Output}{InputOutput}{stopStreams}}
\secondaryEnd{}
\secondaryStart[MiscellaneousRequests]{Miscellaneous~Requests}

			||TBD TBD TBD||

			||/TBD TBD TBD||

\tertiaryStart{\requestsNameD{Miscellaneous}{Miscellaneous}{count}}
The \requestsNameX{Miscellaneous}{Miscellaneous}{count} request

			||TBD TBD TBD||

			||/TBD TBD TBD||

\tertiaryEnd{\requestsNameE{Miscellaneous}{Miscellaneous}{count}}
\tertiaryStart{\requestsNameD{Miscellaneous}{Miscellaneous}{echo}}
The \requestsNameX{Miscellaneous}{Miscellaneous}{echo} request

			||TBD TBD TBD||

			||/TBD TBD TBD||

\tertiaryEnd{\requestsNameE{Miscellaneous}{Miscellaneous}{echo}}
\tertiaryStart{\requestsNameD{Miscellaneous}{Miscellaneous}{reset}}
The \requestsNameX{Miscellaneous}{Miscellaneous}{reset} request

			||TBD TBD TBD||

			||/TBD TBD TBD||

\tertiaryEnd{\requestsNameE{Miscellaneous}{Miscellaneous}{reset}}
\tertiaryStart{\requestsNameD{Miscellaneous}{Miscellaneous}{stats}}
The \requestsNameX{Miscellaneous}{Miscellaneous}{stats} request

			||TBD TBD TBD||

			||/TBD TBD TBD||

\tertiaryEnd{\requestsNameE{Miscellaneous}{Miscellaneous}{stats}}
\secondaryEnd{}
\primaryEnd{}
