\ProvidesFile{NNservice.tex}[v1.0.0]
\primaryStart[TheNNIService]{The~\NNI{}~Service}
The \asCode{m+mNatNetInputService} application is an Input service,
generating a stream of information on the position and orientation of one or more bodies.
The application responds to the standard Input service requests and can be used as a
standalone data generator, without the need for a client connection.\\

The \emph{configure} request request has no arguments and does nothing.\\

The \emph{restartStreams} request stops and then starts the output stream.\\

The \emph{startStreams} request request initiates listening to the Natural Point NatNet
controller.
Once started, the service will send groups of body data via the output \yarp{} network
connection.\\

The \emph{stopStreams} request stops the Natural Point NatNet controller listener, which
stops the output \yarp{} network connection.\\ 

Note that the application will exit if the \emph{\RS} is not running.
\secondaryStart[StartingFromCommandLine]{Starting from the command\longDash{}line}
\insertAppParameters
\insertTagDescription{\NNI}
\insertInputServiceComment\\

\insertStandardServiceCommands
\secondaryEnd
\condPage
\secondaryStart[StartingFromMMManager]{Starting from \emph{\MMMU}}
If the service is selected for execution from within the \emph{\MMMU} application, the
following dialog will be presented:
\objScaledDiagram{mpm_images/launchNatNetInputService}%
{launchServiceNatNet}{Launch options for the \emph{\NNI} service}{0.8}

For details on the usage of the `endpoint' and `tag' options, see the \emph{\MMMU} manual.
Once the options are set, the following image will appear in the \emph{\MMMU} window when
the service has successfully started:
\objScaledDiagram{mpm_images/runningNatNetInputService}%
{serviceRunningNatNet}{The \emph{\MMMU} entity for the \emph{\NNI} service}{1.0}
\secondaryEnd
\primaryEnd{}
