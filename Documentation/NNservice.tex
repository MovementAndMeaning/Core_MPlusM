\ProvidesFile{NNservice.tex}[v1.0.0]
\primaryStart[TheNNIService]{The~\NNI{}~Service}
The \asCode{m+mNatNetInputService} application is an Input service,
generating a stream of information on the position and orientation of one or more bodies.
The application responds to the standard Input service requests and can be used as a
standalone data generator, without the need for a client connection.\\

The \emph{configure} request has three arguments \longDash{} a string value for the host
name of the Natural Point NatNet controller, an integer value for the command port and an
integer value for the data port on the Natural Point NatNet controller to connect to.
These values will be applied when the output stream is started or restarted.\\ 

The \emph{restartStreams} request stops and then starts the output stream.\\

The \emph{startStreams} request request initiates listening to the Natural Point NatNet
controller.
Once started, the service will send groups of body data via the output \yarp{} network
connection.\\

The \emph{stopStreams} request stops the Natural Point NatNet controller listener, which
stops the output \yarp{} network connection.\\ 

Note that the application will exit if the \emph{\RS} is not running.
\secondaryStart[StartingFromCommandLine]{Starting from the command\longDash{}line}
The application has three optional arguments \longDash{} the host name, command port and
data port for the Natural Point NatNet controller.
\insertAppParameters
\insertTagDescription{\NNI}
\insertInputServiceComment\\

\insertStandardServiceCommands
\secondaryEnd
\secondaryStart[StartingFromMMManager]{Starting from the \emph{\MMMU} application}
If the service is selected for execution from within the \emph{\MMMU} application, the
following dialog will be presented:
\objScaledDiagram{mpm_images/launchNatNetInputService}%
{launchServiceNatNet}{Launch options for the \emph{\NNI} service}{0.8}
\condPage{}
For details on the usage of the `endpoint' and `tag' options, see the \emph{\MMMU} manual.
Once the options are set, the following image will appear in the \emph{\MMMU} window when
the service has successfully started:
\objScaledDiagram{mpm_images/runningNatNetInputService}%
{serviceRunningNatNet}{The \emph{\MMMU} entity for the \emph{\NNI} service}{1.0}

If the \textbf{Configure the service} menu item has been selected, the following will
appear:
\objScaledDiagram{mpm_images/configureNatNetInputService}%
{configureServiceNatNet}{Configuration window for the \emph{\NNI} service}{1.0}
\secondaryEnd
\primaryEnd
\primaryStart[TheOSIService]{The~\NNBI{}~Service}
The \asCode{m+mNatNetBlobInputService} application is an Input service,
generating a stream of information on the position and orientation of one or more bodies.
The application responds to the standard Input service requests and can be used as a
standalone data generator, without the need for a client connection.\\

The \emph{configure} request has four arguments \longDash{} a floating\longDash{}point
value for the translation scale to apply to the joint coordinates, a string value for
the host name of the Natural Point NatNet controller, an integer value for the command
port and an integer value for the data port on the Natural Point NatNet controller to
connect to.
These values will be applied when the output stream is started or restarted.\\ 

The \emph{restartStreams} request stops and then starts the output stream.\\

The \emph{startStreams} request request initiates listening to the Organic Motion
OpenStage controller.
Once started, the service will send groups of body data via the output \yarp{} network
connection.\\

The \emph{stopStreams} request stops the Natural Point NatNet controller listener, which
stops the output \yarp{} network connection.\\ 

Note that the application will exit if the \emph{\RS} is not running.
\secondaryStart[StartingFromCommandLine]{Starting from the command\longDash{}line}
The application has four optional arguments \longDash{} the translation scale to be
applied to the coordinates, the host name, command port and data port for the Natural
Point NatNet controller.
\insertAppParameters
\insertTagDescription{\NNBI}
\insertInputServiceComment\\

\insertStandardServiceCommands
\secondaryEnd
\secondaryStart[StartingFromMMManager]{Starting from the \emph{\MMMU} application}
If the service is selected for execution from within the \emph{\MMMU} application, the
following dialog will be presented:
\objScaledDiagram{mpm_images/launchNatNetBlobInputService}%
{launchServiceNatNetBlob}{Launch options for the \emph{\NNBI} service}{0.8}
\condPage{}
For details on the usage of the `endpoint' and `tag' options, see the \emph{\MMMU} manual.
Once the options are set, the following image will appear in the \emph{\MMMU} window when
the service has successfully started:
\objScaledDiagram{mpm_images/runningNatNetBlobInputService}%
{serviceRunningNatNetBlob}{The \emph{\MMMU} entity for the \emph{\NNBI} service}{1.0}

If the \textbf{Configure the service} menu item has been selected, the following will
appear:
\objScaledDiagram{mpm_images/configureNatNetBlobInputService}%
{configureServiceNatNetBlob}{Configuration window for the \emph{\NNBI} service}{1.0}
\secondaryEnd
\primaryEnd{}
