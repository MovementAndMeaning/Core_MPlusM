\ProvidesFile{MMMnoRegistryService.tex}[v1.0.0]
\appendixStart[RegistryServiceNotRunning]{\textitcorr{What to do if the \emph{\RS} is not
running}}
If the \emph{\RS} is not running when the \emph{\MMMU} application is launched and the
\emph{\RS} executable can't be found, the following will be displayed:
\objScaledDiagram{mpm_images/noRegistryServiceFound}{noRegistryServiceFound}{\emph{\RS}
not running and could not be located}{1.0}

At this point, it's best to exit the \emph{\MMMU} application and reinstall \mplusm, since
the installation is likely to have become damaged.
If, instead, the following is displayed:
\objScaledDiagram{mpm_images/noRunningRegistryService}{noRunningRegistryService}%
{\emph{\RS} not running when application launched}{0.85}

then a \emph{\RS} executable was found and can be launched.
If the \textbf{No} button is clicked, you should exit the \emph{\MMMU} application and
either manually start the \emph{\RS} locally or update the \yarp{} configuration to refer
to an available \yarp{} server that has a running \emph{\RS}.
\condPage{}
If, instead, the \textbf{Yes} button is clicked, the following will be displayed:
\objScaledDiagram{mpm_images/launchRegistryService}{launchRegistryServiceTwo}%
{Launching the \emph{\RS}}{1.0}

Note that only valid network ports can be entered, and it is usually not necessary to use
a non-default port; clicking on the \textbf{Cancel} button will exit the dialog box, with
no \emph{\RS} launched, while clicking on the \textbf{OK} button will start a new
\emph{\RS} in the background, that will be stopped when the \emph{\MMMU} application
exits.
\appendixEnd{}
