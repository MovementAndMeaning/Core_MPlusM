\ProvidesFile{CLfileFormat.tex}[v1.0.0]
\primaryStart{The~Common~Lisp~File~Format}
The \CLF{} service loads a \CL{} file that contains a number of required and optional
values.
Once loaded, the \CL{} functions and values are used to define the behaviour of the
service itself.
\secondaryStart{The~Common~Lisp~Standard~Classes}
The \CLF{} service provides the standard classes of \ECL{} (Embeddable Common Lisp) as part
of the \CL{} environment.\\

Note that, due to some issues with the threading support in \ECL{} and in \yarp{}, the
multithreading support within \ECL{} is not available.
\secondaryEnd
\secondaryStart{Service~Provided~Classes}
The \CLF{} service provides no extra classes as part of the \CL{} environment.
\secondaryEnd
\secondaryStart{Service~Provided~Functions}
The \CLF{} service provides the following functions as part of the \CL{} environment:
\begin{itemize}
\item\textbf{(\asCode{create-inlet-entry} \textit{n} \textit{p} \textit{d} \textit{h})}
\longDash{} creates an inlet description with name `\textit{n}', protocol `\textit{p}',
protocol description `\textit{d}' and handler function `\textit{h}'
\item\exSp\textbf{(\asCode{create-outlet-entry} \textit{n} \textit{p} \textit{d})}
\longDash{} creates an outlet description with name `\textit{n}', protocol `\textit{p}'
and protocol description `\textit{d}'
\item\exSp\textbf{(\asCode{getTimeNow})} \longDash{} returns the current time in seconds, relative
to an arbitrary time in the past
\item\exSp\textbf{(\asCode{requestStop})} \longDash{} signals that the service is to stop
at the first opportunity
\item\exSp\textbf{(\asCode{sendToChannel} \textit{n} \textit{x})} \longDash{} converts the
value `\textit{x}' to \yarp{} format and sends it to the channel numbered `\textit{n}',
with zero being the first outlet channel
\end{itemize}
\secondaryEnd
\secondaryStart{Service~Provided~Values}
The \CLF{} service provides the following values as part of the \CL{} environment:
\begin{itemize}
\item\textbf{\asCode{argv}} \longDash{} an array of the arguments passed to the
script
\item\exSp\textbf{\asCode{tag}} \longDash{} the (optional) tag argument for the service
script
\end{itemize}
\secondaryEnd
\condPage
\secondaryStart{The~Required~Values}
The \CLF{} service requires that the following values are supplied by the \CL{} file:
\begin{itemize}
\item\textbf{\asCode{scriptDescription}} \longDash{} a variable or a function that
provides a string describing the script; if defined as a function, it takes no arguments
\end{itemize}
\secondaryEnd
\secondaryStart{The~Optional~Values}
The \CLF{} service will use the following values, if they are supplied by the \CL{} file:
\begin{itemize}
\item\textbf{\asCode{scriptHelp}} \longDash{} a variable or a function that provides a
string that can be presented to the user when requested by the `\asCode{?}' command; note
that it should not end with a newline; if defined as a function it takes no argument
\item\exSp\textbf{\asCode{scriptInlets}} \longDash{} a variable or a function that
provides an array of inlet descriptions \openSq\asCode{name}, \asCode{protocol},
\asCode{protocolDescription}, \asCode{handler}\closeSq; note that this is ignored if
\asCode{scriptThread} is defined
\item\exSp\textbf{\asCode{scriptInterval}} \longDash{} a variable or a function that
provides the interval between executions of the \asCode{scriptThread} function; note that
this is ignored if \asCode{scriptThread} is not defined, and it is executed only once,
after all the other values have been processed
\item\exSp\textbf{\asCode{scriptOutlets}} \longDash{} a variable or a function that
provides an array of outlet descriptions \openSq\asCode{name}, \asCode{protocol},
\asCode{protocolDescription}\closeSq
\item\exSp\textbf{\asCode{scriptStarting}} \longDash{} a function that is called before
any inlets are attached or threads started
\item\exSp\textbf{\asCode{scriptStopping}} \longDash{} a function that is called after all
the inlets are detached and threads are stopped
\item\exSp\textbf{\asCode{scriptThread}} \longDash{} a function that is repeatedly
called by the output thread of the service
\end{itemize}
\secondaryEnd
\condPage
\secondaryStart{The~Execution~Sequence}
The following sequence of actions occur when a \CL{} file is loaded and run:
\begin{itemize}
\item The \CL{} environment is set up
\item\exSp{}The \CL{} file is read from disk and any global statements are executed
\item\exSp{}The \asCode{scriptDescription} value is retrieved (or the
\asCode{scriptDescription} function is executed to get a value)
\item\exSp{}The \asCode{scriptHelp} value is retrieved, if it is present
\item\exSp{}The \asCode{scriptThread} function is located, if present
\item\exSp{}If there was no \asCode{scriptThread} function defined, the
\asCode{scriptInlets} value is retrieved (or the\\
\asCode{scriptInlets} function is executed to get a value)
\item\exSp{}The \asCode{scriptOutlets} value is retrieved (or the
\asCode{scriptOutlets} function is executed to get a value)
\item\exSp{}The \asCode{scriptStarting} function is located, if present
\item\exSp{}The \asCode{scriptStopping} function is located, if present
\item\exSp{}If there was a \asCode{scriptThread} function defined, the
\asCode{scriptInterval} value is retrieved (or the\\
\asCode{scriptInterval} function is executed to get a value)
\item\exSp{}The required inlets and outlets are created
\item\exSp{}The service is started
\begin{itemize}
\item If the \asCode{scriptStarting} function was defined, it is executed
\item\exSp{}If the \asCode{scriptThread} function was defined, a thread is created and
given the \asCode{scriptThread} function to be executed
\item\exSp{}If the \asCode{scriptThread} function was not defined, input handlers are
created for each inlet and given the \asCode{handler} function from their inlet
description
\end{itemize}
\item\exSp{}The service is stopped
\begin{itemize}
\item If the \asCode{scriptThread} function was defined, the thread is stopped and
destroyed
\item\exSp{}If the \asCode{scriptThread} function was not defined, the input handlers
for each inlet are deactivated
\item\exSp{}If the \asCode{scriptStopping} function was defined, it is executed
\end{itemize}
\end{itemize}
\secondaryEnd
\primaryEnd{}
