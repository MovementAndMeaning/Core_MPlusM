\ProvidesFile{examples.tex}[v1.0.0]
\appendixStart{\textitcorr{Examples}}%
The examples that are part of the \mplusm{} set of source files are provided to support
the development of custom applications.
There are example clients, services, adapters as well as examples of the more specialized
Input, Output and Filter services.\\

All the example applications respond to a `HUP' signal and will exit gracefully when one
is sent to them.
\secondaryStart{Example~Services}
The set of example services include a pair of simple services
[\examplesNameR{Services}{mpmEchoService} and
\examplesNameR{Services}{mpmRandomNumberService}], a service with client context
[\examplesNameR{Services}{mpmRunningSumService}] as well as an input service
[\examplesNameR{Services}{mpmRandomBurstStreamService}], an output service
[\examplesNameR{Services}{mpmRecordIntegersStreamService}] and a filter service
[\examplesNameR{Services}{mpmTruncateFilterStreamService}].\\

Note that services normally have no direct interface.
That is, they are `faceless' applications and perform their operations without user
interaction.
\tertiaryStart{\examplesNameP{Services}{mpmEchoService}}
The \examplesNameX{Services}{mpmEchoService} application responds to
\requestsNameR{Miscellaneous}{Miscellaneous}{echo} requests from the
\examplesNameR{Clients}{mpmEchoClient} application; it simply returns any arguments
sent with the request.\\

Note that the application will exit if the
\serviceNameR[Service~Registry]{ServiceRegistry} is not running.\\

The application has two optional arguments -- an alternative endpoint name to be used and
the port number to be used, if a non--default port is desired; the matching client
application does not directly use the endpoint name to establish a connection to the
service.
\tertiaryEnd{\examplesNameE{Services}{mpmEchoService}}
\tertiaryStart{\examplesNameP{Services}{mpmRandomBurstStreamService}}

			\begin{Large}\textbf{TBD--TBD--TBD}\end{Large}.

\tertiaryEnd{\examplesNameE{Services}{mpmRandomBurstStreamService}}
\tertiaryStart{\examplesNameP{Services}{mpmRandomNumberService}}
The \examplesNameX{Services}{mpmRandomNumberService} application responds to
\requestsNameR{Examples}{Examples}{random} requests from the
\examplesNameR{Clients}{mpmRandomNumberClient} application; it simply returns any arguments
sent with the request.\\

Note that the application will exit if the
\serviceNameR[Service~Registry]{ServiceRegistry} is not running.\\

The application has two optional arguments -- an alternative endpoint name to be used and
the port number to be used, if a non--default port is desired; the matching client
application does not directly use the endpoint name to establish a connection to the
service.\\

The \requestsNameD{Examples}{Examples}{random} request has an argument of the
count of floating--point random numbers to be returned as a response to the request.
\requestsNameE{Examples}{Examples}{random}%
\tertiaryEnd{\examplesNameE{Services}{mpmRandomNumberService}}
\tertiaryStart{\examplesNameP{Services}{mpmRecordIntegersStreamService}}

			\begin{Large}\textbf{TBD--TBD--TBD}\end{Large}.

\tertiaryEnd{\examplesNameE{Services}{mpmRecordIntegersStreamService}}
\tertiaryStart{\examplesNameP{Services}{mpmRunningSumService}}
The \examplesNameX{Services}{mpmRunningSumService} application responds to
requests from the \examplesNameR{Clients}{mpmRunningSumClient} application or\\
\examplesNameR{Adapters}{mpmRunningSumAdapter} or
\examplesNameR{Adapters}{mpmRunningSumAltAdapter} adapters; it returns the running sum
associated with the requesting application or adapter.\\

Note that the application will exit if the
\serviceNameR[Service~Registry]{ServiceRegistry} is not running.\\

The application has two optional arguments -- an alternative endpoint name to be used and
the port number to be used, if a non--default port is desired; the matching client
application does not directly use the endpoint name to establish a connection to the
service.\\

The \requestsNameD{Examples}{Examples}{addtosum} request updates the running sum
corresponding to the requesting application or adapter and returns the resulting value.
Note that an \requestsNameX{Examples}{Examples}{addtosum} request implicitly starts the
calculation.\\
\requestsNameE{Examples}{Examples}{addtosum}%

The \requestsNameD{Examples}{Examples}{resetsum} request clears the running sum
corresponding to the requesting application or adapter.\\
\requestsNameE{Examples}{Examples}{resetsum}%

The \requestsNameD{Examples}{Examples}{startsum} request starts the running sum
calculation for the requesting application or adapter.
Note that a \requestsNameX{Examples}{Examples}{startsum} request implicitly resets the
calculation, if it has already been started.\\
\requestsNameE{Examples}{Examples}{startsum}%

The \requestsNameD{Examples}{Examples}{stopsum} request stops the running sum
calculation for the requesting application or adapter.
\requestsNameE{Examples}{Examples}{stopsum}%
\tertiaryEnd{\examplesNameE{Services}{mpmRunningSumService}}
\tertiaryStart{\examplesNameP{Services}{mpmTruncateFilterStreamService}}

			\begin{Large}\textbf{TBD--TBD--TBD}\end{Large}.

\tertiaryEnd{\examplesNameE{Services}{mpmTruncateFilterStreamService}}
\secondaryEnd{}
\secondaryStart{Example~Clients}
The example client applications provided use simple terminal--based interfaces to
generate requests for their matching service applications.
\tertiaryStart{\examplesNameP{Clients}{mpmEchoClient}}
The \examplesNameX{Clients}{mpmEchoClient} application reads input from a terminal and
sends it to the \examplesNameR{Services}{mpmEchoService} application using an
\requestsNameR{Miscellaneous}{Miscellaneous}{echo} request; it exits when an empty line is
entered.\\

Note that the application will also exit if the
\serviceNameR[Service~Registry]{ServiceRegistry} or the
\examplesNameR{Services}{mpmEchoService} application are not running or if the application
is not running from an interactive terminal.
\tertiaryEnd{\examplesNameE{Clients}{mpmEchoClient}}
\tertiaryStart{\examplesNameP{Clients}{mpmRandomNumberClient}}
The \examplesNameX{Clients}{mpmRandomNumberClient} application reads the count of
random numbers desired from a terminal and sends a
\requestsNameR{Examples}{Examples}{random} request to the
\examplesNameR{Services}{mpmRandomNumberService} application; it exits when a zero is
entered for the count of random numbers desired.\\

Note that the application will also exit if the
\serviceNameR[Service~Registry]{ServiceRegistry} or the
\examplesNameR{Services}{mpmRandomNumberService} application are not running or if the
application is not running from an interactive terminal.
\tertiaryEnd{\examplesNameE{Clients}{mpmRandomNumberClient}}
\tertiaryStart{\examplesNameP{Clients}{mpmRunningSumClient}}
The \examplesNameX{Clients}{mpmRunningSumClient} application reads commands from a
terminal and sends corresponding requests to the
\examplesNameR{Services}{mpmRunningSumService} application.\\

The commands are:
\begin{itemize}
\item \textbf{+:} read a number from the terminal and send it to the service via an
\requestsNameR{Examples}{Examples}{addtosum} request, to update the running sum for this
client.
\item \textbf{r:} send a \requestsNameR{Examples}{Examples}{resetsum} request to the
service so that it will reset the running sum for this client.
\item \textbf{s:} send a \requestsNameR{Examples}{Examples}{startsum} request to the
service so that it will start calculating the running sum for this client.
\item \textbf{x:} send a \requestsNameR{Examples}{Examples}{stopsum} request to the
service so that it will stop calculating the running sum for this client and then exit the
application.
\end{itemize}

Note that the application will also exit if the
\serviceNameR[Service~Registry]{ServiceRegistry} or the
\examplesNameR{Services}{mpmRunningSumService} application are not running or if the
application is not running from an interactive terminal.
\tertiaryEnd{\examplesNameE{Clients}{mpmRunningSumClient}}
\secondaryEnd{}
\secondaryStart{Example~Adapters}
Adapters are applications that use a client object to connect to an \mplusm{} service,
routing input from one or more input \yarp{} network connection via the client object to
the service and returning results from the client to one or more output \yarp{} network
connections.\\

The example adapters demonstrate using a single input \yarp{} network connection
[\examplesNameR{Adapters}{mpmRandomNumberAdapter} and
\examplesNameR{Adapters}{mpmRunningSumAltAdapter}] or multiple input \yarp{} network
connections [\examplesNameR{Adapters}{mpmRunningSumAdapter}].\\

Note that adapters normally have no direct interface.
That is, they are `faceless' applications and perform their operations without user
interaction.
\tertiaryStart{\examplesNameP{Adapters}{mpmRandomNumberAdapter}}
The \examplesNameX{Adapters}{mpmRandomNumberAdapter} application reads the count of
random numbers desired from an input \yarp{} network connection and sends a
\requestsNameR{Examples}{Examples}{random} request to the
\examplesNameR{Services}{mpmRandomNumberService} application.
The random numbers returned from the service are sent out an output \yarp{} network
connection.\\

Note that the application will also exit if the
\serviceNameR[Service~Registry]{ServiceRegistry} or the
\examplesNameR{Services}{mpmRandomNumberService} application are not running.
\tertiaryEnd{\examplesNameE{Adapters}{mpmRandomNumberAdapter}}
\tertiaryStart{\examplesNameP{Adapters}{mpmRunningSumAdapter}}

The \examplesNameX{Adapters}{mpmRunningSumAdapter} application reads  values from an input
`data' \yarp{} network connection and commands from an input `control' \yarp{} network
connection and sends corresponding requests to the
\examplesNameR{Services}{mpmRunningSumService} application.\\

If a numeric value is read from the `data' input \yarp{} network connection, it is sent to
the service via an \requestsNameR{Examples}{Examples}{addtosum} request, to update the
running sum for this adapter.
If a string is read from the `control' input \yarp{} network connection, is is considered
to be one of the following commands:
\begin{itemize}
\item \textbf{q:} exit the application.
\item \textbf{r:} send a \requestsNameR{Examples}{Examples}{resetsum} request to the
service so that it will reset the running sum for this adapter.
\item \textbf{s:} send a \requestsNameR{Examples}{Examples}{startsum} request to the
service so that it will start calculating the running sum for this adapter.
\item \textbf{x:} send a \requestsNameR{Examples}{Examples}{stopsum} request to the
service so that it will stop calculating the running sum for this adapter.
\end{itemize}

The running sums returned from the service are sent out an output \yarp{} network
connection.\\

Note that the application will also exit if the
\serviceNameR[Service~Registry]{ServiceRegistry} or the
\examplesNameR{Services}{mpmRunningSumService} application are not running.
\tertiaryEnd{\examplesNameE{Adapters}{mpmRunningSumAdapter}}
\tertiaryStart{\examplesNameP{Adapters}{mpmRunningSumAltAdapter}}
The \examplesNameX{Adapters}{mpmRunningSumAltAdapter} application reads  values and
commands from an input \yarp{} network connection and sends corresponding requests to the
\examplesNameR{Services}{mpmRunningSumService} application.\\

If a numeric value is read from the input \yarp{} network connection, it is sent to the
service via an \requestsNameR{Examples}{Examples}{addtosum} request, to update the running
sum for this adapter.
If a string is read from the input \yarp{} network connection, is is considered to be one
of the following commands:
\begin{itemize}
\item \textbf{q:} exit the application.
\item \textbf{r:} send a \requestsNameR{Examples}{Examples}{resetsum} request to the
service so that it will reset the running sum for this adapter.
\item \textbf{s:} send a \requestsNameR{Examples}{Examples}{startsum} request to the
service so that it will start calculating the running sum for this adapter.
\item \textbf{x:} send a \requestsNameR{Examples}{Examples}{stopsum} request to the
service so that it will stop calculating the running sum for this adapter.
\end{itemize}

The running sums returned from the service are sent out an output \yarp{} network
connection.\\

Note that the application will also exit if the
\serviceNameR[Service~Registry]{ServiceRegistry} or the
\examplesNameR{Services}{mpmRunningSumService} application are not running.
\tertiaryEnd{\examplesNameE{Adapters}{mpmRunningSumAltAdapter}}
\secondaryEnd{}
\appendixEnd{}
