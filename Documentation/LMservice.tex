\ProvidesFile{LMservice.tex}[v1.0.0]
\primaryStart[TheLMIService]{The~\LMI{}~Service}
The \asCode{m+mLeapMotionInputService} application is an Input service,
generating a stream of information on the position and orientation of one or more hands.
The application responds to the standard Input service requests and can be used as a
standalone data generator, without the need for a client connection.\\

The \emph{configure} request request has no arguments and does nothing.\\

The \emph{restartStreams} request stops and then starts the output stream.\\

The \emph{startStreams} request request initiates listening to the Leap Motion controller.
Once started, the service will send groups of hand data via the output \yarp{} network
connection.\\

The \emph{stopStreams} request stops the Leap Motion controller listener, which stops the
output \yarp{} network connection.\\ 

Note that the application will exit if the \emph{\RS} is not running.
\secondaryStart[StartingFromCommandLine]{Starting from the command\longDash{}line}
\insertAppParameters
\insertTagDescription{\LMI}
\insertInputServiceComment
\condPage{}
If the application is running from a terminal and has not been automatically started via
the `\asCode{go}' option, the following commands are available:
\begin{itemize}
\item\cmdItem{?}{display a list of the available commands}
\item\exSp\cmdItem{b}{start the output stream, sending hand data}
\item\exSp\cmdItem{c}{configure the service; this has no effect, as the service has no
configurable parameters}
\item\exSp\cmdItem{e}{stop the output stream}
\item\exSp\cmdItem{q}{quit the application}
\item\exSp\cmdItem{r}{restart the output stream}
\item\exSp\cmdItem{u}{reset the configuration so that it will be reprocessed when the
output stream is restarted}
\end{itemize}
\secondaryEnd
\condPage
\secondaryStart[StartingFromMMManager]{Starting from \emph{\MMMU}}
If the service is selected for execution from within the \emph{\MMMU} application, the
following dialog will be presented:
\objScaledDiagram{mpm_images/launchLeapMotionInputService}%
{launchServiceLeapMotion}{Launch options for the \emph{\LMI} service}{0.8}

For details on the usage of the `endpoint' and `tag' options, see the \emph{\MMMU} manual.
Once the options are set, the following image will appear in the \emph{\MMMU} window when
the service has successfully started:
\objScaledDiagram{mpm_images/runningLeapMotionInputService}%
{serviceRunningLeapMotion}{The \emph{\MMMU} entity for the \emph{\LMI} service}{1.0}
\secondaryEnd
\primaryEnd
\primaryStart[TheLTFIService]{The~\LTFI{}~Service}
The \asCode{m+mLeapTwoFingersInputService} application is an Input service,
generating a stream of information on the position of the first detected finger of the
first two detected hands.
The application responds to the standard Input service requests and can be used as a
standalone data generator, without the need for a client connection.\\

The \emph{configure} request request has no arguments and does nothing.\\

The \emph{restartStreams} request stops and then starts the output stream.\\

The \emph{startStreams} request request initiates listening to the Leap Motion controller.
Once started, the service will send pairs of finger position data via the output \yarp{}
network connection.\\

The \emph{stopStreams} request stops the Leap Motion controller listener, which stops the
output \yarp{} network connection.\\ 

Note that the application will exit if the \emph{\RS} is not running.
\secondaryStart[StartingFromCommandLine]{Starting from the command\longDash{}line}
\insertAppParameters
\insertTagDescription{\LTFI}
\insertInputServiceComment
\condPage{}
If the application is running from a terminal and has not been automatically started via
the `\asCode{go}' option, the following commands are available:
\begin{itemize}
\item\cmdItem{?}{display a list of the available commands}
\item\exSp\cmdItem{b}{start the output stream, sending hand data}
\item\exSp\cmdItem{c}{configure the service; this has no effect, as the service has no
configurable parameters}
\item\exSp\cmdItem{e}{stop the output stream}
\item\exSp\cmdItem{q}{quit the application}
\item\exSp\cmdItem{r}{restart the output stream}
\item\exSp\cmdItem{u}{reset the configuration so that it will be reprocessed when the
output stream is restarted}
\end{itemize}
\secondaryEnd
\condPage
\secondaryStart[StartingFromMMManager]{Starting from \emph{\MMMU}}
If the service is selected for execution from within the \emph{\MMMU} application, the
following dialog will be presented:
\objScaledDiagram{mpm_images/launchLeapTwoFingersInputService}%
{launchServiceLeapTwoFingers}{Launch options for the \emph{\LTFI} service}{0.8}

For details on the usage of the `endpoint' and `tag' options, see the \emph{\MMMU} manual.
Once the options are set, the following image will appear in the \emph{\MMMU} window when
the service has successfully started:
\objScaledDiagram{mpm_images/runningLeapTwoFingersInputService}%
{serviceRunningLeapTwoFingers}{The \emph{\MMMU} entity for the \emph{\LTFI} service}{1.0}
\secondaryEnd
\primaryEnd{}
\primaryStart[TheLBIService]{The~\LBI{}~Service}
The \asCode{m+mLeapBlobInputService} application is an Input service,
generating a stream of information on the position and direction of the fingers of each
detected hand.
The application responds to the standard Input service requests and can be used as a
standalone data generator, without the need for a client connection.\\

The \emph{configure} request request has a single argument \longDash{} a
floating\longDash{}point value for the translation scale to apply to the finger
coordinates.
The value will be applied when the output stream is started or restarted.\\ 

The \emph{restartStreams} request stops and then starts the output stream.\\

The \emph{startStreams} request request initiates listening to the Leap Motion controller.
Once started, the service will send sets of finger position and direction data via the
output \yarp{} network connection.\\

The \emph{stopStreams} request stops the Leap Motion controller listener, which stops the
output \yarp{} network connection.\\ 

Note that the application will exit if the \emph{\RS} is not running.
\secondaryStart[StartingFromCommandLine]{Starting from the command\longDash{}line}
The application has one optional argument \longDash{} the translation scale to be applied
to the finger coordinates.
\insertAppParameters
\insertTagDescription{\LBI}
\insertInputServiceComment
\condPage{}
If the application is running from a terminal and has not been automatically started via
the `\asCode{go}' option, the following commands are available:
\begin{itemize}
\item\cmdItem{?}{display a list of the available commands}
\item\exSp\cmdItem{b}{start the output stream, sending hand data}
\item\exSp\cmdItem{c}{configure the service by providing the translation scale to be used}
\item\exSp\cmdItem{e}{stop the output stream}
\item\exSp\cmdItem{q}{quit the application}
\item\exSp\cmdItem{r}{restart the output stream}
\item\exSp\cmdItem{u}{reset the configuration so that it will be reprocessed when the
output stream is restarted}
\end{itemize}
\secondaryEnd
\condPage
\secondaryStart[StartingFromMMManager]{Starting from \emph{\MMMU}}
If the service is selected for execution from within the \emph{\MMMU} application, the
following dialog will be presented:
\objScaledDiagram{mpm_images/launchLeapBlobInputService}%
{launchServiceLeapBlob}{Launch options for the \emph{\LBI} service}{0.8}

For details on the usage of the `endpoint' and `tag' options, see the \emph{\MMMU} manual.
Once the options are set, the following image will appear in the \emph{\MMMU} window when
the service has successfully started:
\objScaledDiagram{mpm_images/runningLeapBlobInputService}%
{serviceRunningLeapBlob}{The \emph{\MMMU} entity for the \emph{\LBI} service}{1.0}
\secondaryEnd
\primaryEnd
\primaryStart[TheLTPIService]{The~\LTPI{}~Service}
The \asCode{m+mLeapTwoPalmsInputService} application is an Input service,
generating a stream of information on the position and normal of the palm of the first two
detected hands.
The application responds to the standard Input service requests and can be used as a
standalone data generator, without the need for a client connection.\\

The \emph{configure} request request has no arguments and does nothing.\\

The \emph{restartStreams} request stops and then starts the output stream.\\

The \emph{startStreams} request request initiates listening to the Leap Motion controller.
Once started, the service will send pairs of palm position and normal data via the output
\yarp{} network connection.\\

The \emph{stopStreams} request stops the Leap Motion controller listener, which stops the
output \yarp{} network connection.\\ 

Note that the application will exit if the \emph{\RS} is not running.
\secondaryStart[StartingFromCommandLine]{Starting from the command\longDash{}line}
\insertAppParameters
\insertTagDescription{\LTPI}
\insertInputServiceComment
\condPage{}
If the application is running from a terminal and has not been automatically started via
the `\asCode{go}' option, the following commands are available:
\begin{itemize}
\item\cmdItem{?}{display a list of the available commands}
\item\exSp\cmdItem{b}{start the output stream, sending palm data}
\item\exSp\cmdItem{c}{configure the service; this has no effect, as the service has no
configurable parameters}
\item\exSp\cmdItem{e}{stop the output stream}
\item\exSp\cmdItem{q}{quit the application}
\item\exSp\cmdItem{r}{restart the output stream}
\item\exSp\cmdItem{u}{reset the configuration so that it will be reprocessed when the
output stream is restarted}
\end{itemize}
\secondaryEnd
\condPage
\secondaryStart[StartingFromMMManager]{Starting from \emph{\MMMU}}
If the service is selected for execution from within the \emph{\MMMU} application, the
following dialog will be presented:
\objScaledDiagram{mpm_images/launchLeapTwoPalmsInputService}%
{launchServiceLeapTwoPalms}{Launch options for the \emph{\LTPI} service}{0.8}

For details on the usage of the `endpoint' and `tag' options, see the \emph{\MMMU} manual.
Once the options are set, the following image will appear in the \emph{\MMMU} window when
the service has successfully started:
\objScaledDiagram{mpm_images/runningLeapTwoPalmsInputService}%
{serviceRunningLeapTwoPalms}{The \emph{\MMMU} entity for the \emph{\LTPI} service}{1.0}
\secondaryEnd
\primaryEnd{}
