\ProvidesFile{LMservice.tex}[v1.0.0]
\primaryStart[TheLMIService]{The~\LMI{}~Service}
The \asCode{mpmLeapMotionInputService} application is an Input service,
generating a stream of information on the position and orientation of one or more hands.
The application responds to the standard Input service requests and can be used as a
standalone data generator, without the need for a client connection.\\

The \emph{configure} request request has no arguments and does nothing.\\

The \emph{restartStreams} request stops and then starts the output stream.\\

The \emph{startStreams} request request initiates listening to the Leap Motion controller.
Once started, the service will send groups of hand data via the output \yarp{} network
connection.\\

The \emph{stopStreams} request stops the Leap Motion controller listener, which stops the
output \yarp{} network connection.\\ 

Note that the application will exit if the \emph{Registry~Service} is not running.\\

The application has two optional arguments -- an alternative endpoint name to be used and
the port number to be used, if a non--default port is desired.\\

The application also has two optional parameters:
\begin{itemize}
\item \textbf{-r:} report the service metrics when the application exits
\item \textbf{-t:} specifies the tag to be used as part of the service name
\end{itemize}
The tag is added to the standard name of the service, so that more than one copy of the
service can execute -- an \mplusm{} installation can support multiple copies of each
\inputOutput{} service, but the \emph{ChannelManager} application cannot display them
without a distinguishing `tag'.
If the tag is not specified, the standard name of the service will be used.
As well as the service name, the output stream name is modified if a tag is specified and
the default endpoint is being used.\\

If the application is running from a terminal, the following commands are available:
\begin{itemize}
\item \textbf{?:} display a list of the available commands.
\item \textbf{b:} start the output stream, sending hand data. 
\item \textbf{c:} configure the service; this has no effect, as the service has no
configurable parameters. 
\item \textbf{e:} stop the output stream. 
\item \textbf{q:} quit the application. 
\item \textbf{r:} restart the output stream. 
\item \textbf{u:} reset the configuration so that it will be reprocessed when the output
stream is restarted. 
\end{itemize}
\primaryEnd{}
