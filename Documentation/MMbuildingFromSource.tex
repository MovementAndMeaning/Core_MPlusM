\ProvidesFile{MMbuildingFromSource.tex}[v1.0.0]
\appendixStart[BuildingFromSource]{\textitcorr{Building~\mplusm{}~From~Source}}%
An \mplusm{} installation is normally simply installed according to the
\emph{\MMM{}~Installation} manual.
It is also possible to build a customized \mplusm{} from the source files, which can be
done if the \asCode{Developer} package has also been installed.
\secondaryStart{Prerequisites}
\tertiaryStart{General~prerequisites}
At a minimum, the source files from the \asCode{Developer} package must be present on the
system, along with a \compLang{C++} development environment.
Note that the source files are provided as \textbf{ZIP} files that contain a snapshot of
the \asCode{git} repositories where the source code is stored.
\tertiaryEnd{}
\tertiaryStart{Prerequisites~for~Macintosh~OS~X}
\begin{itemize}
\item\asCode{Xcode} is available from the Macintosh App Store and provides the
\compLang{C++} development environment for Macintosh OS~X
\item\exSp{} The \asCode{Xcode} command\longDash{}line tools must be installed, using the
command `\asCode{xcode-select~{-}{-}install}'
\item\exSp{} The \asCode{ccmake} and \asCode{cmake} command\longDash{}line tools, which
can be installed using \asCode{MacPorts} or directly from the web\longDash{}page
\companyReference{http://www.cmake.org/download}{http://www.cmake.org/download}
\item\exSp{} The \asCode{autoconf213} command\longDash{}line tool, which can be installed
using \asCode{MacPorts}; note that this is only needed if SpiderMonkey is being built
\item\exSp{} The \asCode{Packages} GUI tool, from
\companyReference{http://s.sudre.free.fr/Software/Packages/about.html}{St\'ephane Sudre};
this is only needed if you wish to create installer packages after building \mplusm
\end{itemize}
\tertiaryEnd{}
\tertiaryStart{Prerequisites~for~Microsoft~Windows}
\begin{itemize}
\item\asCode{Visual Studio} is available from Microsoft and provides the \compLang{C++}
development environment for Microsoft Windows; the version used for development is
`\asCode{Visual Studio Express 2012 for Windows Desktop}', but newer versions should also
work
\item\exSp{} The \asCode{cmake} tool, which can be installed from the web\longDash{}page
\companyReference{http://www.cmake.org/download}{http://www.cmake.org/download}
\item\exSp{} An `\asCode{unzip}' program
\end{itemize}
\tertiaryEnd{}
\tertiaryStart{Prerequisites~for~Linux}
\begin{itemize}
\item\TBD{}
\end{itemize}
\tertiaryEnd{}
\secondaryEnd{}
\secondaryStart{Initial~Steps}
\begin{itemize}
\item Create a directory to hold the source code; we'll refer to it from now on as
`\asCode{MPM}'
\item\exSp{} Copy the desired \textbf{ZIP} files from their installation location
(\asCode{/opt/M+M/src} on Macintosh OS~X, \TBD{} for Microsoft Windows and \TBD{} for
Linux) to the new directory \asCode{MPM}
\item\exSp\asCode{unzip} the \textbf{ZIP} files in the directory \asCode{MPM}
\item\exSp{} Remove the \textbf{ZIP} files from the \asCode{MPM} directory
\end{itemize}
\secondaryEnd{}
\secondaryStart{(Optional)~Building~\textbf{ACE}}
\textbf{ACE} (The ADAPTIVE Communication Environment) provides the low\longDash{}level
networking support used within \mplusm{}; the files needed to compile and link with ACE
are installed as part of the normal \mplusm{} installation procedure, so it's not
necessary to rebuild ACE; the source code is provided in case there is an issue with the
headers or libraries.
\tertiaryStart{Building~on~Macintosh~OS~X}
Note that the file \asCode{ACE\fUS{}wrappers2/ace/config.h} will select the correct
configuration file for Macintosh OS~X, Microsoft Windows or Linux, so it does not need to
be modified.
\begin{itemize}
\item On the command\longDash{}line, move to the directory containing the \mplusm{} source
code, \asCode{MPM}
\item\exSp{} Execute the command `\asCode{cd~ACE\fUS{}wrappers2}'
\item\exSp{} Execute the command `\asCode{export~ACE\fUS{}ROOT=\$(pwd)}'
\item\exSp{} Execute the command `\asCode{make~clean~\&\&~make}'; approximate time = 40
minutes
\item\exSp{} If the build is successful, execute the command `\asCode{sudo~sh}'
\item\exSp{} Execute the following commands within the new subshell:
\begin{itemize}
\item\asCode{export~ACE\fUS{}ROOT=\$(pwd)}
\item\exSp\asCode{export~DYLD\fUS{}LIBRARY\fUS{}PATH=/opt/M+M/lib:%
\$DYLD\fUS{}LIBRARY\fUS{}PATH}
\item\exSp\asCode{make~install}
\item\exSp\asCode{exit}
\end{itemize}
\item\exSp{} Execute the command `\asCode{sudo~./after-install.sh}'; this removes some
unneeded files
\end{itemize}
\tertiaryEnd{}
\tertiaryStart{Building~on~Microsoft~Windows}
\TBD{}
\tertiaryEnd{}
\tertiaryStart{Building~on~Linux}
\TBD{}
\tertiaryEnd{}
\secondaryEnd{}
% Note that we can't use \yarp here, since an environment isn't legal here
\secondaryStart{(Optional)~Building~\textbf{YARP}}
\textbf{\yarp} (Yet Another Robot Platform) provides the connection management support
used within \mplusm{}; the files needed to compile and link with \yarp{} are installed as
part of the normal \mplusm{} installation procedure, so it's not necessary to rebuild
\yarp{}; the source code is provided in case there is an issue with the headers or
libraries.
\tertiaryStart{Building~on~Macintosh~OS~X}
\begin{itemize}
\item On the command\longDash{}line, move to the directory containing the \mplusm{} source
code, \asCode{MPM}
\item\exSp{} Execute the command `\asCode{cd~yarp-master}'
\item\exSp{} Execute the command `\asCode{mkdir~build}'
\item\exSp{} Execute the command `\asCode{cd~build}'
\item\exSp{} Execute the command `\asCode{ccmake~..}'; this will open a graphical
interface to the makefile builder. If this is the first time that the command is executed
within a directory, the text `\asCode{EMPTY~CACHE}' will be displayed \longDash{} press
`\textbf{c}' to do the initial configuration; there may be some warnings for project
developers, which can be cleared by pressing `\textbf{e}'
\item\exSp{} Press `\textbf{t}' to show more options and `\textbf{Control-D}' to go to the
second page of options
\item\exSp{} Change the `\textbf{ACE\fUS{}ACE\fUS{}LIBRARY\fUS{}RELEASE}' value to
\asCode{/opt/M+M/lib/libACE.dylib}
\item\exSp{} Change the `\textbf{ACE\fUS{}INCLUDE\fUS{}DIR}' value to
\asCode{/opt/M+M/include}
\item\exSp{} Press `\textbf{t}' to show fewer options
\item\exSp{} Change the `\textbf{CMAKE\fUS{}INSTALL\fUS{}PREFIX}' value to
\asCode{/opt/M+M}
\item\exSp{} Change any other options that you are interested in applying; it is
recommended that the option\\
`\textbf{ENABLE\fUS{}YARPRUN\fUS{}LOG}' should be enabled and the options
`\textbf{CREATE\fUS{}YMANAGER}' and\\
`\textbf{YARP\fUS{}USE\fUS{}READLINE}' should be disabled
\item\exSp{} Press `\textbf{c}' to configure the makefiles; there may be some CMake
warnings for project developers, which can be cleared by pressing `\textbf{e}'
\item\exSp{} Press `\textbf{g}' to generate the makefiles; there may be some CMake
warnings for project developers, which can be cleared by pressing `\textbf{e}'
\item\exSp{} Execute the command `\asCode{cmake~.}'; again, there will be some project
developer warnings, which can be safely ignored
\item\exSp{} Execute the command `\asCode{make~clean~\&\&~make}'; approximate time = 8
minutes
\item\exSp{} Execute the command `\asCode{sudo~../after-build.sh}'; this corrects some
linking issues with \yarp{} caused by installing it in a non\longDash{}default location
\item\exSp{} Execute the command `\asCode{sudo~make~install}'
\item\exSp{} If desired, you can verify that \yarp{} is correctly built be executing the
command `\asCode{make~test}'
\end{itemize}
\tertiaryEnd{}
\tertiaryStart{Building~on~Microsoft~Windows}
\TBD{}
\tertiaryEnd{}
\tertiaryStart{Building~on~Linux}
\TBD{}
\tertiaryEnd{}
\secondaryEnd{}
\secondaryStart{(Optional, used with \textit{Channel~Manager})~Building~\textbf{OGDF}}
\textbf{OGDF} (Open Graph Drawing Framework) provides the tools to layout graphs for the
Channel~Manager GUI application; the files needed to compile and link with OGDF are
installed as part of the normal \mplusm{} installation procedure, so it's not necessary to
rebuild OGDF; the source code is provided in case there is an issue with the headers or
libraries.
\tertiaryStart{Building~on~Macintosh~OS~X}
\begin{itemize}
\item On the command\longDash{}line, move to the directory containing the \mplusm{} source
code, \asCode{MPM}
\item\exSp{} Execute the command `\asCode{cd~OGDF}'
\item\exSp{} Execute the command `\asCode{make~cleanrelease~\&\&~make~release}';
approximate time = 12 minutes
\item\exSp{} Execute the command `\asCode{sudo~make~install}'
\end{itemize}
\tertiaryEnd{}
\tertiaryStart{Building~on~Microsoft~Windows}
\TBD{}
\tertiaryEnd{}
\tertiaryStart{Building~on~Linux}
\TBD{}
\tertiaryEnd{}
\secondaryEnd{}
\secondaryStart{(Optional)~Building~\textbf{SpiderMonkey}}
\textbf{SpiderMonkey}, is the JavaScript engine used with the JavaScript service or
\mplusm{}; the files needed to compile and link with SpiderMonkey are installed as part of
the normal \mplusm{} installation procedure, so it's not necessary to rebuild
SpiderMonkey.
The source files are not provided with the \asCode{Developer} package, but can be obtained
via the instructions on the web\longDash{}page
\companyReference{https://developer.mozilla.org/en-US/docs/Mozilla/Projects/SpiderMonkey/%
Getting\_SpiderMonkey\_source\_code}%
{SpiderMonkey source code}, as \mplusm{} does not use a modified version of SpiderMonkey
and does not depend on a particular version \longDash{} other than it must be at least
SpiderMonkey \textbf{31}.
\tertiaryStart{Building~on~Macintosh~OS~X}
\begin{itemize}
\item On the command\longDash{}line, move to the directory containing the SpiderMonkey
source code, which will be named `\asCode{spidermonkey}'
\item\exSp{} Execute the command `\asCode{cd~js/src}'
\item\exSp{} Execute the command `\asCode{autoconf213}'
\item\exSp{} Execute the command `\asCode{mkdir~\fUS{}obj\fUS}'
\item\exSp{} Execute the command `\asCode{cd~\fUS{}obj\fUS}'
\item\exSp{} Execute the command `\asCode{../configure {-}{-}prefix=/opt/M+M/spidermonkey
{-}{-}with-system-icu\\
{-}{-}enable-release {-}{-}enable-more-deterministic}'
\item\exSp{} Execute the command `\asCode{make~clean~\&\&~make}'; approximate time = 8
minutes
\item\exSp{} Execute the command `\asCode{sudo~make~install}'
\item\exSp{} On the command\longDash{}line, move to the directory containing the \mplusm{}
source code, \asCode{MPM}
\item\exSp{} Execute the command\\
`\asCode{sudo~Core\fUS{}MPlusM-master/replace-links.sh~/opt/M+M/spidermonkey/include/mozjs-}';
this\\
replaces symbolic links in the header directories for SpiderMonkey with the original
items, so that it is safe to distribute the resulting directories
\end{itemize}
\tertiaryEnd{}
\tertiaryStart{Building~on~Microsoft~Windows}
\TBD{}
\tertiaryEnd{}
\tertiaryStart{Building~on~Linux}
\TBD{}
\tertiaryEnd{}
\secondaryEnd{}
\secondaryStart{Building~Core~\mplusm}
\tertiaryStart{Building~on~Macintosh~OS~X}
\begin{itemize}
\item On the command\longDash{}line, move to the directory containing the \mplusm{} source
code, \asCode{MPM}
\item\exSp{} Execute the command `\asCode{cd~Core\fUS{}MPlusM-master}'
\item\exSp{} Execute the command `\asCode{ccmake~.}'; this will open a graphical interface
to the makefile builder. If this is the first time that the command is executed within a
directory, the text `\asCode{EMPTY~CACHE}' will be displayed \longDash{} press
`\textbf{c}' to do the initial configuration
\item\exSp{} Change the `\textbf{CMAKE\fUS{}BUILD\fUS{}TYPE}' to \asCode{Release}
\item\exSp{} Change any other options that you are interested in applying; the default
options minimize the amount of text that is output when \mplusm{} command\longDash{}line
tools are executed
\item\exSp{} Press `\textbf{c}' to configure the makefiles
\item\exSp{} Press `\textbf{g}' to generate the makefiles
\item\exSp{} Execute the command `\asCode{cmake~.}'
\item\exSp{} Execute the command `\asCode{make~clean~\&\&~make}'; approximate time = 7
minutes
\item\exSp{} To run the unit tests for \mplusm{}, do the following:
\begin{itemize}
\item Execute the command `\asCode{ps | grep yarp | grep -v grep}'; if there is a line
similar to `\asCode{30339 ttys004 0:00.01 yarp server\textellipsis}' then the \yarp{}
server is already running
\item\exSp{} If the \yarp{} server is not already running, start a separate terminal
session and launch the \yarp{} server by executing the command `\asCode{yarp~server}'
\item\exSp{} If you see something similar to the following, there is another process using
the network port that \yarp{} uses by default:
\outputBegin{}
\begin{verbatim}
__   __ _    ____  ____  
\ \ / // \  |  _ \|  _ \ 
 \ V // _ \ | |_) | |_) |
  | |/ ___ \|  _ <|  __/ 
  |_/_/   \_\_| \_\_|    
\end{verbatim}
Call with \longDash{}help for information on available options\\
Options can be set on command line or in /Users/M\textunderscore{}M/Library/Application\\
\hspace*{5em}Support/yarp/config/yarpserver.conf\\
Using port database:\ :memory:\\
Using subscription database:\ :memory:\\
IP address:\ default\\
Port number:\ 10000\\
yarp:\ Port /root failed to activate at tcp://10.0.1.2:10000 (address conflict)\\
Name server failed to open
\outputEnd{}
\item\exSp{} If you see the above error message, you will need to clear the \yarp{}
configuration by issuing the command `\asCode{yarp~conf~{-}{-}clean}'.
\item\exSp{} Once the \yarp{} configuration is cleared, execute the command
`\asCode{yarp server \longDash{}socket 11223 \longDash{}write}' to start the \yarp{}
server, where `\asCode{11223}' is an arbitrary number greater than 1024.
\item\exSp{} In the terminal session where \mplusm{} was built, execute the command
`\asCode{make~test}'
\item\exSp{} When the test completes, close the new terminal session if one was started
for the test
\end{itemize}
\item\exSp{} To install the freshly\longDash{}built \mplusm{} files, execute the command
`\asCode{sudo~make~install}'
\item\exSp{} To confirm the normal operation of \mplusm{}, do the following:
\begin{itemize}
\item Check that the \yarp{} server is running and start it if it is not, using the
instructions above
\item\exSp{} Execute the command `\asCode{mpmRegistryService~\&}'; this will launch the
Registry Service in the background and return the \textbf{PID} that can be used to halt it
later
\item\exSp{} Execute the command `\asCode{mpmServiceLister}; this should display the
properties of the Registry Service
\item\exSp{} If you wish to halt the Registry Service, execute the command
`\asCode{halt -s HUP}
\#\#\#\#', where ``\#\#\#\#'' is the \textbf{PID} that was returned when the Registry
Service was launched
\end{itemize}
\end{itemize}
\tertiaryEnd{}
\tertiaryStart{Building~on~Microsoft~Windows}
\TBD{}
\tertiaryEnd{}
\tertiaryStart{Building~on~Linux}
\TBD{}
\tertiaryEnd{}
\secondaryEnd{}
\secondaryStart{Building~\textit{Channel~Manager}}
\textit{Channel~Manager} is not built from the command\longDash{}line but, rather, from
the IDE.
\tertiaryStart{Building~on~Macintosh~OS~X}
\begin{itemize}
\item In Finder, open the directory containing the \mplusm source code, \asCode{MPM}
\item\exSp{} Open the directory `\asCode{Utilities\fUS{}ChannelManager-master}' within
\asCode{MPM}
\item\exSp{} Open the directory `\asCode{Builds/MacOSX}'
\item\exSp{} Open the file `\asCode{Channel Manager.xcodeproj}' with Xcode
\item\exSp{} Within Xcode, select either the `\asCode{Channel Manager Debug}' or
`\asCode{Channel Manager Release}' Scheme
\item\exSp{} Perform the menu item `\asCode{Product \bigRightArrow{} Build}' or
`\asCode{Product \bigRightArrow{} Run}'
\item\exSp{} To create an executable that can be placed in the `\asCode{/Applications}'
folder, do an `\asCode{Archive \bigRightArrow{} Export \bigRightArrow{} Export as a Mac
Application}'
and place the generated application in a writable location, like your home directory
\item\exSp{} Exit from Xcode
\item\exSp{} If the generated application is named \asCode{Channel Manager Debug}, rename
it to \asCode{Channel Manager} and, if it is named \asCode{Channel Manager Release},
rename it to \asCode{Channel Manager}
\item\exSp{} Copy the \asCode{Channel Manager} application to the \asCode{/Applications}
directory
\end{itemize}
\tertiaryEnd{}
\tertiaryStart{Building~on~Microsoft~Windows}
\TBD{}
\tertiaryEnd{}
\tertiaryStart{Building~on~Linux}
\TBD{}
\tertiaryEnd{}
\secondaryEnd{}
\secondaryStart{Creating~the~Distribution~Packages}
The distribution packages are standalone files that can be used to create a complete
\mplusm{} installation.
\tertiaryStart{Creating~on~Macintosh~OS~X}
\begin{itemize}
\item Launch the \asCode{Packages} GUI tool
\item\exSp{} Open the \asCode{Developer} file, which is located in the
\asCode{Packages/Developer} directory within the directory containing the \mplusm{} source
code, \asCode{MPM}
\item\exSp{} Perform the menu item \asCode{Build \bigRightArrow{} Clean\textellipsis}
\item\exSp{} Perform the menu item \asCode{Build \bigRightArrow{} Build}
\item\exSp{} Open the \asCode{M+M} file, which is located in the \asCode{Packages}
directory
\item\exSp{} Perform the menu item \asCode{Build \bigRightArrow{} Clean\textellipsis}
\item\exSp{} Perform the menu item \asCode{Build \bigRightArrow{} Build}
\item\exSp{} The distribution package is the \asCode{M+M.pkg} file located in the
\asCode{Packages/build} directory
\end{itemize}
\tertiaryEnd{}
\tertiaryStart{Creating~on~Microsoft~Windows}
\TBD{}
\tertiaryEnd{}
\tertiaryStart{Creating~on~Linux}
\TBD{}
\tertiaryEnd{}
\secondaryEnd{}
\appendixEnd{}
