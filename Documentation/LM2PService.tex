\ProvidesFile{LM2PService.tex}[v1.0.0]
\primaryStart[TheLTPIService]{The~\LTPI{}~Service}
The \asCode{m+mLeapTwoPalmsInputService} application is an Input service,
generating a stream of information on the position and normal of the palm of the first two
detected hands.
The application responds to the standard Input service requests and can be used as a
standalone data generator, without the need for a client connection.\\

The \emph{configuration} request has no arguments and returns nothing.\\

The \emph{configure} request request has no arguments and does nothing.\\

The \emph{restartStreams} request stops and then starts the output stream.\\

The \emph{startStreams} request request initiates listening to the Leap Motion controller.
Once started, the service will send palm position, normal data and palm velocity via the
output \yarp{} network connection.\\

The \emph{stopStreams} request stops the Leap Motion controller listener, which stops the
output \yarp{} network connection.\\ 

Note that the application will exit if the \emph{\RS} is not running.
\secondaryStart[StartingFromCommandLine]{Starting from the command\longDash{}line}
\insertAppParameters
\insertTagDescription{\LTPI}
\insertInputServiceComment
\condPage
\insertStandardServiceCommands
\secondaryEnd
\condPage
\secondaryStart[StartingFromMMManager]{Starting from the \emph{\MMMU} application}
If the service is selected for execution from within the \emph{\MMMU} application, the
following dialog will be presented:
\objScaledDiagram{mpm_images/launchLeapTwoPalmsInputService}%
{launchServiceLeapTwoPalms}{Launch options for the \emph{\LTPI} service}{0.8}

For details on the usage of the `endpoint' and `tag' options, see the \emph{\MMMU} manual.
Once the options are set, the following image will appear in the \emph{\MMMU} window when
the service has successfully started:
\objScaledDiagram{mpm_images/runningLeapTwoPalmsInputService}%
{serviceRunningLeapTwoPalms}{The \emph{\MMMU} entity for the \emph{\LTPI} service}{1.0}
\secondaryEnd
\primaryEnd{}
