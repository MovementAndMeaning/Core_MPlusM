\ProvidesFile{JSexamples.tex}[v1.0.0]
\appendixStart[Examples]{\textitcorr{JavaScript~Examples}}
\secondaryStart{TruncateFloat.js}
This script reproduces the behaviour of the example service,
\emph{mpmTruncateFloatFilterService}; it processes sequences of floating\longDash{}point
numbers and sends the corresponding integral values.
\codeBegin
\begin{listing}[5]{1}
function doTruncateFloat(portNumber, incomingData)
{
    var aValue;
    var outValues;
    
    if (Array.isArray(incomingData))
    {
        outValues = [];
        for (var ii = 0, mm = incomingData.length; mm > ii; ++ii)
        {
            aValue = Number(incomingData[ii]);
            if (! isNaN(aValue))
            {
                outValues.push(Math.floor(aValue));
            }
        }
    }
    else
    {
        aValue = Number(incomingData);
        if (! isNaN(aValue))
        {
            outValues = Math.floor(aValue);
        }
    }
    sendToChannel(0, outValues);
} // doTruncateFloat

var scriptDescription = 'A script that truncates floating-point numbers';

var scriptInlets = [ { name: 'incoming', protocol: 'd*',
                        protocolDescription: 'A sequence of floating-point numbers',
                        handler: doTruncateFloat } ];

var scriptOutlets = [ { name: 'outgoing', protocol: 'i*',
                        protocolDescription: 'A sequence of integers' } ];
\end{listing}
\codeEnd{}
The function \asCode{doTruncateFloat}(), defined on lines \asCode{1}\longDash\asCode{27},
acts as the input handler for the inlet stream, defined on lines
\asCode{31}\longDash\asCode{33}.
The function \asCode{isNaN}() is used to protect against invalid data received by the
input handler.
\secondaryEnd
\condPage
\secondaryStart{RunningSum.js}
This script reproduces the behaviour of the example service,
\emph{mpmRunningSumService}; it accepts commands (`\asBoldCode{add}' and
`\asBoldCode{reset}') and tallies up the values coming from its inlet.
\codeBegin
\begin{listing}[5]{1}
var runningSum = 0;

function doCommand(aCommand, itsArgs)
{
    var aValue;
    
    switch (aCommand)
    {
        case 'add' :
            if (Array.isArray(itsArgs))
            {
                for (var ii = 0, mm = itsArgs.length; mm > ii; ++ii)
                {
                    aValue = Number(itsArgs[ii]);
                    if (! isNaN(aValue))
                    {
                        runningSum += Number(aValue);
                    }
                }
            }
            else
            {
                aValue = Number(itsArgs);
                if (! isNaN(aValue))
                {
                    runningSum += Number(aValue);
                }
            }
            break;
            
        case 'reset' :
            runningSum = 0;
            break;
            
        default :
            break;
            
    }
} // doCommand

function doRunningSum(portNumber, incomingData)
{
    if (Array.isArray(incomingData))
    {
        var cmd = String(incomingData.shift());
        
        doCommand(cmd, incomingData);
    }
    else
    {
        doCommand(String(incomingData), []);
    }
    sendToChannel(0, runningSum);
} // doRunningSum

var scriptDescription = 'A script that calculates running sums';

var scriptInlets = [ { name: 'incoming', protocol: 'sd*',
                        protocolDescription: 'A command and data',
                        handler: doRunningSum } ];

var scriptOutlets = [ { name: 'outgoing', protocol: 'd',
                        protocolDescription: 'The running sum' } ];
\end{listing}
\codeEnd{}
The function \asCode{doRunningSum}(), defined on lines \asCode{41}\longDash\asCode{54},
acts as the input handler for the inlet stream, defined on lines
\asCode{58}\longDash\asCode{60}.\\

The function \asCode{doCommand}(), defined on lines \asCode{3}\longDash\asCode{39}, parses
the commands sent to the script \longDash{} note the use of string values as \asCode{case}
expressions to simplify the command interpretation and the use of the function
\asCode{isNaN}() to protect against invalid data received by the input handler.
\secondaryEnd
\condPage
\secondaryStart{RecordIntegers.js}
This script reproduces the behaviour of the example service,
\emph{mpmRecordIntegersOutputService}; it writes out the integer values from its inlet to
a text file.
\codeBegin
\begin{listing}[5]{1}
var outStream = null;

function doRecordIntegers(portNumber, incomingData)
{
    var aValue;
    
    if (Array.isArray(incomingData))
    {
        var mm = incomingData.length;
        
        for (var ii = 0; mm > ii; ++ii)
        {
            if (0 < ii)
            {
                outStream.write(' ');
            }
            aValue = Number(incomingData[ii]);
            if (! isNaN(aValue))
            {
                outStream.write(aValue);
            }
        }
        if (0 < mm)
        {
            outStream.writeLine('');
        }
    }
    else
    {
        aValue = Number(incomingData);
        if (! isNaN(aValue))
        {
            outStream.writeLine(aValue);
        }
    }
} // doRecordIntegers

var scriptDescription = 'A script that writes integer values to a file';

var scriptHelp = 'The first argument is the path to the output file';

var scriptInlets = [ { name: 'incoming', protocol: 'i*',
                        protocolDescription: 'A sequence of integer values',
                        handler: doRecordIntegers } ];

function scriptStarting()
{
    var okSoFar = false;
    
    writeLineToStdout('script starting');
    dumpObjectToStdout('argv:', argv);
    if (1 < argv.length)
    {
        var path = argv[1];
        
        writeLineToStdout('path = ' + path);
        outStream = new Stream();
        outStream.open(path, "w");
        dumpObjectToStdout('outStream:', outStream);
        if (outStream.isOpen())
        {
            writeLineToStdout('outStream isOpen = true');
            writeLineToStdout('outStream atEof = ' + outStream.atEof());
            writeLineToStdout('outStream hasError = ' + outStream.hasError());
        }
        okSoFar = true;
    }
    return okSoFar;
} // scriptStarting

function scriptStopping()
{
    writeLineToStdout('script stopping');
    outStream.close();
    outStream = null;
} // scriptStopping
\end{listing}
\codeEnd{}
The function \asCode{doRecordIntegers}(), defined on lines \asCode{3}\longDash\asCode{36},
acts as the input handler for the inlet stream, defined on lines
\asCode{42}\longDash\asCode{44}.\\

The function \asCode{scriptStarting}(), defined on lines \asCode{46}\longDash\asCode{69},
is executed as soon as the script is loaded, and the function \asCode{scriptStopping}(),
defined on lines \asCode{71}\longDash\asCode{76}, is executed after the main function of
the script stops.\\

Note the use of the function \asCode{writeLineToStdout}() as a means of communicating the
script status to the user, the use of the function \asCode{isNaN}() to protect
against invalid data received by the input handler and the presence of a
\asCode{scriptHelp} string, defined on line \asCode{40}, which is used with the \textbf{?}
application command.
\secondaryEnd
\condPage
\secondaryStart{RandomBurst.js}
This script reproduces the behaviour of the example service,
\emph{mpmRandomBurstInputService}; it generates lists of random numbers and sends them to
its outlet.
\codeBegin
\begin{listing}[5]{1}
var burstSize;

var scriptDescription = 'A script that generates random blocks of floating-point numbers';

var scriptHelp = 'Argument 1 is the burst period and argument 2 is the burst size';

// Note that the following function processes both arguments, since it will be called at
// a specific point in the execution sequence.
function scriptInterval()
{
    var interval;
    var inlets = [];
    
    if (1 < argv.length)
    {
        interval = parseInt(argv[1]);
        if (isNaN(interval))
        {
            interval = 1;
        }
        if (2 < argv.length)
        {
            burstSize = parseInt(argv[2]);
            if (isNaN(burstSize))
            {
                burstSize = 1;
            }
        }
        else
        {
            burstSize = 1;
        }
    }
    else
    {
        burstSize = 1;
        interval = 1;
    }
    writeLineToStdout('burst interval is ' + interval + ' and burst size is ' + burstSize);
    return interval;
} // scriptInterval

var scriptOutlets = [ { name: 'output', protocol: 'd+',
                        protocolDescription: 'One or more numeric values' } ];

function scriptThread()
{
    var outList = [];
    
    for (var ii = 0; burstSize > ii; ++ii)
    {
        outList[ii] = (10000 * Math.random());
    }
    sendToChannel(0, outList);
} // scriptThread
\end{listing}
\codeEnd{}
The function \asCode{scriptThread}(), defined on lines \asCode{46}\longDash\asCode{55}, is
executed repetitively, at an interval defined by the output of the function
\asCode{scriptInterval}(), defined on lines \asCode{9}\longDash\asCode{41}.\\

Note that the \asCode{scriptInterval}() function processes the arguments to the script and
records the size of the burst as well as the burst period, before returning the burst
period as its result.
\secondaryEnd
\appendixEnd{}
