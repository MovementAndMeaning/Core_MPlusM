\ProvidesFile{frontispiece.tex}
\startObject{Foreword}{Foreword}

||TBD TBD TBD||

This document is a description of the \MaxName{} objects that I've written over the past
several years, for myself and my friends, and will be a `living' document---I intend to update it as new objects join the family.
It is intended for \MaxName{} programmers, and assumes a basic understanding of how \MaxName{} works---I've made no attempt to explain
standard \MaxName{} objects or how to use my objects with standard \MaxName{} objects, unless there are special considerations for use
of my objects with standard \MaxName{} objects.
I don't provide examples, except in the form of online help files, as some of the objects are more easily understood from
experimentation than exposition.
This is not to say that the objects are difficult to use---I consider myself a lazy \MaxName{} programmer and an experienced
\compLang{C} programmer, so I've attempted to make the objects robust and responsive, with few idiosyncrasies.
Some of the objects may show a \compLang{LISP} or an \compLang{APL} flavour, which is more a reflection on
my languages of choice than an indication that \MaxName{} has or doesn't have these elements.
The objects were written using \companyReference{http://www.metrowerks.com}{Metrowerks} CodeWarrior,
\companyReference{http://www.apple.com}{Apple} Macintosh Programmer's Workshop (MPW), Apple ResEdit\texttrademark{} and the
\MaxName{} Software Development Kit from \companyReference{http://www.cycling74.com}{Cycling '74}.

||/TBD TBD TBD||






But before I get into the goodies, some preliminaries:
\begin{quote}
\begin{small}
Copyright: \copyright{} 2014 by HPlus~Technologies~Ltd. and Simon~Fraser~University.
\\
All rights reserved. Redistribution and use in source and binary forms,
with or without modification, are permitted provided that the following conditions are met:\\
$\bullet$ Redistributions of source code must retain the above copyright notice,
this list of conditions and the following disclaimer.\\
$\bullet$ Redistributions in binary form must reproduce the above copyright notice,
this list of conditions and the following disclaimer in the documentation and/or other materials provided with the distribution.\\
$\bullet$ Neither the name of the copyright holders nor the names of its contributors may be used to endorse
or promote products derived from this software without specific prior written permission.\\
THIS SOFTWARE IS PROVIDED BY THE COPYRIGHT HOLDERS AND CONTRIBUTORS "AS IS" AND ANY EXPRESS OR IMPLIED WARRANTIES,
INCLUDING, BUT NOT LIMITED TO, THE IMPLIED WARRANTIES OF MERCHANTABILITY AND FITNESS FOR A PARTICULAR PURPOSE ARE DISCLAIMED.
IN NO EVENT SHALL THE COPYRIGHT OWNER OR CONTRIBUTORS BE LIABLE FOR ANY DIRECT, INDIRECT, INCIDENTAL, SPECIAL, EXEMPLARY,
OR CONSEQUENTIAL DAMAGES (INCLUDING, BUT NOT LIMITED TO, PROCUREMENT OF SUBSTITUTE GOODS OR SERVICES;
LOSS OF USE, DATA, OR PROFITS; OR BUSINESS INTERRUPTION) HOWEVER CAUSED AND ON ANY THEORY OF LIABILITY, WHETHER IN CONTRACT,
STRICT LIABILITY, OR TORT (INCLUDING NEGLIGENCE OR OTHERWISE) ARISING IN ANY WAY OUT OF THE USE OF THIS SOFTWARE,
EVEN IF ADVISED OF THE POSSIBILITY OF SUCH DAMAGE.

Adobe, the Adobe logo, Acrobat, the Acrobat logo, Distiller, Illustrator, Photoshop and PageMaker are trademarks of
\companyReference{http://www.adobe.com}{Adobe Systems Incorporated}.

Apple, Applescript, Mac, the Mac logo, and Macintosh are trademarks of \companyReference{http://www.apple.com}{Apple Incorporated}.

Xcode copyright \copyright{} 1999--2014 \companyReference{http://www.apple.com}{Apple Incorporated}.

dvips(k) copyright \copyright{} 2013 \companyReference{http://www.radicaleye.com}{Radical Eye Software}.

TextWrangler copyright \copyright{} \companyReference{http://www.barebones.com}{Bare Bones Software Incorporated}.

GPL Ghostscript copyright \copyright{} 2002 \companyReference{http://www.artifex.com}{Artifex Software Incorporated}.

GNU Make copyright \copyright{} 2006 \companyReference{http://www.gnu.org}{Free Software Foundation, Incorporated}.

cmake copyright \copyright{} 2000 \companyReference{http://www.kitware.com}{Kitware Incorporated}.

ACE copyright \copyright{} 1993--2009 \companyReference{http://www.cs.wustl.edu/\%7Eschmidt/ACE.html}{Douglas C. Schmidt}.

YARP copyright \copyright{} 2006 \companyReference{http://wiki.icub.org/yarpdoc/what_is_yarp.html}{RobotCub Consortium}.

JUCE copyright \copyright{} 2013 \companyReference{http://www.juce.com}{Raw Material Software Limited}.

OGDF copyright \copyright{} 2007 \companyReference{http://www.ogdf.net}{Open Graph Drawing Framework}.

doxygen copyright \copyright{} 1997--2014 \companyReference{http://www.stack.nl/~dimitri/doxygen}{Dimitri van Heesch}.

openFrameworks copyright \copyright{} 2005 \companyReference{http://openframeworks.cc}{Zachary Lieberman}.

MacTeX copyright \copyright{} 2005 \companyReference{https://tug.org/mactex}{MacTeX}.

\end{small}
\end{quote}



||TBD TBD TBD||





For each object, I present an image of the object, showing its inlets and outlets, along with the following text entries:
\begin{enumerate}[  1)]
\item A description of the object.
\item The year that the object was created.
\item Whether there is an online help file for the object.
Wherever possible, the help file provides a comprehensive set of examples of use of the object.
For some objects, the help file cannot convey the full complexity of the object---especially for objects with nontrivial state.
For some objects, their command language is too complex to be placed in a help file, without making the help file difficult to use.
\item What theme or category the object belongs to.
I've defined the following groups of objects:
  \begin{enumerate}[a)]
  \item Device interface (\objName{fidget}, \objName{gvp100}, \objName{ldp1550}, \objName{listen}, \objName{mtc}, \objName{rcx},
    \objName{serialX}, \objName{spaceball}, \objName{speak}, \objName{x10})
  \item Miscellaneous (\objName{caseShift}, \objName{dataType}, \objName{fileLogger}, \objName{gcd}, \objName{listType},
    \objName{mtcTrack}, \objName{notX}, \objName{shotgun}, \objName{sysLogger}, \objName{x10units})
  \item Programming aids (\objName{changes}, \objName{compares}, \objName{map1d}, \objName{map2d}, \objName{map3d},
    \objName{memory}, \objName{pfsm}, \objName{queue}, \objName{selectX}, \objName{stack})
  \item QuickTime\texttrademark{} (\objName{bqt}, \objName{wqt})
  \item TCP/IP (\objName{tcpClient}, \objName{tcpLocate}, \objName{tcpMultiServer}, \objName{tcpServer}, \objName{udpPort})
  \item Vector manipulation (\objName{Vabs}, \objName{Vceiling}, \objName{Vcollect}, \objName{Vcos}, \objName{Vdistance},
    \objName{Vdrop}, \objName{Vexp}, \objName{Vfloor}, \objName{Vinvert}, \objName{Vjet}, \objName{Vlength}, \objName{Vlog},
    \objName{Vmean}, \objName{Vnegate}, \objName{Vreduce}, \objName{Vreverse}, \objName{Vrotate}, \objName{Vround},
    \objName{Vscan}, \objName{Vsegment}, \objName{Vsin}, \objName{Vsplit}, \objName{Vsqrt}, \objName{Vtake}, \objName{Vtruncate})
  \end{enumerate}
\item Which class list(s) the object appears in within \MaxName.
\item A description of the arguments, if any, for the object.
Any special values are identified, and default values are identified for optional arguments.
\item A description of each of the inlets for the object.
The domain of values for the inlets are identified.
If the values have a special format, it is described later.
\item A description of each of the outlets for the object.
The range of values for the outlets are identified, where possible.
\item The names of other objects that are normally used with the object.
An example is the \objName{serialX} object, which is used with \objName{x10} and \objName{ldp1550} objects.
\item Whether the object is standalone or communicates with other objects.
The TCP/IP objects \objName{tcpClient}, \objName{tcpMultiServer} and \objName{tcpServer} work with each other;
a \objName{memory} object works with other \objName{memory} objects that have the same tag.
\item Whether the object retains state.
An object that doesn't retain state will respond to a given signal presented to an inlet in the same way each time that
the signal appears;
an object that does retain state, such as a \objName{stack} or \objName{memory} object, may respond differently each time
that the same signal appears.
\item Which versions of Max can use the object (\MaxName{} 3.x, \MaxName{} 4.x or \MaxName{} 4.5+) and which operating systems are supported (OS 9 or OS X).
\item Whether the object can be used only on older Macintoshes (68K-only), Power Macintoshes only (PPC-only) or either (Fat).
Note that the objects were all built to work with \MaxName{} 3.5 or newer.
\item If the object uses a command language, such as the inputs for the \objName{memory} or \objName{x10} objects, it's described in
detail.
\item If the object uses an external file, such as \objName{bqt} or \objName{pfsm} objects, its format is described in detail.
\item Miscellaneous comments or anecdotes.
This is whatever I felt needed to be mentioned, that didn't quite fit in one of the other categories.
\end{enumerate}

For those objects, such as \objName{gvp100}, that require specific connections to other objects, a simple diagram of the non-obvious
connections is provided.
Where it would assist in understanding, I provide a state diagram or a syntax chart.



||/TBD TBD TBD||





\newpage
||TBD TBD TBD||

I'd like to thank the following people for inspiring me to write the objects described here, and for motivating me to actually
document them:\\
My best friend, Thecla Schiphorst, who has helped me find my focus.\\
My friends Sang Mah, Ken Gregory and Grant Gregson, who've presented \MaxName{} programming challenges to me on many occasions, and
who've been my unwitting (but willing) guinea pigs for years.\\
The students of IA308, 2002, \companyReference{http://www.surrey.sfu.ca}{Technical University of British Columbia} (now known as the Simon Fraser University Surrey campus),
for inspiring me and invigorating me.\\
My friends Larry Wasik and Glen Taylor, who've helped me learn how to refine my code, and when refining is not necessary.\\
My friends Jerry Barenholtz, Tom Calvert, Ron Harrop and Doug Seeley for helping me find order and discipline within the chaos
of software design and development.\\
My friends Ron McOuat and Chris Duncombe, who've shown me that a cool head can accomplish great things and overcome obstacles,
both from people and computers.\\
My friends Torsten Belschner, Robb Lovell, Stock and Aadjan van der Helm for providing valuable feedback and ideas about
these objects.

||/TBD TBD TBD||

\vspace{1ex}
||TBD TBD TBD||

The cover page photograph was originally made by Larry Wasik.
It's the card deck for the program CONTK, along with a (partial) listing of the source code.
CONTK, if you're interested, was a process control program written in \compLang{FORTRAN IV} for the \companyReference{http://ibm1130.org/related}{IBM 1800} Process Control Computer,
to control a Kamyr digester.
If those terms aren't familiar to you, don't be surprised---I included the photograph as an attempt at humour, reflecting on
my early days of programming.

||/TBD TBD TBD||





\vspace{1ex}
\begin{flushleft}
Norm Jaffe,\\
Vancouver, British Columbia, Canada\\
\today
\end{flushleft}

\vspace{\fill}

This document was created using \companyReference{https://tug.org/mactex}{MacTeX}-2014,
\companyReference{http://www.barebones.com}{TextWrangler} 4.5.9,
\companyReference{http://www.artifex.com}{GPL Ghostscript} 9.10,
\companyReference{http://www.gnu.org}{GNU Make version} 3.81,
\companyReference{http://www.tug.org/metapost.html}{MetaPost} 1.902 and
\companyReference{http://www.radicaleye.com}{dvips(k)} 5.994 on an Apple iMac.

\objEnd{}
