\ProvidesFile{IMinstalling.tex}[v1.0.0]
\primaryStart[InstallationProcedure]{Installing \mplusm}
\secondaryStart{\mac}
The files needed for an \mplusm{} installation are all contained within the
\asCode{m+m.pkg} file; note that Administrator privileges are needed to perform the
installation.
\tertiaryStart{Installing the packages}
After obtaining the \asCode{m+m-\MMVn.pkg} file, open it by double\longDash{}clicking, at
which point the \asEmphCode{Introduction} page is presented.
\objScaledPicture{mpm_images/Installer01}{0.75}

Click on the \asBoldCode{Continue} button to proceed.
\condPage{}
The \asEmphCode{Read~Me} page will now appear, which describes what software tools are
needed if you wish to do development with \mplusm{} \longDash{} these are not part of the
\asCode{M+M.pkg} file and will need to be installed separately.
\objScaledPicture{mpm_images/Installer02}{0.75}
Determine whether you have the software tools mentioned and click on the
\asBoldCode{Continue} button to proceed, or the \asBoldCode{Print\textellipsis} or
\asBoldCode{Save\textellipsis} to record the other software tools that you would need for
development.
\condPage{}
The \asEmphCode{License} page will now appear; please read it and click on the
\asBoldCode{Continue} button to proceed, or the \asBoldCode{Print\textellipsis} or
\asBoldCode{Save\textellipsis} buttons to record the license.
\objScaledPicture{mpm_images/Installer03}{0.75}
Once you click on the \asBoldCode{Continue} button, you will be asked to confirm that you
agree to the license; click on the \asBoldCode{Agree} button to accept the license, on the
\asBoldCode{Disagree} button to stop the installation process or on the
\asBoldCode{Read~License} button to go back to the \asEmphCode{License} page.
If you agree to the license, the installation will proceed.
\objScaledPicture{mpm_images/Installer04}{0.75}
\condPage{}
The \asEmphCode{Installation~Type} page will now appear \longDash{} at this point you can
select which items will be installed and where to install them.
\objScaledPicture{mpm_images/Installer05}{0.75}
Click on the \asBoldCode{Change~Install~Location\textellipsis} button to specify the disk
drive where the \mplusm{} packages will be installed, on the \asBoldCode{Customize} button
if you wish to select the packages that will be installed, on the \asBoldCode{Go~Back}
button if you would like to review the license and on the \asBoldCode{Install} button if
you are ready to proceed with the installation.
\condPage{}
If you clicked on the \asBoldCode{Change~Install~Location\textellipsis} button on the
\asEmphCode{Installation~Type} page, the\\
\asEmphCode{Select~a~Destination} page will appear.
You can select any disk drive that contains the \mac{} operating system on it.
Typically, only one disk drive is selectable; the current operating system disk will be
selected if there is sufficient space to do the installation.
The amount of disk space required to install the packages that you have selected will also
appear.
\objScaledPicture{mpm_images/Installer05a}{0.75}
Select a disk drive by clicking on it and then click on the \asBoldCode{Continue} button
to return to the \asEmphCode{Installation~Type} page.
\condPage{}
If you clicked on the \asBoldCode{Customize} button on the \asEmphCode{Installation~Type}
page, the \asEmphCode{Custom~Install} page will appear.
\objScaledPicture{mpm_images/Installer06}{0.75}
Click on the \asBoldCode{Standard~Install} button to return to the
\asEmphCode{Installation~Type} page, with just the \asCode{Core} package selected, on the
\asBoldCode{Go~Back} button to return to the \asEmphCode{Installation~Type} page with the
current selections and on the \asBoldCode{Install} button to proceed with the
installation.
\condPage{}
If you clicked on the \asCode{Core} package, a description of the package appears.
The \asCode{Core} package is always installed \longDash{} if you have an existing
\mplusm{} installation, the Action shown will be `Upgrade', not `Install'.
\objScaledPicture{mpm_images/Installer06a}{0.75}
\condPage{}
If you clicked on the \asCode{Developer} package, a description of the package appears.
The \asCode{Developer} package is optional \longDash{} if you have an existing \mplusm{}
installation that includes the \asCode{Developer} package, the Action shown will be
`Upgrade', not `Install'.
The default Action is to `Skip' the installation of the \asCode{Developer} package.
\objScaledPicture{mpm_images/Installer06b}{0.75}
Note that the amount of disk space required to the install the packages will be adjusted
to reflect if the \asCode{Developer} package has been selected for installation.
\condPage{}
Clicking on the \asBoldCode{Install} button on the \asEmphCode{Installation~Type} page
will result in a page being displayed which requests an administrator name and password.
\objScaledPicture{mpm_images/Installer07}{0.75}
Clicking on the \asBoldCode{Cancel} button will return you to the
\asEmphCode{Installation~Type} page; entering a valid administrator name and password and
clicking on the \asBoldCode{Install~Software} button will result in a series of pages
being displayed, reflecting the installation progress.
\condPage{}
Once the installation completes, the \asEmphCode{Summary} page will appear.
\objScaledPicture{mpm_images/Installer08}{0.75}
Click on the \asBoldCode{Close} button to exit from the installer.
\condPage
\tertiaryEnd
\tertiaryStart{Testing the installation}
Open the \asCode{Terminal} program and enter the following:
\outputBegin
\begin{verbatim}
ps | grep yarp | grep -v grep
\end{verbatim}
\outputEnd{}
If there is a line similar to `\asCode{30339 ttys004 0:00.01 yarp server\textellipsis}'
then the \yarp{} server is already running. If not, enter the following:
\outputBegin
\begin{verbatim}
yarp server
\end{verbatim}
\outputEnd{}
If you then see something similar to the following, there is another process using the
network port that \yarp{} uses by default:
\outputBegin
\begin{verbatim}
__   __ _    ____  ____  
\ \ / // \  |  _ \|  _ \ 
 \ V // _ \ | |_) | |_) |
  | |/ ___ \|  _ <|  __/ 
  |_/_/   \_\_| \_\_|    

Call with --help for information on available options
Options can be set on command line or in /Users/M_M/Library/Application
       Support/yarp/config/yarpserver.conf
Using port database: :memory:
Using subscription database: :memory:
IP address: default
Port number: 10000
yarp: Port /root failed to activate at tcp://10.0.1.2:10000 (address conflict)
Name server failed to open
\end{verbatim}
\outputEnd{}
If you chose to stop the running \yarp{} server in order to use the version installed with
\mplusm{} or if you see the above error message, you will need to clear the \yarp{}
configuration before issuing the `\asCode{yarp~server}' command.
The following command will clear the \yarp{} configuration:
\outputBegin
\begin{verbatim}
yarp conf --clean
\end{verbatim}
\outputEnd{}
If there was an address conflict due to another process using \yarp's default port, use
the following command to start the \yarp{} server:
\outputBegin
\begin{verbatim}
yarp server --socket 11223 --write
\end{verbatim}
\outputEnd{}
where `\asCode{11223}' is an arbitrary number greater than 1024.
Note that \mplusm{} has been tested with \yarp{} versions 2.3.61 through 2.3.63 and should
work with newer versions.
To identify the version of \yarp{} that is installed, issue the command:
\outputBegin
\begin{verbatim}
yarp version
\end{verbatim}
\outputEnd{}
If you see the following, the \asCode{Terminal} program was running during the
installation and did not pick up the correct settings:
\outputBegin
\begin{verbatim}
-bash: yarp: No such file or directory
\end{verbatim}
\outputEnd{}
Exit from the \asCode{Terminal} program and open it again, so that its settings will be
updated.\\

If you see the following, \yarp{} started correctly, and the \mplusm{} installation can be
confirmed:
\outputBegin
\begin{verbatim}
__   __ _    ____  ____  
\ \ / // \  |  _ \|  _ \ 
 \ V // _ \ | |_) | |_) |
  | |/ ___ \|  _ <|  __/ 
  |_/_/   \_\_| \_\_|    

Call with --help for information on available options
Options can be set on command line or in /Users/M_M/Library/Application
       Support/yarp/config/yarpserver.conf
Using port database: :memory:
Using subscription database: :memory:
IP address: default
Port number: 10000
yarp: Port /root active at tcp://10.0.1.2:10000

Registering name server with itself:
 * register "/root" tcp "10.0.1.2" 10000
   + set "/root" ips "127.0.0.1" "10.0.1.2" "25.142.170.49"
   + set "/root" process 90494
 * register fallback mcast "224.2.1.1" 10000
   + set fallback ips "127.0.0.1" "10.0.1.2" "25.142.170.49"
   + set fallback process 90494
Name server can be browsed at http://10.0.1.2:10000/

Ok.  Ready!
\end{verbatim}
\outputEnd{}
Once you've confirmed that \yarp{} is running correctly, it's time to start \mplusm.
Start another \asCode{Terminal} session or open another tab in the \asCode{Terminal}
application and enter the following:
\outputBegin
\begin{verbatim}
m+mRegistryService &
\end{verbatim}
\outputEnd{}
The output similar to the following should appear:
\outputBegin
\begin{verbatim}
[1] 63918
registered admin's iMac._yarpns._tcp.local.
\end{verbatim}
\outputEnd{}
where `\asBoldCode{1}' will be incremented each time that a process is run in the
background, `\asBoldCode{63918}' will be replaced by the process identifier for the copy
of \asCode{m+mRegistryService} that is running and `\asCode{admin's~iMac}' will be
replaced by the name of the computer on which \mplusm{} has been installed.\\

If something similar to the following appears:
\outputBegin
\begin{verbatim}
[1] 64234
[1]+ Done        m+mRegistryService
\end{verbatim}
\outputEnd{}
then the \emph{m+mRegistryService} application was already running when \mplusm{} was
installed, indicating that this was not a new installation but, rather, an upgrade.
\condPage{}
It will be necessary to stop the earlier copy of \emph{m+mRegistryService}.
To to this, find the running copy via the command:
\outputBegin
\begin{verbatim}
ps -ax | grep -i m+m | grep -v grep
\end{verbatim}
\outputEnd{}
A single line of output should appear, similar to:
\outputBegin
\begin{verbatim}
63918 ttys015   16:42.19 m+mRegistryService
\end{verbatim}
\outputEnd{}
where `\asBoldCode{63918}' will be the process identifier of the already\longDash{}running
copy of \emph{m+mRegistryService}.
Enter the following command, replacing `\asBoldCode{63918}' with the first number that
appeared in the output that was just displayed.
\outputBegin
\begin{verbatim}
kill -s HUP 63918
\end{verbatim}
\outputEnd{}
It should now be possible to issue the `\asCode{m+mRegistryService~\&}' command as
described earlier.\\

The last step in testing the \mplusm{} installation is to launch the \emph{\MMMU}
application, which was placed in the \asCode{/Applications}
directory.
You should see something similar to the following:
\objScaledPicture{mpm_images/Runningm+mManager}{1.0}
You now have a basic \mplusm{} installation and can begin using it for your projects!
\tertiaryEnd
\secondaryEnd
\condPage
\secondaryStart{\win}
To install \mplusm, simply execute the installer package titled
``\asCode{m+m-\MMVn-win64.exe}''.
Depending on your security settings, you may receive a User Access Warning.
Simply click the \asBoldCode{Allow access} button if that is the case.
You will be greeted with the following window:
\objScaledPicture{mpm_images/win_install01}{0.75}
Click the \asBoldCode{Next~>} button to proceed.
You will be greeted with the following window:
\objScaledPicture{mpm_images/win_install02}{0.75}
Click the \asBoldCode{I~Agree} button to proceed.
\condPage{}
You will now have the option of adding \mplusm{} to a \textbf{PATH} environment variable.
This is highly recommended, as it will make launching the applications provided with
\mplusm{} much easier.
Selecting \emph{Create M+M Desktop Icon} will have no effect, as \mplusm{} consists of a
number of applications, rather than the single application that the installer expects.
\objScaledPicture{mpm_images/win_install03}{0.75}
Select the \textbf{PATH} option that you would prefer and click the \asBoldCode{Next~>}
button to proceed.
\objScaledPicture{mpm_images/win_install04}{0.75}
Select the destination for the installed files and click the \asBoldCode{Next~>} button to
proceed.
\condPage
\objScaledPicture{mpm_images/win_install05}{0.75}
An option to create a Start Menu folder will be provided.
Currently this will contain a shortcut to the uninstaller application.
After making your selection, click the \asBoldCode{Next~>} button to proceed.
\objScaledPicture{mpm_images/win_install06}{0.75}
Here you can select the components to be installed.
The \asBoldCode{Core} selection contains just the libraries, standard applications and
header files, while the \asBoldCode{Optional} selection contains the example applications
and source code.
Select what you wish to install and click the \asBoldCode{Install} button to proceed.
\condPage{}
The installer will now proceed to place the elements of \mplusm{} on the system.
\objScaledPicture{mpm_images/win_install07}{0.75}
If this is the first time that \mplusm{} is being installed, the following may be
displayed, as \mplusm{} is built with \asCode{Visual Studio} from Microsoft and needs
some libraries to be installed as well:
\objScaledPicture{mpm_images/win_install08}{0.75}
After choosing to accept the license agreement, click on the \asBoldCode{Install} button
to proceed.
\condPage{}
Once the Microsoft libraries are placed on the system, the following window will appear.
\objScaledPicture{mpm_images/win_install09}{0.75}
Click on the \asBoldCode{Close} button to complete the installation of the Microsoft
libraries.\\

The \mplusm{} Library makes use of the \emph{Apple Bonjour SDK}, and its installer will be
launched at this point:
\objScaledPicture{mpm_images/win_install10}{0.75}
Click the \asBoldCode{Next~>} button to proceed.
\objScaledPicture{mpm_images/win_install11}{0.75}
After choosing to accept the license agreement, click on the \asBoldCode{Next~>} button to
proceed.
\objScaledPicture{mpm_images/win_install12}{0.75}
Please read the system requirements to ensure that you can safely install the \emph{Apple
Bonjour SDK} and then click on the \asBoldCode{Next~>} button to proceed.
\condPage{}
\objScaledPicture{mpm_images/win_install13}{0.75}
At this point, you can select an alternative location to install the \emph{Apple Bonjour
SDK}, if you wish.
Click on the \asBoldCode{Install} button to proceed once you've made your selection.\\

After a short while, the following window will appear:
\objScaledPicture{mpm_images/win_install14}{0.75}
Click on the \asBoldCode{Finish} button to return to the \mplusm{} installer.
\objScaledPicture{mpm_images/win_install15}{0.75}
Click on the \asBoldCode{Finish} button to exit from the \mplusm{} installer \longDash{}
the installation is complete.
\condPage
\tertiaryStart{Testing the \win{} installation}
For the most part, testing the \win{} installation is similar to the \mac{} version,
except most of the applications can be executed directly by double clicking the icons in
the \asCode{/bin} folder of the installation directory.
If command\longDash{}line arguments are required (e.g. for clearing of the \yarp{}
configuration file), it can be done by opening a \win{} ``command prompt'' and
\asBoldCode{cd}ing into the \asCode{/bin} folder.\\

One other thing to note, is that the first time you open any of these network
applications, a warning may be generated by \win{} Firewall as follows:
\objScaledPicture{mpm_images/win_run01}{0.75}
Please click on the \asBoldCode{Allow access} button to enable the networking features of
each application.
\tertiaryEnd
\secondaryEnd
\primaryEnd{}
