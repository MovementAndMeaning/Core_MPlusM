\ProvidesFile{KV2specialMessageFormat.tex}[v1.0.0]
\appendixStart[KV2specialFormat]{\textitcorr{\KVtwoSI{} Message Format}}
Each time that the Kinect V2 controller reports that there is information
available, the \asCode{KinectV2SpecialInputService} packages it into a single message
\openSq\asCode{Bottle}\closeSq, containing the following:
\begin{itemize}
\item A sequence of bodies
\end{itemize}

For each valid body, a sequence of values is added:
\begin{itemize}
\item\textbf{index} \longDash{} the index of the body, starting at zero
\item\textbf{lefthand} \longDash{} the state of the left hand (\textbf{3} =
\asCode{closed}, \textbf{2} = \asCode{open}, \textbf{4} = \asCode{lasso})
\item\exSp\textbf{righthand} \longDash{} the state of the right hand
\item\exSp\textbf{lefthandconfidence} \longDash{} the confidence in the state of the left
hand (\textbf{0} = \asCode{low}, \textbf{1} = \asCode{high})
\item\exSp\textbf{righthandconfidence} \longDash{} the confidence in the state of the
right hand
\item\exSp\textbf{joints} \longDash{} a sequence of values for the following joints:
\textbraceleft{} \emph{head}, \emph{neck}, \emph{spineshoulder}, \emph{spinemid},
\emph{spinebase},\\
\emph{shoulderright}, \emph{shoulderleft}, \emph{hipright}, \emph{hipleft},
\emph{elbowright}, \emph{wristright}, \emph{handright}, \emph{handtipright},
\emph{thumbright},\\
\emph{elbowleft}, \emph{wristleft}, \emph{handleft}, \emph{handtipleft}, \emph{thumbleft},
\emph{kneeright}, \emph{ankleright}, \emph{footright}, \emph{kneeleft},
\emph{ankleleft},\\
\emph{footleft} \textbraceright
\end{itemize}

For each joint, there are two possibilities:
\begin{itemize}
\item\textbf{not present} \longDash{} the value \textbf{0}, followed by seven
floating\longDash{}point zeroes
\item\exSp\textbf{present} \longDash{} the value \textbf{1}, followed by the
three\longDash{}dimensional coordinates \openSq{}X, Y, Z\closeSq{} of the position of the
joint, as a sequence of three floating\longDash{}point numbers \longDash{} the units are
most likely metres from the Kinect V2 controller \longDash{} followed by the
four\longDash{}dimensional coordinates \openSq{}x, y, z, w\closeSq{} of the orientation of
the joint, as a sequence of four floating\longDash{}point numbers
\end{itemize}
\appendixEnd{}
