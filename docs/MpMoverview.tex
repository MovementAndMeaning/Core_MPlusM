\ProvidesFile{MpMoverview.tex}[v1.0.0]
\primaryStart{Overview}
\textbf{Movement~and~Meaning~Middleware} (also known as \mplusm) is a software system
that acts as an intermediary between subsystems that provide sensor data, such as
accelerometers and motion capture cameras, and actuators such as projectors and sound
systems.
It provides mechanisms for reporting and interrogating the protocols used by the sensors
and actuators, as well as a standard architecture for creating services.\\

\mplusm{} is available as a standalone installer for \mac{} or \win{}, or it can be built
from source; for \linux{}, \mplusm{} (currently) must be built from source.\\

\mplusm{} requires \mac{} 10.7 or later, \win{} 10 or Ubuntu \linux{} 15.x or later, on a
64-bit Intel processor, and roughly 500MB of disk space, if building from source.\\

An \mplusm{} installation consists of a set of programs that interconnect using the
\companyReference{http://wiki.icub.org/yarpdoc/what_is_yarp.html}{\yarp} networking
protocols, along with libraries that can be linked to applications to provide access to
the \mplusm{} features.\\

There are three main classes of programs in the set \longDash{} services, clients and
utilities.
Clients use a formalized protocol to connect to services and utilities manage or monitor
the aggregate state of an \mplusm{} configuration.
There is a unique service, the \serviceNameR[\RS]{RegistryService}, that maintains
information on all the active services that are accessible to clients within an \mplusm{}
installation.
The primary utility is the \emph{\MMMU} application, which provides a GUI\longDash{}based
view of the state of connections, services and clients within the installation.
It is described in its own manual, \emph{MMManagerUtility\fUS{}Manual.pdf}.\\

Each service can support multiple client connections, and the client functionality can be
embedded in command\longDash{}line tools, GUI\longDash{}based applications or headless
background processes.\\

There are subtypes of services as well as of clients, distinguished by the presence of
additional \yarp{} network connections and their characteristics.\\

Basic services have only the client~/~service \yarp{} connection and provide a central
resource\longDash{}sharing facility; simple services, such as the
\serviceNameR[\RS]{RegistryService}, have one or more additional \yarp{} network
connections that have no special properties and \inputOutput{} services have one or more
additional \yarp{} network connections with specific protocols.
\inputOutput{} services provide a mechanism for `packaging' sensors and actuators as
\mplusm{} resources.\\

Basic clients have only the client~/~service \yarp{} network connection while adapters
have one or more additional \yarp{} network connections.
Note that connections to services or clients that don't involve the \yarp{} network are
considered to be intrinsic to the process involved and are not `visible' to \mplusm{}
applications.\\

Adapters provide a mechanism for non\longDash\mplusm{} applications to make requests of
\mplusm{} services via \yarp{} network connections.
\primaryEnd{}
