\ProvidesFile{MpMupdatingVersionNumber.tex}[v1.0.0]
\appendixStart[UpdatingVersionNumber]{\textitcorr{Updating the Version Number}}
When a significant change is made to the source files for \mplusm, such as the addition of
new core messages or methods, the version number should be updated to reflect the
change.\\

The version number is recorded in a number of locations:
\begin{itemize}
\item The documentation version number is located in the \asCode{commonStrings.tex} file
in the \asCode{docs} directory in the \asCode{Core\fUS{}MPlusM} directory
\longDash{} look for the `\asCode{\textbackslash{}MMVn}' macro
\item\exSp{}The version number is also referred to in the description of the output of
the \utilityNameR{m+mVersion} application \longDash{} it appears in three `verbatim'
sections in the \asCode{MpMutilities.tex} file in the \asCode{docs} directory in the
\asCode{Core\fUS{}MPlusM} directory
\item\exSp{}The primary location for the source code version number is in the
\asCode{CMakeLists.txt} file in the \asCode{Core\fUS{}MPlusM} directory; the
`\asCode{set(MpM\fUS{}VERSION\fUS{}MAJOR)}', `\asCode{set(MpM\fUS{}VERSION\fUS{}MINOR}'
and `\asCode{set(MpM\fUS{}VERSION\fUS{}PATCH)}' commands are used to construct the version
number
\item\exSp{}The \emph{\MMMU} version number is located in the \asCode{Xcode} project
settings \longDash{} make sure to change both of the \asCode{Version} and \asCode{Build}
settings in the \asCode{General} settings for the application target \longDash{} and the
\asCode{m+mmResources.rc} file, where the `\asCode{FILEVERSION}', `\asCode{FileVersion}'
and `\asCode{ProductVersion}' values will need to be modified
\item\exSp{}The \emph{m+mLeapDisplayOutputService} version number is located in the
\asCode{Xcode} project settings \longDash{} make sure to change both of the
\asCode{Version} and \asCode{Build} settings in the \asCode{General} settings for the
application target \longDash{} and the \asCode{m+mmResources.rc} file, where the
`\asCode{FILEVERSION}', `\asCode{FileVersion}' and `\asCode{ProductVersion}' values will
need to be modified
\item\exSp{}The \emph{m+mPlatonicDisplayOutputService} version number is located in the
\asCode{Xcode} project settings \longDash{} make sure to change both of the
\asCode{Version} and \asCode{Build} settings in the \asCode{General} settings for the
application target \longDash{} and the \asCode{m+mmResources.rc} file, where the
`\asCode{FILEVERSION}', `\asCode{FileVersion}' and `\asCode{ProductVersion}' values will
need to be modified
\item\exSp{}The distribution build numbers for \mac{} are changed via the
\asCode{Packages} GUI tool \longDash{} update the \asCode{Developer} file, located in the
\asCode{Core\fUS{}MPlusM/Packages/Developer} directory within the \asCode{devel}
directory, as well as the \asCode{m+m} file, located in the
\asCode{Core\fUS{}MPlusM/Packages} directory
\item\exSp{}The \asCode{doxygen} configuration files should also be updated; the two
files are \asCode{Core\fUS{}MPlusM/Doxyfile} and
\asCode{Utilities\fUS{}mm\fUS{}manager/Doxyfile}
\end{itemize}
To update to the new version number, once the files have been edited, perform the normal
build process.\\

When changing the version number, the first and second sections of the number should only
be modified when substantial modifications have been made - significant enough that users
will need to review the source code or documentation; normally only the third section of
the number is updated.
\appendixEnd{}



