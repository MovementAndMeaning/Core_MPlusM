\ProvidesFile{CLexamples.tex}[v1.0.0]
\appendixStart[CLexamples]{\textitcorr{\CL{} Examples}}
\secondaryStart{TruncateFloat.lisp}
This script reproduces the behaviour of the example service,
\emph{m+mTruncateFloatFilterService}; it processes sequences of floating\longDash{}point
numbers and sends the corresponding integral values.
\codeBegin
\begin{listing}[5]{1}
(defun truncAValue (inValue)
  (if (numberp inValue) (truncate inValue) inValue))
  
(defun doTruncateFloat (portNumber incomingData)
  (let* (outValues)
    (if (arrayp incomingData)
	(let ((outSize (array-dimension incomingData 0)))
	  (setq outValues (make-array (list outSize)))
	  (dotimes (ii outSize)
	    (setf (aref outValues ii) (truncAValue (aref incomingData ii)))))
      (setq outValues (truncAValue incomingData)))
    (sendToChannel 0 outValues)))

(setq scriptDescription "A script that truncates floating-point numbers")

(setq scriptInlets (make-array '(1) :initial-element
			       (create-inlet-entry "incoming" "d+"
						   "A sequence of floating-point numbers"
						   'doTruncateFloat)))

(setq scriptOutlets (make-array '(1) :initial-element
				(create-outlet-entry "outgoing" "i+"
						     "A sequence of integers")))
\end{listing}
\codeEnd{}
The function \asCode{doTruncateFloat}, defined on lines \asCode{4}\longDash\asCode{12},
acts as the input handler for the inlet stream, defined on lines
\asCode{16}\longDash\asCode{19}.
\secondaryEnd
\condPage
\secondaryStart{RunningSum.lisp}
This script reproduces the behaviour of the example service,
\emph{m+mRunningSumService}; it accepts commands (`\asBoldCode{addtosum}',
`\asBoldCode{quit}', `\asBoldCode{resetsum}', `\asBoldCode{startsum}' and
`\asBoldCode{stopsum}') and tallies up the values coming from its inlet.
\codeBegin
\begin{listing}[5]{1}
(setq summing nil)
(setq runningSum 0)

(defun do-a-command (aCommand)
  (cond ((numberp aCommand)
	 (incf runningSum (if summing aCommand 0)))
	((stringp aCommand)
	 (setq bCommand (string-downcase aCommand))
	 (cond ((string= "addtosum" bCommand))
	       ((string= "quit" bCommand) (requestStop))
	       ((string= "resetsum" bCommand) (setq runningSum 0))
	       ((string= "startsum" bCommand) (setq summing t))
	       ((string= "stopsum" bCommand) (setq summing nil))
	       (t (format t "unrecognized~%"))))
	(t (format t "no match for ~A~%" aCommand))))

(defun do-input-sequence (incomingData)
  (cond
   ((stringp incomingData)
    (do-a-command incomingData))
   ((arrayp incomingData)
    (dotimes (ii (array-dimension incomingData 0))
      (do-input-sequence (aref incomingData ii))))
   (t (do-a-command incomingData))))

(defun doRunningSum (portNumber incomingData)
  (do-input-sequence incomingData)
  (sendToChannel 0 runningSum))

(setq scriptDescription "A script that calculates running sums")

(setq scriptInlets (make-array '(1) :initial-element
			       (create-inlet-entry "incoming" "sd*" "A command and data"
						   'doRunningSum)))

(setq scriptOutlets (make-array '(1) :initial-element
				(create-outlet-entry "outgoing" "d" "The running sum")))
\end{listing}
\codeEnd{}
The function \asCode{doRunningSum}, defined on lines \asCode{26}\longDash\asCode{28},
acts as the input handler for the inlet stream, defined on lines
\asCode{32}\longDash\asCode{34}.\\

The function \asCode{do-a-command}, defined on lines \asCode{4}\longDash\asCode{15},
parses the commands sent to the script.
\secondaryEnd
\condPage
\secondaryStart{RecordIntegers.lisp}
This script reproduces the behaviour of the example service,
\emph{m+mRecordIntegersOutputService}; it writes out the integer values from its inlet to
a text file.
\codeBegin
\begin{listing}[5]{1}
(setq out-stream nil)

(defun doRecordIntegers (portNumber incomingData)
  (if (arrayp incomingData)
      (let ((outSize (array-dimension incomingData 0)))
	(dotimes (ii outSize)
	  (format out-stream (if (< 0 ii) " ~A" "~A") (aref incomingData ii)))
	(if (< 0 outSize) (format out-stream "~%")))
    (format out-stream "~A~%" incomingData)))

(setq scriptDescription "A script that writes integer values to a file")

(setq scriptHelp "The first argument is the path to the output file")

(setq scriptInlets (make-array '(1) :initial-element
			       (create-inlet-entry "incoming" "i+"
						   "A sequence of integer values"
						   'doRecordIntegers)))

(defun scriptStarting ()
  (format t "script starting~%")
  (if (< 1 (array-dimension mmcl:argv 0))
      (let ((file-path (aref mmcl:argv 1)))
	(setq out-stream (open file-path :direction :output
			       :if-exists :overwrite
			       :if-does-not-exist :create))
	(if (open-stream-p out-stream)
	    (format t "out-stream open~%")
	  (setq out-stream nil))
	t)
    nil))

(defun scriptStopping ()
  (format t "script stopping~%")
  (if out-stream (close out-stream)))
\end{listing}
\codeEnd{}
The function \asCode{doRecordIntegers}, defined on lines \asCode{3}\longDash\asCode{9},
acts as the input handler for the inlet stream, defined on lines
\asCode{15}\longDash\asCode{18}.\\

The function \asCode{scriptStarting}, defined on lines \asCode{20}\longDash\asCode{31},
is executed as soon as the script is loaded, and the function \asCode{scriptStopping},
defined on lines \asCode{33}\longDash\asCode{35}, is executed after the main function of
the script stops.\\

Note the presence of a \asCode{scriptHelp} string, defined on line \asCode{13}, which is
used with the \textbf{?} application command.
\secondaryEnd
\condPage
\secondaryStart{RandomBurst.lisp}
This script reproduces the behaviour of the example service,
\emph{m+mRandomBurstInputService}; it generates lists of random numbers and sends them to
its outlet.
\codeBegin
\begin{listing}[5]{1}
(setq burstSize 1)

(setq scriptDescription "A script that generates random blocks of floating-point numbers")

(setq scriptHelp
      "The first argument is the burst period and the second argument is the burst size")

(defun numberOr1 (inValue)
  (if (numberp inValue) inValue 1))

;; Note that the following function processes both arguments, since it will be called at a
;; specific point in the execution sequence.
(defun scriptInterval ()
  (let* ((argCount (array-dimension mmcl:argv 0)) interval)
    (if (< 1 argCount)
	(setq interval (numberOr1 (let* ((*read-default-float-format* 'double-float))
				    (read-from-string (aref mmcl:argv 1))))
	      burstSize (if (< 2 argCount)
			    (numberOr1 (parse-integer (aref mmcl:argv 2) :junk-allowed t))
			  1))
      (setq burstSize 1 interval 1))
    interval))

(setq scriptOutlets (make-array '(1) :initial-element
				(create-outlet-entry "outgoing" "d+"
						     "One or more numeric values")))

(defun scriptThread ()
  (let* (outList)
    (setq outList (make-array (list burstSize)))
    (dotimes (ii burstSize)
      (setf (aref outList ii) (* 10000 (random 1.0))))
    (sendToChannel 0 outList)))
\end{listing}
\codeEnd{}
The function \asCode{scriptThread}, defined on lines \asCode{28}\longDash\asCode{33}, is
executed repetitively, at an interval defined by the output of the function
\asCode{scriptInterval}, defined on lines \asCode{13}\longDash\asCode{22}.\\

Note that the \asCode{scriptInterval} function processes the arguments to the script and
records the size of the burst as well as the burst period, before returning the burst
period as its result.
\secondaryEnd
\appendixEnd{}
